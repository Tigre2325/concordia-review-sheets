\documentclass[10pt, twocolumn]{article}

%%%%%%%%%%%%%%%%%%%%%%%%%%%%%%%%%%%%%%%%%%%%%%%%%%%%%%%%%%%%%%%%%%%%%%%%%%%%%%%
%%%% Cover page
\title{AERO 201: Introduction to Flight and Aerospace Systems}
\date{\today}
\author{Anthony Bourboujas}

\makeatletter
\let\Title\@title
\let\Author\@author
\let\Date\@date
\makeatother

%%%%%%%%%%%%%%%%%%%%%%%%%%%%%%%%%%%%%%%%%%%%%%%%%%%%%%%%%%%%%%%%%%%%%%%%%%%%%%%
%%%% Preamble
%%%%%%%%%%%%%%%%%%%%%%%%%%%%%%%%%%%%%%%%%%%%%%%%%%%%%%%%%%%%%%%%%%%%%%%%%%%%%%%
%%%% Packages
\usepackage[utf8x]{inputenc} % Accept different input encodings
\usepackage[T1]{fontenc} % Standard package for selecting font encodings
\usepackage{lmodern} % Font name; classic: lmodern
\usepackage[english]{babel} % Multilingual support for LaTeX
% \usepackage{abstract} % Control the typesetting of the abstract environment
\usepackage{amsmath} % AMS mathematical facilities for LaTeX
\usepackage{amssymb} % TeX fonts from the American Mathematical Society
\usepackage{amsthm} % Typesetting theorems (AMS style)
\usepackage{array} % Extending the array and tabular environments
% \usepackage[backend=biber,style=ieee,sorting=none]{biblatex}
\usepackage{bold-extra} % Use bold small caps and typewriter fonts
\usepackage{cellspace} % Ensure minimal spacing for table cells
\usepackage{chemformula} % Command for typesetting chemical formulas and reactions
% \usepackage{colortbl} % Add colour to LaTeX tables
\usepackage{comment} % Selectively include/exclude portions of text
\usepackage{csquotes} % Context sensitive quotation facilities
% \usepackage[en-US,showdow]{datetime2} % Formats for dates, times and time zones
% \usepackage{diagbox} % Table heads with diagonal lines
\usepackage{enumitem} % Control layout of itemize, enumerate, description
\usepackage{esint} % Extended set of integrals for Computer Modern
\usepackage{graphicx} % Enhanced support for graphics
% \usepackage{listings} % Typeset source code listings using LaTeX
% \usepackage{lipsum} % Easy access to the Lorem Ipsum dummy text
\usepackage{mathrsfs} % Support for using RSFS fonts in maths
% \usepackage{matlab-prettifier} % Pretty-print Matlab source code
\usepackage{moreverb} % Extended verbatim
\usepackage{multicol} % Intermix single and multiple columns
\usepackage{multirow} % Create tabular cells spanning multiple rows
% \usepackage{pgfplots} % Plots
% \usepackage{pgfplotstable} % Loads, rounds, format and post-processes numerical tables (generates table from CSV)
% \usepackage{pdfpages} % Include PDF document in LaTeX
% \usepackage{rotating} % Rotation tools, including rotated full-page floats with sidewaysfigure
\usepackage[scr]{rsfso} % A mathematical calligraphic font based on rsfs
\usepackage{setspace} % Set space between lines
\usepackage{soul} % Hyphenation for letterspacing, underlining, and more
\usepackage{threeparttable} % Tables with captions and notes all the same width
% \usepackage{verbatim} % Reimplementation of and extensions to LaTeX verbatim
\usepackage{wrapfig} % Produces figures which text can flow around
\usepackage{xcolor} % Driver-independent color extensions for LaTeX
\usepackage{xurl} % Verbatim with URL-sensitive line breaks, allow URL breaks at any alphanumerical character

%%%%%%%%%%%%%%%%%%%%%%%%%%%%%%%%%%%%%%%%%%%%%%%%%%%%%%%%%%%%%%%%%%%%%%%%%%%%%%%
%%%% Lengths
% 1cm = 10mm = 28pt = 1/2.54in
% 1ex = height of a lowercase 'x' in the current font
% 1em = width of an uppercase 'M' in the current font

%%%% Spacing in math mode
% \!                         = -3/18em
% \,                         = 3/18em
% \:                         = 4/18em
% \;                         = 5/18em
% \ (space after backslash!) = space in normal text
% \quad                      = 1em
% \qquad                     = 2em

% \setlength{\baselineskip}{1em} % Vertical distance between lines in a paragraph
% \renewcommand{\baselinestretch}{1.0} % A factor multiplying \baslineskip
\setlength{\columnsep}{0.75cm} % Distance between columns
% \setlength{\columnwidth}{} % The width of a column
\setlength{\columnseprule}{1pt} % The width of the vertical ruler between columns
% \setlength{\evensidemargin}{} % Margin of even pages, commonly used in two-sided documents such as books
% \setlength{\linewidth}{} % Width of the line in the current environment.
% \setlength{\oddsidemargin}{} % Margin of odd pages, commonly used in two-sided documents such as books
% \setlength{\paperwidth}{} % Width of the page
% \setlength{\paperheight}{} % Height of the page
\setlength{\parindent}{0cm} % Paragraph indentation
\setlength{\parskip}{6pt} % Vertical space between paragraphs
% \setlength{\tabcolsep}{} % Separation between columns in a table (tabular environment)
% \setlength{\textheight}{} % Height of the text area in the page
% \setlength{\textwidth}{} % Width of the text area in the page
% \setlength{\topmargin}{} % Length of the top margin
\setlist{
  %%%% Vertical spacing
  topsep = 0pt,
  partopsep = 0pt,
  parsep = 0pt,
  itemsep = 0pt,
  %%%% Horizontal spacing
  leftmargin = 0.5cm,
  rightmargin = 0cm,
  % listparindent = 0cm,
  % labelwidth = 0cm,
  % labelsep = 0cm,
  % itemindent = 0cm
}
\addtolength{\cellspacetoplimit}{2pt}
\addtolength{\cellspacebottomlimit}{2pt}

%%%%%%%%%%%%%%%%%%%%%%%%%%%%%%%%%%%%%%%%%%%%%%%%%%%%%%%%%%%%%%%%%%%%%%%%%%%%%%%
%%%% Page layout
\usepackage{layout} % View the layout of a document
\usepackage{geometry} % Flexible and complete interface to document dimensions
% 1cm = 10mm = 28pt = 1/2.54in
% ex = height of a lowercase 'x' in the current font
% em = width of an uppercase 'M' in the current font
\geometry{
  a4paper,
  top         = 1cm,
  bottom      = 1cm,
  left        = 1.5cm,
  right       = 1.5cm,
  includehead = true,
  includefoot = true,
  landscape   = false, % Paper orientation
  twoside     = false,
}
% \geometry{showframe} % Show paper outline for the text area and page

%%%%%%%%%%%%%%%%%%%%%%%%%%%%%%%%%%%%%%%%%%%%%%%%%%%%%%%%%%%%%%%%%%%%%%%%%%%%%%%
%%%% Header and footer style
\usepackage{fancyhdr} % Extensive control of page headers and footers in LaTeX
\pagestyle{fancy}
% Options: \leftmark (chapter title), \rightmark(section title), \thepage (page number), \thechapter(chapter number), \thesection (section number)
\lhead{\thetitle}
\chead{}
\rhead{}
\lfoot{}
\cfoot{\thepage}
\rfoot{}

%%%%%%%%%%%%%%%%%%%%%%%%%%%%%%%%%%%%%%%%%%%%%%%%%%%%%%%%%%%%%%%%%%%%%%%%%%%%%%%
%%%% URL insertion settings
\usepackage{hyperref} % Extensive support for hypertext in LaTeX
\definecolor{black}{RGB}{0, 0, 0} % rgb(0, 0, 0)
\definecolor{blue}{RGB}{0, 0, 255} % rgb(0, 0, 255)
\hypersetup{
  % unicode            = true,
  pdftitle           = {\thetitle},
  pdfauthor          = {\theauthor},
  % pdfsubject       = {},
  %%%% Reference
  % bookmarks          = true,
  bookmarksnumbered  = true,
  bookmarksopen      = true, % Open the bookmarks
  bookmarksopenlevel = 2, % Open until 1 level (section)
  %%%% Bookmarks
  breaklinks         = true,
  pdfborder          = {0 0 0},
  % backref            = true, % Add links into bibliography
  % pagebackref        = true,
  % hyperindex         = true, % Add links into index
  %%%% Color
  colorlinks         = true,
  linkcolor          = black, % Internal links color
  citecolor          = black,
  urlcolor           = blue, % Hyperlinks color
  filecolor          = black,
}

\usepackage{varioref} % Intelligent page reference
\usepackage[capitalise,noabbrev]{cleveref}
\usepackage{prettyref} % Make label references "self-identity" with \prettyref{#1}
\newrefformat{cha}{chapter \textbf{\nameref{#1}} \vpageref{#1}} % {chapter \textbf{\nameref{#1}} on page \pageref{#1}}
\newrefformat{sec}{section \textbf{\nameref{#1}} \vpageref{#1}} % {section \textbf{\nameref{#1}} on page \pageref{#1}}
% \newrefformat{fig}{\vref{#1}} % {Figure \ref{#1} on page \pageref{#1}}
% \newrefformat{tab}{\vref{#1}} % {Table \ref{#1} on page \pageref{#1}}
% \newrefformat{eqn}{\vref{#1}}
% \newrefformat{lis}{\emph{\nameref{#1}} \vpageref{#1}}

%%%%%%%%%%%%%%%%%%%%%%%%%%%%%%%%%%%%%%%%%%%%%%%%%%%%%%%%%%%%%%%%%%%%%%%%%%%%%%%
%%%% Physics units settings
% Dependencies
\usepackage{booktabs} % Publication quality tables in LaTeX
\usepackage{caption} % Customizing captions in floating environments
\usepackage{helvet} % Load Helvetica, scaled
\usepackage{cancel} % Place lines through maths formulae

\usepackage{siunitx} % A comprehensive (SI) units package
\sisetup{
  exponent-product     = \cdot, % Symbol between number and power of ten
  group-minimum-digits = 5, % Number of digits when 3 digits separation appear
  % inter-unit-product   = \cdot, % Symbol between units (when several units are used)
  output-complex-root  = \ensuremath{i}, % How i math should be seen
  % prefixes-as-symbols  = false, % Translate prefixes (kilo, centi, milli, micro,...) into a power of ten
  separate-uncertainty = true, % Write uncertainty with +-
  scientific-notation  = engineering,
}

%%%%%%%%%%%%%%%%%%%%%%%%%%%%%%%%%%%%%%%%%%%%%%%%%%%%%%%%%%%%%%%%%%%%%%%%%%%%%%%
%%%% Theorems and proofs
\numberwithin{equation}{section}
% \makeatletter
% \g@addto@macro\th@remark{\thm@headpunct{:}}
% \makeatother
\theoremstyle{remark}
\newtheorem*{example}{Example}
\newtheorem*{remark}{Remark}

%%%%%%%%%%%%%%%%%%%%%%%%%%%%%%%%%%%%%%%%%%%%%%%%%%%%%%%%%%%%%%%%%%%%%%%%%%%%%%%
%%%% User-defined environments
% Remove the space before the enumerate and itemize environments
\let\oldenumerate\enumerate % Keep a copy of \enumerate (or \begin{enumerate})
\let\endoldenumerate\endenumerate % Keep a copy of \endenumerate (or \end{enumerate})
\renewenvironment{enumerate}{
  \begin{oldenumerate}
    \vspace{-6pt}
    }{
  \end{oldenumerate}
}

\let\olditemize\itemize % Keep a copy of \itemize (or \begin{itemize})
\let\endolditemize\enditemize % Keep a copy of \enditemize (or \end{itemize})
\renewenvironment{itemize}{
  \begin{olditemize}
    \vspace{-6pt}
    }{
  \end{olditemize}
}

\let\olddescription\description % Keep a copy of \description (or \begin{description})
\let\endolddescription\enddescription % Keep a copy of \enddescription (or \end{description})
\renewenvironment{description}{
  \begin{olddescription}
    \vspace{-6pt}
    }{
  \end{olddescription}
}

%%%%%%%%%%%%%%%%%%%%%%%%%%%%%%%%%%%%%%%%%%%%%%%%%%%%%%%%%%%%%%%%%%%%%%%%%%%%%%%
%%%% User-defined commands
\newcommand{\Romannumeral}[1]{\MakeUppercase{\romannumeral #1}} % Capital roman numbers
% \newcommand{\gui}[1]{\og #1 \fg{}} % French quotation marks
\renewcommand{\thefootnote}{[\arabic{footnote}]}

%%% Figure command
%% Include SVG files
\newcommand{\executeiffilenewer}[3]{
  \ifnum\pdfstrcmp{\pdffilemoddate{#1}}
    {\pdffilemoddate{#2}}>0
    {\immediate\write18{#3}}\fi
}
\newcommand{\includesvg}[1]{
  \executeiffilenewer{#1.svg}{#1.pdf}
  {
    % Inkscape must be installed in PATH and the user must include '--shell-escape' in the build arguments
    inkscape #1.svg --export-type=pdf --export-latex
  }
  \input{#1.pdf_tex}
}

%%% Math commands
%% Tables (requires cellspace package)
\newcolumntype{L}{>{\(\displaystyle}Cl<{\)}} % Column type for left-aligned math column
\newcolumntype{D}{>{\(\displaystyle}Cc<{\)}} % Column type for centered math column

%% Functions
\newcommand{\constant}{\mathrm{constant}} % Constant
\newcommand{\abs}[1]{\left| #1 \right|} % Absolute function
\newcommand{\erf}[1]{\mathrm{erf} \left( #1 \right)} % Error function
\newcommand{\erfc}[1]{\mathrm{erfc} \left( #1 \right)} % Complementary error function
\newcommand{\unitstep}[1]{\,\mathcal{U}\left( #1 \right)} % Unit step function
\newcommand{\diracdelta}[2]{\,\delta_{#1}\left( #2 \right)} % Dirac delta function


%% Derivatives and integrals
\newcommand{\diff}[2]{\mathrm{d}^{#1} #2} % Letter 'd' of differentials
\newcommand{\diffint}[1]{\,\diff{}{#1}} % Differential with a space for integrals
\newcommand{\derivative}[2]{\frac{\diff{}{#1}}{\diff{}{#2}}} % Derivative
\newcommand{\nderivative}[3]{\frac{\diff{#1}{#2}}{\diff{}{#3^{#1}}}} % Derivative of degree n
\newcommand{\partialderivative}[2]{\frac{\partial #1}{\partial #2}} % Partial derivative
\newcommand{\npartialderivative}[3]{\frac{\partial^{#1} #2}{\partial #3^{#1}}} % Partial derivative of degree n
\newcommand{\direcderivative}[2]{D_{\vec{#1}}\,#2} % Directional derivative

\newcommand{\Laplace}[1]{\mathcal{L}\left\{ #1 \right\}} % Laplace transform notation
\newcommand{\invLaplace}[1]{\mathcal{L}^{-1}\left\{ #1 \right\}} % Inverse Laplace transform notation

%% Set
\newcommand{\set}[3]{\mathbb{#1}_{#2}^{#3}} % Set of numbers
\newcommand{\integerset}{\mathbb{Z}} % Set of integer numbers (compatibility)
\newcommand{\realset}{\mathbb{R}} % Set of real numbers (compatibility)

%% Limits
\newcommand{\limit}[3]{\lim_{#1 \to #2}{#3}} % Limit from a point to another
\newcommand{\rlimit}[3]{\lim_{#1 \to #2^{+}}{#3}} % Right limit from a point to another
\newcommand{\llimit}[3]{\lim_{#1 \to #2^{-}}{#3}} % Left imit from a point to another
\newcommand{\modulus}[1]{\,\left[ #1 \right]} % Modulus notation

%% Vectors
\newcommand{\ivec}{\hat{\mathrm{i}}} % i vector
\newcommand{\jvec}{\hat{\mathrm{j}}} % j vector
\newcommand{\kvec}{\hat{\mathrm{k}}} % k vector
\renewcommand{\Vec}[1]{\overrightarrow{#1}} % Vector notation for expression with more than one letter
\newcommand{\norm}[1]{\left\| #1 \right\|} % Norm notation for expression with just one letter
\newcommand{\normvec}[1]{\left\| \vec{#1} \right\|} % Norm notation for expression with just one letter
\newcommand{\Normvec}[1]{\left\| \Vec{#1} \right\|} % Norm notation for expression with more than one letter
\newcommand{\comp}[2]{\mathrm{comp}_{\vec{#2}}\vec{#1}} % Components
\newcommand{\proj}[2]{\mathrm{proj}_{\vec{#2}}\vec{#1}}
\newcommand{\grad}[1]{\vec{\nabla}#1} % Gradient notation
\newcommand{\frames}[2]{\left( #1 \right)_{#2}} % Frame definition

\newcommand{\curl}[1]{\mathrm{curl}\,\vec{#1}} % Curl of a vector field
\newcommand{\divergence}[1]{\mathrm{div}\,\vec{#1}} % Divergence of a vector field


%%%%%%%%%%%%%%%%%%%%%%%%%%%%%%%%%%%%%%%%%%%%%%%%%%%%%%%%%%%%%%%%%%%%%%%%%%%%%%%
%%%% Beginning of the document
\begin{document}
\maketitle % Insert the cover page
% \tableofcontents
% \layout % Show a drawing of page layout
% \renewcommand{\abstractname}{} % Change the abstract title

\section{Formulas}
\paragraph{Bernoulli equation}
Any increase in fluid velocity corresponds to a decrease in pressure (Venturi effect)
\begin{align*}
  \diff{}{P}                            & = - \rho v \diff{}{v}                    \\
  \iff P_1 + \frac{1}{2} \rho_1 {v_1}^2 & = P_2 + \frac{1}{2} \rho_2 {v_2}^2 = P_0
\end{align*}
\[
  \begin{array}{|l}
    P [\si{\pascal}] \text{: flow pressure}           \\
    v [\si{\metre\per\second}] \text{: flow velocity} \\
    \rho [\si{\kilogram\per\metre\cubed}] \text{: flow density}
  \end{array}
\]

\paragraph{Continuity equation}
Mass can neither be created nor destroyed, thus
\begin{align*}
  \dot{m}_1           & = \dot{m}_2      \\
  \iff A_1 v_1 \rho_1 & = A_2 v_2 \rho_2
\end{align*}
\[
  \begin{array}{|l}
    \dot{m} [\si{\kilogram\per\second}] \text{: mass flow} \\
    A [\si{\metre\squared}] \text{: flow area}             \\
    v [\si{\metre\per\second}] \text{: flow velocity}      \\
    \rho [\si{\kilogram\per\metre\cubed}] \text{: flow density}
  \end{array}
\]

\paragraph{Excess power}
\[
  P_\mathrm{excess} = v(T - D) = W v_\mathrm{climb}
\]
\[
  \begin{array}{|l}
    P_\mathrm{excess} [\si{\watt}] \text{: excess power}  \\
    v [\si{\meter\per\second}] \text{: aircraft velocity} \\
    T [\si{\newton}] \text{: thrust}                      \\
    D [\si{\newton}] \text{: drag}                        \\
    W [\si{\newton}] \text{: aircraft weight}             \\
    v_\mathrm{climb} [\si{\metre\per\second}] \text{: rate of climb}
  \end{array}
\]

\paragraph{Density in non-isothermal region}
\[
  \rho = \rho_\mathrm{ref} \left( \frac{T}{T_\mathrm{ref}} \right)^{-\frac{g}{aR_\mathrm{air}} + 1}
\]
\[
  \begin{array}{|l}
    \rho [\si{\kilogram\per\metre\cubed}] \text{: density at the given temperature} \\
    \rho_\mathrm{ref} [\si{\kilogram\per\metre\cubed}] \text{: reference density}   \\
    T [\si{\kelvin}] \text{: given temperature}                                     \\
    T_\mathrm{ref} [\si{\kelvin}] \text{: reference temperature}                    \\
    g = \SI{9.81}{\metre\per\second\squared}                                        \\
    a [\si{\kelvin\per\metre}] \text{: temperature gradient}                        \\
    R_\mathrm{air} = \SI[scientific-notation = false]{287}{\joule\per\kilogram\per\kelvin} \text{: specific gas constant}
  \end{array}
\]

\paragraph{Density in isothermal region}
\[
  \rho = \rho_\mathrm{ref} e^{-\frac{g}{R_\mathrm{air}T} \left( h - h_\mathrm{ref} \right)}
\]
\[
  \begin{array}{|l}
    \rho [\si{\pascal}] \text{: density at the given height}                                                              \\
    \rho_\mathrm{ref} [\si{\pascal}] \text{: reference density}                                                           \\
    g = \SI{9.81}{\metre\per\second\squared}                                                                              \\
    R_\mathrm{air} = \SI[scientific-notation = false]{287}{\joule\per\kilogram\per\kelvin} \text{: specific gas constant} \\
    T [\si{\kelvin}] \text{: temperature}                                                                                 \\
    h [\si{\metre}] \text{: altitude}                                                                                     \\
    h_\mathrm{ref} [\si{\metre}] \text{: reference altitude}                                                              \\
  \end{array}
\]

\paragraph{Drag}
\[
  D = \frac{1}{2} \rho v^2 S_\mathrm{ref} c_D
\]
\[
  \begin{array}{|l}
    \rho [\si{\kilogram\per\metre\cubed}] \text{: air density at the given altitude} \\
    v [\si{\metre\per\second}] \text{: speed of the body relative to air}            \\
    S_\mathrm{ref} [\si{\metre\squared}] \text{: wing area}                          \\
    c_D \text{: drag coefficient}
  \end{array}
\]

\paragraph{Drag polar}
The aerodynamic data for an aircraft is presented in from of a drag polar
\[
  c_D = c_{D,0} + \frac{{c_L}^2}{\pi e \mathrm{AR}}
\]
\[
  \begin{array}{|l}
    c_D \text{: drag coefficient}                                                \\
    c_{D,0} \text{: parasitic (form and friction) drag coefficient at zero lift} \\
    c_L \text{: lift coefficient}                                                \\
    e \text{: Oswald efficiency factor}                                          \\
    \mathrm{AR} \text{: aspect ratio}
  \end{array}
\]

\paragraph{Dynamic pressure}
Dynamic pressure is the pressure developed bu the forward motion of a body.
\[
  P = \frac{1}{2} \rho v^2
\]
\[
  \begin{array}{|l}
    P [\si{\pascal}] \text{: dynamic pressure}                                       \\
    \rho [\si{\kilogram\per\metre\cubed}] \text{: air density at the given altitude} \\
    v [\si{\metre\per\second}] \text{: speed of the body relative to air}
  \end{array}
\]

\paragraph{Equivalent air speed}
Equivalent speed on sea level
\[
  v_e = v \sqrt{\frac{\rho}{\rho_0}}
\]
\[
  \begin{array}{|l}
    v_e [\si{\metre\per\second}] \text{: equivalent air speed}                       \\
    v [\si{\metre\per\second}] \text{: speed of the body relative to air}            \\
    \rho [\si{\kilogram\per\metre\cubed}] \text{: air density at the given altitude} \\
    \rho_0 [\si{\kilogram\per\metre\cubed}] \text{: air density at sea level}
  \end{array}
\]

\paragraph{Flow}
Flow of a fluid in a pump
\[
  Q = \omega V
\]
\[
  \begin{array}{|l}
    Q [\si{\metre\cubed\per\second}] \text{: flow}        \\
    \omega [\si{\radian\per\second}] \text{: shaft speed} \\
    V [\si{\metre\cubed}] \text{: volume of fluid displaced}
  \end{array}
\]

\paragraph{Lift}
\[
  L = \frac{1}{2} \rho v^2 S_\mathrm{ref} c_L
\]
\[
  \begin{array}{|l}
    \rho [\si{\kilogram\per\metre\cubed}] \text{: air density at the given altitude} \\
    v [\si{\metre\per\second}] \text{: speed of the body relative to air}            \\
    S_\mathrm{ref} [\si{\metre\squared}] \text{: wing area}                          \\
    c_L \text{: lift coefficient}
  \end{array}
\]


\paragraph{Mach number}
The mach number is used to identify different aerodynamic flow regimes (subsonic, sonic and supersonic)
\[
  M = \frac{v}{a} = \frac{v}{\sqrt{\gamma R T}}
\]
\[
  \begin{array}{|l}
    v [\si{\metre\per\second}] \text{: true air speed}                                                                    \\
    a [\si{\metre\per\second}] \text{: speed of sound}                                                                    \\
    \gamma \text{: isentropic coefficient, } \gamma = 1.4 \text{ at } T = \SI[scientific-notation = false]{300}{\kelvin}  \\
    R_\mathrm{air} = \SI[scientific-notation = false]{287}{\joule\per\kilogram\per\kelvin} \text{: specific gas constant} \\
    T [\si{\kelvin}] \text{: air temperature}
  \end{array}
\]

\paragraph{Orbit equation}
The minimal orbit velocity is \(v_\mathrm{orbit} \approx \SI{7.9}{\kilo\metre\per\second}\) and the escape velocity is \(v_\mathrm{escape} \approx \SI{11.2}{\kilo\metre\per\second}\)
\[
  \begin{split}
    r & = \frac{p}{1 + e \cos(\theta - C)} \\
    & = \frac{\frac{(r^2 \dot{\theta})^2}{GM}}{1 + A\left( \frac{(r^2 \dot{\theta})^2}{GM} \right) \cos(\theta - C)}
  \end{split}
\]
\[
  \begin{array}{|l}
    e = 0 \text{: path is a circle}   \\
    e < 1 \text{: path is an ellipse} \\
    e = 1 \text{: path is a parabola} \\
    e >1 \text{: path is a hyperbola}
  \end{array}
\]


\paragraph{Power required}
Power required from the thrust required and the velocity of the aircraft (level, constant speed flight):
\[
  P = Tv = \sqrt{\frac{2W^3 {c_D}^2}{\rho S_\mathrm{ref} {c_L}^3}}
\]
\[
  \begin{array}{|l}
    P [\si{\watt}] \text{: power required}                           \\
    T [\si{\newton}] \text{: thrust required}                        \\
    v [\si{\meter\per\second}] \text{: velocity relative to the air} \\
    W [\si{\newton}] \text{: aircraft weight}                        \\
    c_D \text{: drag coefficient}                                    \\
    \rho [\si{\kilogram\per\meter\cubed}] \text{: air density}       \\
    S_\mathrm{ref} [\si{\metre\squared}] \text{: wing area}          \\
    c_L \text{: lift coefficient}
  \end{array}
\]

\paragraph{Pressure in non-isothermal region}
\[
  P = P_\mathrm{ref} \left( \frac{T}{T_\mathrm{ref}} \right)^{-\frac{g}{aR_\mathrm{air}}}
\]
\[
  \begin{array}{|l}
    P [\si{\pascal}] \text{: pressure at the given temperature}  \\
    P_\mathrm{ref} [\si{\pascal}] \text{: reference pressure}    \\
    T [\si{\kelvin}] \text{: given temperature}                  \\
    T_\mathrm{ref} [\si{\kelvin}] \text{: reference temperature} \\
    g = \SI{9.81}{\metre\per\second\squared}                     \\
    a [\si{\kelvin\per\metre}] \text{: temperature gradient}     \\
    R_\mathrm{air} = \SI[scientific-notation = false]{287}{\joule\per\kilogram\per\kelvin} \text{: specific gas constant}
  \end{array}
\]

\paragraph{Pressure in isothermal region}
\[
  P = P_\mathrm{ref} e^{-\frac{g}{R_\mathrm{air}T} \left( h - h_\mathrm{ref} \right)}
\]
\[
  \begin{array}{|l}
    P [\si{\pascal}] \text{: pressure at the given height}                                                                \\
    P_\mathrm{ref} [\si{\pascal}] \text{: reference pressure}                                                             \\
    g = \SI{9.81}{\metre\per\second\squared}                                                                              \\
    R_\mathrm{air} = \SI[scientific-notation = false]{287}{\joule\per\kilogram\per\kelvin} \text{: specific gas constant} \\
    T [\si{\kelvin}] \text{: temperature}                                                                                 \\
    h [\si{\metre}] \text{: altitude}                                                                                     \\
    h_\mathrm{ref} [\si{\metre}] \text{: reference altitude}                                                              \\
  \end{array}
\]

\paragraph{Range}
Range is the max distance the aircraft can make with a given amount of fuel:
\[
  R = \frac{v}{C} \frac{L}{D} \ln\left( \frac{W_\mathrm{cruise,start}}{W_\mathrm{cruise,end}} \right)
\]
\[
  \begin{array}{|l}
    R [\si{\metre}] \text{: range}                          \\
    v [\si{\metre\per\second}] \text{: aircraft velocity}   \\
    C [\si{\per\second}] \text{: specific fuel consumption} \\
    L [\si{\newton}] \text{: lift}                          \\
    D [\si{\newton}] \text{: drag}                          \\
    W [\si{\newton}] \text{: weight}
  \end{array}
\]

\paragraph{Reynolds number}
Reynolds number is used to characterize the flow (laminar, transient, turbulent)
\[
  \mathrm{Re}_x = \frac{\rho v x}{\mu}
\]
\[
  \begin{array}{|l}
    \mathrm{Re}_x \text{: Reynolds number}                             \\
    \rho [\si{\kilogram\per\metre\cubed}] \text{: free stream density} \\
    v [\si{\metre\per\second}] \text{: free stream velocity}           \\
    x [\si{\metre}] \text{: characteristic length}                     \\
    \mu [\si{\kilogram\per\meter\per\second}] \text{: free stream viscosity}
  \end{array}
\]

\paragraph{Rocket equation}
The rocket equation relates the burnout velocity \(v_b\) of a rocket vehicle to the specific impulse \(I_\mathrm{sp}\) associated with teh engine and teh mass ration of the initial mass \(m_i\) and the final mass \(m_f\):
\begin{align*}
  \frac{m_i}{m_f} & = \exp \left( \frac{v_b}{g I_\mathrm{sp}} \right)   \\
  \iff v_b        & = g I_\mathrm{sp} \ln\left( \frac{m_i}{m_f} \right)
\end{align*}

\paragraph{Take-off length}
\[
  L_\mathrm{TO} = \frac{m v^2}{2T} \approx \frac{1.44 W^2}{g \rho S c_{L,\mathrm{max}}T}
\]
\[
  \begin{array}{|l}
    L_\mathrm{TO} [\si{\meter}] \text{: take-off length}        \\
    m [\si{\kilogram}] \text{: aircraft mass}                   \\
    v [\si{\meter\per\second}] \text{: aircraft take-off speed} \\
    T [\si{\newton}] \text{: thrust}                            \\
    W [\si{\newton}] \text{: aircraft weight}                   \\
    \rho [\si{\kilogram\per\meter\cubed}] \text{: air density}  \\
    S [\si{\meter\squared}] \text{: wing area}                  \\
    c_{L,\mathrm{max}} \text{: max lift coefficient}
  \end{array}
\]

\paragraph{Thrust}
Thrust equation for generic propulsive device
\[
  T = (\dot{m}_\mathrm{air} + \dot{m}_\mathrm{fuel})v_\mathrm{out} + P_\mathrm{out}A_\mathrm{out} - \dot{m}_\mathrm{air}v_\mathrm{in} - P_\mathrm{in}A_\mathrm{in}
\]
\[
  \begin{array}{|l}
    T [\si{\newton}] \text{: thrust generated}             \\
    \dot{m} [\si{\kilogram\per\second}] \text{: mass flow} \\
    v [\si{\metre\per\second}] \text{: flow velocity}      \\
    P [\si{\pascal}] \text{: flow pressure}                \\
    A [\si{\metre\squared}] \text{: flow area}
  \end{array}
\]

\paragraph{Thrust required}
Thrust required for a given aircraft at a given altitude caries with its velocity (level, constant speed flight)
\[
  T = W\frac{c_D}{c_L}
\]


\paragraph{Weight}
\[
  W_\mathrm{TO} = W_\mathrm{empty} + W_\mathrm{fixed} + W_\mathrm{fuel}
\]
\[
  \begin{array}{|l}
    W_\mathrm{TO} [\si{\newton}] \text{: take-off gross weight, MTOW} \\
    W_\mathrm{empty} [\si{\newton}] \text{: empty weight, OWE}        \\
    W_\mathrm{fixed} [\si{\newton}] \text{: fixed weight, payload}    \\
    W_\mathrm{fuel} [\si{\newton}] \text{: fuel weight}
  \end{array}
\]


\section{Unit conversion}
% 1cm = 10mm = 28pt = 1/2.54in
\begin{table}[ht] % Options: b (bottom), t (top), h (here), ! (insist)
  \caption{Unit conversion EES, SI and others}
  % \label{tab:unit-conversion}
  \begin{center}
    \centering % Horizontal alignment of the table
    \begin{tabular}{ % Number of letter (l: left, c: center, r: right) = number of column
        l|l|l
      }
      % Visible row border: \hline (needed for each row)
      % Visible column border: | next to tabular declaration (needed for each column)
      % Column separation: &, row separation: \\

               & EES                                                                                & SI                                                                   \\ \hline\hline
      \(m\)    & 1 slug                                                                             & 14.5939 kg                                                           \\ \hline
      \(F\)    & 1 lb                                                                               & 4.44822 N                                                            \\ \hline
      \(l\)    & 1 ft                                                                               & 0.3048 m                                                             \\ \hline
      \(P\)    & 1 \(\mathrm{lb} \cdot \mathrm{ft}^{-2}\)                                           & 47.8803 Pa                                                           \\ \hline
      \(T\)    & 1 °R                                                                               & 5/9\(\cdot\)R K                                                      \\ \hline
      \(\rho\) & 1 \(\mathrm{slugs} \cdot \mathrm{ft}^{-3}\)                                        & 515.378 \(\mathrm{kg} \cdot \mathrm{m}^{-3}\)                        \\ \hline
      \(v\)    & 1 \(\mathrm{ft} \cdot \mathrm{s}^{-1}\)                                            & 0.3048 \(\si{\metre\per\second}\)                                    \\ \hline
      \(R\)    & 1 \(\mathrm{ft} \cdot \mathrm{lb} \cdot \mathrm{slugs}^{-1} \cdot \text{°R}^{-1}\) & 0.167226 \(\mathrm{J} \cdot \mathrm{kg}^{-1} \cdot \mathrm{K}^{-1}\) \\
    \end{tabular}
    \begin{tabular}{l|l}
               & Others                                                        \\ \hline\hline
      \(m\)    & 32.1740 lb                                                    \\ \hline
      \(l\)    & 12 in, 1/5280 mi, 1/6076 nmi                                  \\ \hline
      \(P\)    & 1/144 psi                                                     \\ \hline
      \(T\)    & 5/9\(\cdot\)R - 273.15 °C, R - 459.67 °F                      \\ \hline
      \(\rho\) & 32.1740 \(\mathrm{lb} \cdot \mathrm{ft}^{-3}\)                \\ \hline
      \(v\)    & 3600/5280 \(\mathrm{mi} \cdot \mathrm{h}^{-1}\), 3600/6076 kn \\
    \end{tabular}
  \end{center}
\end{table}


\section{International standard atmosphere}
Atmosphere model (oxygen supply at 12 500 ft or more for more than 30 min):


ISA temperature (\(T\) [K], \(h\) [km]):
\begin{description}
  \item[0 - 11 km:] \(T = -6.5 h + 288.16\)
  \item[11 - 25 km:] \(T = 216.66\)
  \item[25 - 47 km:] \(T = 3 (h - 25) + 216.66\)
  \item[47 - 53 km:] \(T = 282.66\)
  \item[53 - 79 km:] \(T = -4.5 (h - 53) + 282.66\)
  \item[79 - 90 km:] \(T = 165.66\)
  \item[90 - 105 km:] \(T = 4 (h - 90) + 165.66\)
\end{description}

\section{Common acronyms}
\begin{description}
  \item[AFDS:] aircraft flight direction system
  \item[AC:] advisory circulars, example of acceptable means, but not the only means, of demonstrating compliance with regulations and standards
  \item[AD:] airworthiness directives, legally enforceable regulation to correct unsafe conditions in a product
  \item[AFDX:] aviation full duplex
  \item[AGB:] accessory gear box
  \item[AR:] aspect ratio
        \[
          AR = \frac{b^2}{S_\mathrm{ref}}
        \]
        \[
          \begin{array}{|l}
            AR \text{: aspect ratio}           \\
            b [\si{\metre}] \text{: wing span} \\
            S_\mathrm{ref} [\si{\metre\squared}] \text{: wing area}
          \end{array}
        \]
  \item[ARINC:] aeronautical radio incorporated
  \item[ARP:] aircraft recommended practice
  \item[ASI:] air speed indicator
  \item[ATA:] Air Transport Association
  \item[ATC:] air traffic control
  \item[BPR:] bypass ratio
        \[
          BPR = \frac{\dot{m}_\mathrm{fan}}{\dot{m}_\mathrm{core}}
        \]
        \begin{description}
          \item[BPR < 2:] Low
          \item[2 < BPR < 4:] Medium
          \item[BPR > 4:] High  
          \item[BPR = 12:] Ultra-high
        \end{description}
  \item[EASA:] European Aviation Safety Agency
  \item[CAR:] Canadian Aviation Regulations
  \item[EBHA:] electrical backup hydraulic actuator
  \item[EDP:] engine driven pump
  \item[EHA:] electro-hydraulic actuator
  \item[EHSA:] electro-hydraulic servo actuator
  \item[EICAS:] engine indicating and crew alerting system
  \item[EMA:] electro-mechanical actuator
  \item[EMC:] electro-magnetic compatibility
  \item[EMI:] electro-magnetic interference
  \item[EMP:] electrical motor pump
  \item[FAA:] Federal Aviation Administration
  \item[FAR:] Federal Aviation Regulations
  \item[FADEC:] full authority digital engine control system
  \item[FBW:] fly-by-wire
  \item[FCU:] fuel control unit, controls thrust generated by the engine
  \item[FCU:] flight control unit
  \item[FMS:] flight management system
  \item[HF:] high frequency
  \item[HMA:] hydro-mechanical actuator
  \item[IATA:] International Air Transport Association
  \item[IEEE:] Institute of Electrical and Electronics Engineers
  \item[IDG:] integrated drive generator
  \item[IMA:] integrated modular architecture
  \item[ISA:] International Standard Atmosphere
  \item[ITT:] inter turbine temperature
  \item[JAR:] Joint Aviation Regulations
  \item[LCC:] life-cycle cost
  \item[LRU:] line replaceable units
  \item[LRM:] line replaceable modules
  \item[MCDU:] multipurpose control and display unit
  \item[MLG:] main landing gear
  \item[MTOW:] maximum takeoff weight
  \item[NLG:] nose landing gear
  \item[NACA:] National Advisory Committee for Aeronautics
  \item[NH:] rotation speed of the engine high pressure shaft
  \item[NF:] rotation speed of the engine free turbine shaft
  \item[OAT:] outside air temperature
  \item[OBOGS:] on-board oxygen generation system
  \item[OEM:] original equipment manufacturer
  \item[OWE:] operational weight empty (structure propulsion, systems, instruments, avionics)
  \item[Payload:] crew, equipment, passengers, baggage, food, drinks, cargo, missiles bombs, ammunition
  \item[PSR:] primary surveillance radar
  \item[PTU:] power transfer unit (transfer power from a hydraulic system to another one)
  \item[RAT:] ram air turbine
  \item[RNP:] required navigation performance
  \item[SELCAL:] selective call
  \item[SFC:] specific fuel consumption
        \[
          \frac{m_\text{fuel consumed}}{\dot{W}_\text{brake horse}t}
        \]
        \[
          \begin{array}{|l}
            m_\text{fuel consumed} \text{: mass of fuel consumed}                     \\
            \dot{W}_\text{brake horse} \text{: for propeller engines, power provided} \\
            \text{to the propeller}                                                   \\
            t [\si{\hour}] \text{: duration of the fuel consumption}
          \end{array}
        \]
  \item[SSR:] secondary surveillance radar
  \item[TCCA:] Transport Canada's Civil Aviation
  \item[TGT:] turbine gas temperature
  \item[TRL:] technology readiness level
  \item[TRU:] transformer rectifier unit
  \item[TSFC:] thrust specific fuel consumption
        \[
          \frac{m_\text{fuel consumed}}{F_\text{thrust generated}t}
        \]
        \[
          \begin{array}{|l}
            m_\text{fuel consumed} [\si{\kilogram}] \text{: mass of fuel consumed}          \\
            F_\text{thrust generated} [\si{\newton}] \text{: force generated by the thrust} \\
            t [\si{\hour}] \text{: duration of the fuel consumption}
          \end{array}
        \]
  \item[VFG:] variable frequency generator
  \item[VHF:] very high frequency
  \item[VSCFG:] variable speed constant frequency generator
  \item[WTB:] wing tip brake
\end{description}


\section{Materials and structures}
Categories and function of aircraft structures:
\begin{description}
  \item[Primary structure] Load-bearing, stress and safety (wing box, fuselage, load paths)
  \item[Secondary structure] Shaping, contour, surface (fairings, engine nacelle, nose cone)
  \item[Inner structural configuration] Partitioning (payload, cockpit, system installation zones, safety-critical zones, segregation)
\end{description}

Structural layout of the aircraft fuselage:
\begin{description}
  \item[External skin] Carries shear from external transverse, torsional and cabin pressure loads
  \item[Tranverse elements] Bulkheads support high concentrated loads in strategic zones (wings, empennage, landing gear) and distribute it into the skin; Frames maintain geometric integrity, support small loads
  \item[Longitudinal elements] Longerons carry major portion of axial loads from bending; stringer carry residual axial loads from bending
\end{description}

Wing structure layout:
\begin{description}
  \item[Multi-rib] transport aircraft with high aspect ratio and relatively large thickness
  \item[Multi-spar] high speed fighter aircraft, thin and highly loaded wings
\end{description}

% 1cm = 10mm = 28pt = 1/2.54in
\begin{table}[ht] % Options: b (bottom), t (top), h (here), ! (insist)
  \caption{Material properties}
  % \label{tab:Table reference}
  \begin{center}
    \centering % Horizontal alignment of the table
    \begin{tabular}{ % Number of letter (l: left, c: center, r: right) = number of column
      l|p{2.5cm}|l
      }
      % Visible row border: \hline (needed for each row)
      % Visible column border: | next to tabular declaration (needed for each column)
      % Column separation: &, row separation: \\

                 & Strength/weight         & Cost                  \\ \hline\hline
      Al-alloy   & good                    & reasonable            \\ \hline
      Steel      & too heavy               & reasonable            \\ \hline
      Titanium   & better than Al          & very high             \\ \hline
      Composites & very good               & depends               \\ \hline\hline
                 & Temperature sensitivity & Corrosion             \\ \hline\hline
      Al-alloy   & good                    & good                  \\ \hline
      Steel      & good                    & stainless steel: good \\ \hline
      Titanium   & very good               & good                  \\ \hline
      Composites & bad                     & very good             \\
    \end{tabular}
  \end{center}
\end{table}

% 1cm = 10mm = 28pt = 1/2.54in
\begin{table*}[ht] % Options: b (bottom), t (top), h (here), ! (insist)
  \caption{Fastener types}
  % \label{tab:Table reference}
  \centering % Horizontal alignment of the table
  \begin{tabular}{ % Number of letter (l: left, c: center, r: right) = number of column
    m{0.2\linewidth}|p{0.3\linewidth}|p{0.3\linewidth}
    }
    % Visible row border: \hline (needed for each row)
    % Visible column border: | next to tabular declaration (needed for each column)
    % Column separation: &, row separation: \\

    Fastener             & Advantages                                                                                         & Disadvantages                                                                                                \\ \hline\hline
    Rivets               & Low cost, low weight, flush surface possible, high rigidity                                        & Permanent, limited static shear, low tension, noisy installation                                             \\ \hline
    Bolts                & Removable, high shear, high tension                                                                & no flush surface, high weight, high cost                                                                     \\ \hline
    Solid rivet          & Low cost, low weight, good shear strength, automation, good clamp-up                               & Low tension strength, access to both side, permanent, noisy installation, incompatible with composite        \\ \hline
    Blind rivet          & One side installation, low cost, low weight                                                        & Low tension strength, low shear strength, permanent, moderate clamp-up, poor fatigue, moderate reliability   \\ \hline
    Blind Bolt           & One side installation, good shear strength                                                         & Low tension strength, high cost and weight, permanent, moderate clamp-up, poor fatigue, moderate reliability \\ \hline
    Hi-Lites and Hi-Loks & High shear strength, high clamp-up, high reliability, moderate tension strength, high interference & Access to both side, permanent, noisy installation, moderate cost and weight, very high cost for Taper-Lok   \\
  \end{tabular}
\end{table*}


\section{Aviation alphabet}
\begin{description}
  \item[A:] ALPHA
  \item[B:] BRAVO
  \item[C:] CHARLIE
  \item[D:] DELTA
  \item[E:] ECHO
  \item[F:] FOXTROT
  \item[G:] GOLF
  \item[H:] HOTEL
  \item[I:] INDIA
  \item[J:] JULIET
  \item[K:] KILO
  \item[L:] LIMA
  \item[M:] MIKE
  \item[N:] NOVEMBER
  \item[O:] OSCAR
  \item[P:] PAPA
  \item[Q:] QUEBEC
  \item[R:] ROMEO
  \item[S:] SIERRA
  \item[T:] TANGO
  \item[U:] UNIFORM
  \item[V:] VICTOR
  \item[W:] WHISKEY
  \item[X:] XRAY
  \item[Y:] YANKEE
  \item[Z:] ZULU
\end{description}

\end{document}