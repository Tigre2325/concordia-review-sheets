\documentclass[10pt, twocolumn]{article}

%%%%%%%%%%%%%%%%%%%%%%%%%%%%%%%%%%%%%%%%%%%%%%%%%%%%%%%%%%%%%%%%%%%%%%%%%%%%%%%
%%%% Cover page
\title{ENGR 213: Applied Ordinary Differential Equations}
\date{\today}
\author{Anthony Bourboujas}

\makeatletter
\let\Title\@title
\let\Author\@author
\let\Date\@date
\makeatother

%%%%%%%%%%%%%%%%%%%%%%%%%%%%%%%%%%%%%%%%%%%%%%%%%%%%%%%%%%%%%%%%%%%%%%%%%%%%%%%
%%%% Preamble
%%%%%%%%%%%%%%%%%%%%%%%%%%%%%%%%%%%%%%%%%%%%%%%%%%%%%%%%%%%%%%%%%%%%%%%%%%%%%%%
%%%% Packages
\usepackage[utf8x]{inputenc} % Accept different input encodings
\usepackage[T1]{fontenc} % Standard package for selecting font encodings
\usepackage{lmodern} % Font name; classic: lmodern
\usepackage[english]{babel} % Multilingual support for LaTeX
% \usepackage{abstract} % Control the typesetting of the abstract environment
\usepackage{amsmath} % AMS mathematical facilities for LaTeX
\usepackage{amssymb} % TeX fonts from the American Mathematical Society
\usepackage{amsthm} % Typesetting theorems (AMS style)
\usepackage{array} % Extending the array and tabular environments
% \usepackage[backend=biber,style=ieee,sorting=none]{biblatex}
\usepackage{bold-extra} % Use bold small caps and typewriter fonts
\usepackage{cellspace} % Ensure minimal spacing for table cells
\usepackage{chemformula} % Command for typesetting chemical formulas and reactions
% \usepackage{colortbl} % Add colour to LaTeX tables
\usepackage{comment} % Selectively include/exclude portions of text
\usepackage{csquotes} % Context sensitive quotation facilities
% \usepackage[en-US,showdow]{datetime2} % Formats for dates, times and time zones
% \usepackage{diagbox} % Table heads with diagonal lines
\usepackage{enumitem} % Control layout of itemize, enumerate, description
\usepackage{esint} % Extended set of integrals for Computer Modern
\usepackage{graphicx} % Enhanced support for graphics
% \usepackage{listings} % Typeset source code listings using LaTeX
% \usepackage{lipsum} % Easy access to the Lorem Ipsum dummy text
\usepackage{mathrsfs} % Support for using RSFS fonts in maths
% \usepackage{matlab-prettifier} % Pretty-print Matlab source code
\usepackage{moreverb} % Extended verbatim
\usepackage{multicol} % Intermix single and multiple columns
\usepackage{multirow} % Create tabular cells spanning multiple rows
% \usepackage{pgfplots} % Plots
% \usepackage{pgfplotstable} % Loads, rounds, format and post-processes numerical tables (generates table from CSV)
% \usepackage{pdfpages} % Include PDF document in LaTeX
% \usepackage{rotating} % Rotation tools, including rotated full-page floats with sidewaysfigure
\usepackage[scr]{rsfso} % A mathematical calligraphic font based on rsfs
\usepackage{setspace} % Set space between lines
\usepackage{soul} % Hyphenation for letterspacing, underlining, and more
\usepackage{threeparttable} % Tables with captions and notes all the same width
% \usepackage{verbatim} % Reimplementation of and extensions to LaTeX verbatim
\usepackage{wrapfig} % Produces figures which text can flow around
\usepackage{xcolor} % Driver-independent color extensions for LaTeX
\usepackage{xurl} % Verbatim with URL-sensitive line breaks, allow URL breaks at any alphanumerical character

%%%%%%%%%%%%%%%%%%%%%%%%%%%%%%%%%%%%%%%%%%%%%%%%%%%%%%%%%%%%%%%%%%%%%%%%%%%%%%%
%%%% Lengths
% 1cm = 10mm = 28pt = 1/2.54in
% 1ex = height of a lowercase 'x' in the current font
% 1em = width of an uppercase 'M' in the current font

%%%% Spacing in math mode
% \!                         = -3/18em
% \,                         = 3/18em
% \:                         = 4/18em
% \;                         = 5/18em
% \ (space after backslash!) = space in normal text
% \quad                      = 1em
% \qquad                     = 2em

% \setlength{\baselineskip}{1em} % Vertical distance between lines in a paragraph
% \renewcommand{\baselinestretch}{1.0} % A factor multiplying \baslineskip
\setlength{\columnsep}{0.75cm} % Distance between columns
% \setlength{\columnwidth}{} % The width of a column
\setlength{\columnseprule}{1pt} % The width of the vertical ruler between columns
% \setlength{\evensidemargin}{} % Margin of even pages, commonly used in two-sided documents such as books
% \setlength{\linewidth}{} % Width of the line in the current environment.
% \setlength{\oddsidemargin}{} % Margin of odd pages, commonly used in two-sided documents such as books
% \setlength{\paperwidth}{} % Width of the page
% \setlength{\paperheight}{} % Height of the page
\setlength{\parindent}{0cm} % Paragraph indentation
\setlength{\parskip}{6pt} % Vertical space between paragraphs
% \setlength{\tabcolsep}{} % Separation between columns in a table (tabular environment)
% \setlength{\textheight}{} % Height of the text area in the page
% \setlength{\textwidth}{} % Width of the text area in the page
% \setlength{\topmargin}{} % Length of the top margin
\setlist{
  %%%% Vertical spacing
  topsep = 0pt,
  partopsep = 0pt,
  parsep = 0pt,
  itemsep = 0pt,
  %%%% Horizontal spacing
  leftmargin = 0.5cm,
  rightmargin = 0cm,
  % listparindent = 0cm,
  % labelwidth = 0cm,
  % labelsep = 0cm,
  % itemindent = 0cm
}
\addtolength{\cellspacetoplimit}{2pt}
\addtolength{\cellspacebottomlimit}{2pt}

%%%%%%%%%%%%%%%%%%%%%%%%%%%%%%%%%%%%%%%%%%%%%%%%%%%%%%%%%%%%%%%%%%%%%%%%%%%%%%%
%%%% Page layout
\usepackage{layout} % View the layout of a document
\usepackage{geometry} % Flexible and complete interface to document dimensions
% 1cm = 10mm = 28pt = 1/2.54in
% ex = height of a lowercase 'x' in the current font
% em = width of an uppercase 'M' in the current font
\geometry{
  a4paper,
  top         = 1cm,
  bottom      = 1cm,
  left        = 1.5cm,
  right       = 1.5cm,
  includehead = true,
  includefoot = true,
  landscape   = false, % Paper orientation
  twoside     = false,
}
% \geometry{showframe} % Show paper outline for the text area and page

%%%%%%%%%%%%%%%%%%%%%%%%%%%%%%%%%%%%%%%%%%%%%%%%%%%%%%%%%%%%%%%%%%%%%%%%%%%%%%%
%%%% Header and footer style
\usepackage{fancyhdr} % Extensive control of page headers and footers in LaTeX
\pagestyle{fancy}
% Options: \leftmark (chapter title), \rightmark(section title), \thepage (page number), \thechapter(chapter number), \thesection (section number)
\lhead{\thetitle}
\chead{}
\rhead{}
\lfoot{}
\cfoot{\thepage}
\rfoot{}

%%%%%%%%%%%%%%%%%%%%%%%%%%%%%%%%%%%%%%%%%%%%%%%%%%%%%%%%%%%%%%%%%%%%%%%%%%%%%%%
%%%% URL insertion settings
\usepackage{hyperref} % Extensive support for hypertext in LaTeX
\definecolor{black}{RGB}{0, 0, 0} % rgb(0, 0, 0)
\definecolor{blue}{RGB}{0, 0, 255} % rgb(0, 0, 255)
\hypersetup{
  % unicode            = true,
  pdftitle           = {\thetitle},
  pdfauthor          = {\theauthor},
  % pdfsubject       = {},
  %%%% Reference
  % bookmarks          = true,
  bookmarksnumbered  = true,
  bookmarksopen      = true, % Open the bookmarks
  bookmarksopenlevel = 2, % Open until 1 level (section)
  %%%% Bookmarks
  breaklinks         = true,
  pdfborder          = {0 0 0},
  % backref            = true, % Add links into bibliography
  % pagebackref        = true,
  % hyperindex         = true, % Add links into index
  %%%% Color
  colorlinks         = true,
  linkcolor          = black, % Internal links color
  citecolor          = black,
  urlcolor           = blue, % Hyperlinks color
  filecolor          = black,
}

\usepackage{varioref} % Intelligent page reference
\usepackage[capitalise,noabbrev]{cleveref}
\usepackage{prettyref} % Make label references "self-identity" with \prettyref{#1}
\newrefformat{cha}{chapter \textbf{\nameref{#1}} \vpageref{#1}} % {chapter \textbf{\nameref{#1}} on page \pageref{#1}}
\newrefformat{sec}{section \textbf{\nameref{#1}} \vpageref{#1}} % {section \textbf{\nameref{#1}} on page \pageref{#1}}
% \newrefformat{fig}{\vref{#1}} % {Figure \ref{#1} on page \pageref{#1}}
% \newrefformat{tab}{\vref{#1}} % {Table \ref{#1} on page \pageref{#1}}
% \newrefformat{eqn}{\vref{#1}}
% \newrefformat{lis}{\emph{\nameref{#1}} \vpageref{#1}}

%%%%%%%%%%%%%%%%%%%%%%%%%%%%%%%%%%%%%%%%%%%%%%%%%%%%%%%%%%%%%%%%%%%%%%%%%%%%%%%
%%%% Physics units settings
% Dependencies
\usepackage{booktabs} % Publication quality tables in LaTeX
\usepackage{caption} % Customizing captions in floating environments
\usepackage{helvet} % Load Helvetica, scaled
\usepackage{cancel} % Place lines through maths formulae

\usepackage{siunitx} % A comprehensive (SI) units package
\sisetup{
  exponent-product     = \cdot, % Symbol between number and power of ten
  group-minimum-digits = 5, % Number of digits when 3 digits separation appear
  % inter-unit-product   = \cdot, % Symbol between units (when several units are used)
  output-complex-root  = \ensuremath{i}, % How i math should be seen
  % prefixes-as-symbols  = false, % Translate prefixes (kilo, centi, milli, micro,...) into a power of ten
  separate-uncertainty = true, % Write uncertainty with +-
  scientific-notation  = engineering,
}

%%%%%%%%%%%%%%%%%%%%%%%%%%%%%%%%%%%%%%%%%%%%%%%%%%%%%%%%%%%%%%%%%%%%%%%%%%%%%%%
%%%% Theorems and proofs
\numberwithin{equation}{section}
% \makeatletter
% \g@addto@macro\th@remark{\thm@headpunct{:}}
% \makeatother
\theoremstyle{remark}
\newtheorem*{example}{Example}
\newtheorem*{remark}{Remark}

%%%%%%%%%%%%%%%%%%%%%%%%%%%%%%%%%%%%%%%%%%%%%%%%%%%%%%%%%%%%%%%%%%%%%%%%%%%%%%%
%%%% User-defined environments
% Remove the space before the enumerate and itemize environments
\let\oldenumerate\enumerate % Keep a copy of \enumerate (or \begin{enumerate})
\let\endoldenumerate\endenumerate % Keep a copy of \endenumerate (or \end{enumerate})
\renewenvironment{enumerate}{
  \begin{oldenumerate}
    \vspace{-6pt}
    }{
  \end{oldenumerate}
}

\let\olditemize\itemize % Keep a copy of \itemize (or \begin{itemize})
\let\endolditemize\enditemize % Keep a copy of \enditemize (or \end{itemize})
\renewenvironment{itemize}{
  \begin{olditemize}
    \vspace{-6pt}
    }{
  \end{olditemize}
}

\let\olddescription\description % Keep a copy of \description (or \begin{description})
\let\endolddescription\enddescription % Keep a copy of \enddescription (or \end{description})
\renewenvironment{description}{
  \begin{olddescription}
    \vspace{-6pt}
    }{
  \end{olddescription}
}

%%%%%%%%%%%%%%%%%%%%%%%%%%%%%%%%%%%%%%%%%%%%%%%%%%%%%%%%%%%%%%%%%%%%%%%%%%%%%%%
%%%% User-defined commands
\newcommand{\Romannumeral}[1]{\MakeUppercase{\romannumeral #1}} % Capital roman numbers
% \newcommand{\gui}[1]{\og #1 \fg{}} % French quotation marks
\renewcommand{\thefootnote}{[\arabic{footnote}]}

%%% Figure command
%% Include SVG files
\newcommand{\executeiffilenewer}[3]{
  \ifnum\pdfstrcmp{\pdffilemoddate{#1}}
    {\pdffilemoddate{#2}}>0
    {\immediate\write18{#3}}\fi
}
\newcommand{\includesvg}[1]{
  \executeiffilenewer{#1.svg}{#1.pdf}
  {
    % Inkscape must be installed in PATH and the user must include '--shell-escape' in the build arguments
    inkscape #1.svg --export-type=pdf --export-latex
  }
  \input{#1.pdf_tex}
}

%%% Math commands
%% Tables (requires cellspace package)
\newcolumntype{L}{>{\(\displaystyle}Cl<{\)}} % Column type for left-aligned math column
\newcolumntype{D}{>{\(\displaystyle}Cc<{\)}} % Column type for centered math column

%% Functions
\newcommand{\constant}{\mathrm{constant}} % Constant
\newcommand{\abs}[1]{\left| #1 \right|} % Absolute function
\newcommand{\erf}[1]{\mathrm{erf} \left( #1 \right)} % Error function
\newcommand{\erfc}[1]{\mathrm{erfc} \left( #1 \right)} % Complementary error function
\newcommand{\unitstep}[1]{\,\mathcal{U}\left( #1 \right)} % Unit step function
\newcommand{\diracdelta}[2]{\,\delta_{#1}\left( #2 \right)} % Dirac delta function


%% Derivatives and integrals
\newcommand{\diff}[2]{\mathrm{d}^{#1} #2} % Letter 'd' of differentials
\newcommand{\diffint}[1]{\,\diff{}{#1}} % Differential with a space for integrals
\newcommand{\derivative}[2]{\frac{\diff{}{#1}}{\diff{}{#2}}} % Derivative
\newcommand{\nderivative}[3]{\frac{\diff{#1}{#2}}{\diff{}{#3^{#1}}}} % Derivative of degree n
\newcommand{\partialderivative}[2]{\frac{\partial #1}{\partial #2}} % Partial derivative
\newcommand{\npartialderivative}[3]{\frac{\partial^{#1} #2}{\partial #3^{#1}}} % Partial derivative of degree n
\newcommand{\direcderivative}[2]{D_{\vec{#1}}\,#2} % Directional derivative

\newcommand{\Laplace}[1]{\mathcal{L}\left\{ #1 \right\}} % Laplace transform notation
\newcommand{\invLaplace}[1]{\mathcal{L}^{-1}\left\{ #1 \right\}} % Inverse Laplace transform notation

%% Set
\newcommand{\set}[3]{\mathbb{#1}_{#2}^{#3}} % Set of numbers
\newcommand{\integerset}{\mathbb{Z}} % Set of integer numbers (compatibility)
\newcommand{\realset}{\mathbb{R}} % Set of real numbers (compatibility)

%% Limits
\newcommand{\limit}[3]{\lim_{#1 \to #2}{#3}} % Limit from a point to another
\newcommand{\rlimit}[3]{\lim_{#1 \to #2^{+}}{#3}} % Right limit from a point to another
\newcommand{\llimit}[3]{\lim_{#1 \to #2^{-}}{#3}} % Left imit from a point to another
\newcommand{\modulus}[1]{\,\left[ #1 \right]} % Modulus notation

%% Vectors
\newcommand{\ivec}{\hat{\mathrm{i}}} % i vector
\newcommand{\jvec}{\hat{\mathrm{j}}} % j vector
\newcommand{\kvec}{\hat{\mathrm{k}}} % k vector
\renewcommand{\Vec}[1]{\overrightarrow{#1}} % Vector notation for expression with more than one letter
\newcommand{\norm}[1]{\left\| #1 \right\|} % Norm notation for expression with just one letter
\newcommand{\normvec}[1]{\left\| \vec{#1} \right\|} % Norm notation for expression with just one letter
\newcommand{\Normvec}[1]{\left\| \Vec{#1} \right\|} % Norm notation for expression with more than one letter
\newcommand{\comp}[2]{\mathrm{comp}_{\vec{#2}}\vec{#1}} % Components
\newcommand{\proj}[2]{\mathrm{proj}_{\vec{#2}}\vec{#1}}
\newcommand{\grad}[1]{\vec{\nabla}#1} % Gradient notation
\newcommand{\frames}[2]{\left( #1 \right)_{#2}} % Frame definition

\newcommand{\curl}[1]{\mathrm{curl}\,\vec{#1}} % Curl of a vector field
\newcommand{\divergence}[1]{\mathrm{div}\,\vec{#1}} % Divergence of a vector field


%%%%%%%%%%%%%%%%%%%%%%%%%%%%%%%%%%%%%%%%%%%%%%%%%%%%%%%%%%%%%%%%%%%%%%%%%%%%%%%
%%%% New commands
% \renewcommand{\contentsname}{} % Change the table of contents title

%%%%%%%%%%%%%%%%%%%%%%%%%%%%%%%%%%%%%%%%%%%%%%%%%%%%%%%%%%%%%%%%%%%%%%%%%%%%%%%
%%%% Theorems and proofs
\theoremstyle{definition}
\newtheorem*{definition}{Definition}


%%%%%%%%%%%%%%%%%%%%%%%%%%%%%%%%%%%%%%%%%%%%%%%%%%%%%%%%%%%%%%%%%%%%%%%%%%%%%%%
%%%% Beginning of the document
\begin{document}
\maketitle
% \tableofcontents
% \layout % Show a drawing of page layout
% \renewcommand{\abstractname}{} % Change the abstract title
\setlength{\abovedisplayskip}{2pt} % Space above displayed equations
\setlength{\belowdisplayskip}{2pt} % Space below displayed equations


\section{Introduction to differential equations}
\subsection{Definitions and terminology}
A differential equation (differential equation) is an equation containing the derivatives of one or more dependent variables with respect to one or more independent variables.

Differential equations can be classified by:
\begin{itemize}
  \item Type:
        \begin{itemize}
          \item Ordinary differential equations (ODE): derivatives are with respect to a single independent variable.
                \begin{example}
                  \[
                    \derivative{y}{x} + 5y = e^x \text{ or } \derivative{x}{t} + \derivative{y}{t} = 3x + 2y
                  \]
                \end{example}
          \item Partial differential equations (PDE): derivatives are with respect to two or more independent variables.
                \begin{example}
                  \[
                    \npartialderivative{2}{u}{x} = \npartialderivative{2}{u}{t} - 2\partialderivative{u}{t} \text{ or } \partialderivative{u}{y} = - \partialderivative{v}{x}
                  \]
                \end{example}
        \end{itemize}
  \item Order: the order of an ordinary differential equation or a partial differential equation is the order of the highest derivative in the equation
        \begin{example}
          \begin{align*}
            \derivative{y}{x} + 5y = e^x                                     & \rightarrow \text{ order 1} \\
            2\npartialderivative{4}{u}{x} + \npartialderivative{2}{u}{x} = 0 & \rightarrow \text{ order 4}
          \end{align*}
        \end{example}
  \item Linearity:
        \begin{itemize}
          \item Form of a linear ordinary differential equation:
                \begin{multline*}
                  a_n (x)\nderivative{n}{y}{x} + a_{n - 1}(x)\nderivative{n - 1}{y}{x} + \cdots \\
                  + a_1 (x)\derivative{y}{x} + a_0 (x)y = g(x)
                \end{multline*}
                \begin{multline*}
                  a_n (x) y^{(n)} + a_{n - 1}(x) y^{(n - 1)} + \cdots \\
                  + a_1 (x) y' + a_0 (x)y = g(x)
                \end{multline*}

          \item Non-linear ordinary differential equation:
                \begin{itemize}
                  \item The coefficient \(a_0 (x)\), \(a_1 (x)\), \dots,\(a_{n - 1}(x)\), \(a_n (x)\) contains the dependent variable \(y\) or its derivative
                  \item Powers of \(y\), \(y'\), \dots, \(y^{(n - 1)}\), \(y^{(n)}\) appears in the equation
                  \item Non-linear functions of the dependent variable or its derivatives, such as \(\sin{y}\) or \(e^y \), appear in the equation
                \end{itemize}
        \end{itemize}
  \item Homogeneity:
        \begin{itemize}
          \item Homogeneous: \(g(x) = 0\) in a linear ordinary differential equation or \(\derivative{y}{x}\) can be represented as \(f\left( \frac{y}{x} \right)\)
          \item Nonhomogeneous: \(g(x)\) is a constant or a function of \(x\) in a linear ordinary differential equation or \(\derivative{y}{x}\) cannot be represented as \(f\left( \frac{y}{x} \right)\)
        \end{itemize}
\end{itemize}

Solutions of a ordinary differential equation:
\begin{itemize}
  \item Solution curve: the solution is a function.
        \begin{itemize}
          \item Families of solutions: solution contains one or more arbitrary constant \(c\), \( c_1 \), \dots, \( c_n \)
          \item Particular solution: solution does not contain an arbitrary constant
        \end{itemize}
  \item Singular solution: solution which is not member of a family of solutions
  \item Trivial solution: \(y(x) = 0\)
\end{itemize}


\begin{definition}
  If a linear homogeneous ordinary differential equation has solutions \(y = f(x)\) and \(y = g(x)\), then \(y = af(x) + bg(x)\) (where \(a\) and \(b\) are constants) is also a solution.
\end{definition}


\subsection{Initial-value problem (IVP)}
In an IVP, we seek a solution \(y(x)\) of a differential equation so that \(y(x)\) satisfies one or more initial conditions at \(x_0 \) in order to get a particular solution.

\vspace{-10pt}
\paragraph*{Existence of a unique solution}
If \(\derivative{y}{x} = f(x,y)\) and \(\partialderivative{f}{y}\) are continuous a region \(R\), then there exists a unique function \(y(x)\) that is a solution of the IVP.


\subsection{Differential equations as mathematical models}
A mathematical model is a description of a system or a phenomenon.

Steps of the modeling process:
\begin{enumerate}
  \item Make assumptions and hypotheses about the system or phenomenon
  \item Express these assumptions in terms of differential equations
  \item Get a mathematical formulation
  \item Solve the differential equations
  \item Obtain solutions
  \item Display predictions of the model (e.g. graphically)
  \item Check the model's predictions with known facts
  \item If necessary alter assumptions or increase resolution of the model
\end{enumerate}

\begin{example}
  \begin{itemize}
    \item Population dynamics
    \item Radioactive decay
    \item Newton's law of cooling and warming
    \item Spread of a disease
    \item Chemical reactions
    \item Mixtures
    \item Draining a tank
    \item Series circuits
    \item Falling bodies
    \item Falling bodies and air resistance
    \item Suspended cables
    \item \dots{} and many more
  \end{itemize}
\end{example}


\section{First-order differential equations}
\subsection{Separable equations}
A first-order differential equation of the form
\[
  \derivative{y}{x} = g(x)h(y)
\]
is said to be separable or to have separable variables.

A separable first-order differential equation can be rewritten in the form \(h(y)\diff{}{y} = g(x)\diff{}{x}\), which can be solved by integrating both sides.


\subsection{Linear equations}
A first-order differential equation of the form
\[
  a_1 (x)\derivative{y}{x} + a_0 (x)y = g(x)
\]
is said to be a linear equation in the dependent variable \(y\).

The standard form of a linear differential equation is obtained by dividing both sides by the lead coefficient \(a_1 (x)\):
\[
  \derivative{y}{x} + P(x)y = f(x)
\]

Method of solution:
\begin{enumerate}
  \item Rewrite the equation in its standard form:
        \[
          \derivative{y}{x} + P(x)y = f(x)
        \]
  \item Determine the integrating factor \(e^{\int{P(x) \diffint{x}}}\)
  \item Write
        \[
          \derivative{}{x}\left[ e^{\int{P(x) \diffint{x}}}y \right] = e^{\int{P(x) \diffint{x}}}f(x)
        \]
  \item Multiply both sides by \(\diff{}{x}\)
  \item Integrate both sides
  \item Divide both sides by the integrating factor \(e^{\int{P(x) \diffint{x}}}\) to isolate for \(y\)
\end{enumerate}

Using these steps, the solution is:
\[
  y = e^{-\int{P(x) \diffint{x}}} \int{e^{\int{P(x) \diffint{x}}}f(x) \diffint{x}}
\]


\subsubsection*{Piecewise-linear differential equation}
\begin{enumerate}
  \item Solve the initial-value problem in several parts corresponding to the intervals over which \(f(x)\) is defined
  \item Piece the two solutions together so that the solution \(y(x)\) is continuous on a larger interval
\end{enumerate}


\subsubsection*{Error functions}
Error function:
\[
  \erf(x) = \frac{2}{\sqrt{\pi}}\int_0 ^x {e^{-t^2 } \diffint{t}}
\]

Complementary error function:
\[
  \erfc(x) = \frac{2}{\sqrt{\pi}}\int_x ^{\infty}{e^{-t^2 } \diffint{t}}
\]

Property of the error functions:
\[
  \erf(x) + \erfc(x) = 1
\]


\subsection{Exact equations}
A first-order differential equation of the form
\[
  M(x,y)\diff{}{x} + N(x,y)\diff{}{y} = 0
\]
is said to be an exact equation if the expression on the left side is an exact differential.

The condition of exact differentials is:
\[
  \partialderivative{M(x, y)}{y} = \partialderivative{N(x, y)}{x}
\]

Method of solution:
\begin{enumerate}
  \item Check for exact differentials
  \item If the equation is exact, there exists a function \(f(x,y)\) such that:
        \[
          \partialderivative{f(x, y)}{x} = M(x,y) \text{ and } \partialderivative{f(x, y)}{y} = N(x,y)
        \]
  \item Find \(f(x,y)\) by integrating \(M(x,y)\) with respect to \(x\), while holding \(y\) constant.
        This gives:
        \[
          f(x,y) = \int{M(x,y)\diffint{x}} + g(y)
        \]
        where an arbitrary function \(g(y)\) is the "constant" of integration
  \item Differentiate \(f(x,y)\) with respect to \(y\) and set it equals to \(N(x,y)\):
        \[
          \partialderivative{f(x, y)}{y} = \partialderivative{}{y}\left[ \int{M(x,y)\diffint{x}} \right] + g'(y) = N(x,y)
        \]
  \item This gives:
        \[
          g'(y) = N(x,y) - \partialderivative{}{y}\left[ \int{M(x,y)\diffint{x}} \right]
        \]
  \item Integrate \(g'(y)\) with respect to \(y\)
  \item Substitute the result in \(f(x,y) = \int{M(x,y)\diffint{x}} + g(y)\)
  \item The implicit solution of \(M(x,y)\diff{}{x} + N(x,y)\diff{}{y} = 0\) is \(f(x,y) = c\)
\end{enumerate}

Therefore, if the equation is exact, the full solution becomes:
\begin{multline*}
  f(x,y) = \int{M(x,y)\diffint{x}} \\
  + \int{N(x,y) - \partialderivative{}{y}\left[ \int{M(x,y)\diffint{x}} \right]\diffint{y}} = c
\end{multline*}


\subsubsection*{Integrating factor for exact equations}
An equivalent exact equation can be created from a non-exact equation using an integrating factor \(\mu(x,y)\):
\begin{enumerate}
  \item Check for integrating factor:
        \begin{itemize}
          \item If \(\frac{M_y - N_x}{N}\) is a function of \(x\) alone, then an integrating factor for \(\mu(x,y)M(x,y)\diff{}{x} + \mu(x,y)N(x,y)\diff{}{y} = 0\) is:
                \[
                  \mu(x) = e^{\int{\frac{M_y - N_x}{N} \diffint{x}}}
                \]
          \item If \(\frac{N_x - M_y}{M}\) is a function of \(y\) alone, then an integrating factor for \(\mu(x,y)M(x,y)\diff{}{x} + \mu(x,y)N(x,y)\diff{}{y} = 0\) is:
                \[
                  \mu(y) = e^{\int{\frac{N_x - M_y}{M} \diffint{y}}}
                \]
        \end{itemize}
  \item If there exists an integrating factor \(\mu(x,y)\), then the new exact equation is
        \[
          \left[ \mu(x,y)M(x,y) \right]\diff{}{x} + \left[ \mu(x,y)N(x,y) \right]\diff{}{y} = 0
        \]
  \item Apply the method of solution of an exact equation taking \(M(x,y) = \mu(x,y)M(x,y)\) and \(N(x,y) = \mu(x,y)N(x,y)\)
\end{enumerate}


\subsection{Solution by substitution (Bernoulli, homogeneous and linear substitution)}
Substitution is used to get a differential equation in a form such that a known procedure can be used to find a solution.
The chain rule is often used as a tool to get the differential form:
\begin{enumerate}
  \item Let a function \(u(x,y)\), then \(y = g(x,u)\) and
        \[
          \derivative{y}{x} = \partialderivative{g(x, u)}{x} + \partialderivative{g(x, u)}{u}\derivative{u}{x}
        \]
  \item Solve for \(u\) using \(\derivative{u}{x}\) and another method (separable, linear or exact equations)
  \item Replace back \(u\) in \(y = g(x,u)\) to solve for \(y\)
\end{enumerate}


\subsubsection*{Homogeneous equations}
In \(M(x,y)\diff{}{x} + N(x,y)\diff{}{y} = 0\), if \(M(tx,ty) = t^{\alpha}M(x,y)\) and \(N(tx,ty) = t^{\alpha}N(x,y)\), then the equation is homogeneous.

Method of solution:
\begin{enumerate}
  \item Let \(y = ux \Rightarrow \diff{}{y} = u\diff{}{x} + x\diff{}{u}\)
  \item Replace \(y\) and \(dy\) in the equation by \(ux\) and \(u\diff{}{x} + x\diff{}{u}\) respectively
  \item Solve for \(u\) using another method (separable, linear or exact equations)
  \item Replace back \(u\) by \(\frac{y}{x}\)
  \item Solve to get an implicit or an explicit solution
\end{enumerate}


\subsubsection*{Bernoulli's equation}
A Bernoulli's equation is a differential equation of the form
\[
  \derivative{y}{x} + P(x)y = f(x)y^n , \quad n \in \realset
\]

Method of solution:
\begin{enumerate}
  \item Let \(u = y^{1 - n}\).
        The equation becomes:
        \[
          \derivative{u}{x} + (1 - n)P(x)u = (n - 1)f(x)
        \]
  \item Solve for \(u\) using the linear equation method
  \item Solve for \(y\) by using \(y = u^{\frac{1}{n - 1}}\)
\end{enumerate}


\subsubsection*{Reduction to separation of variables}
A differential equation of the form
\[
  \derivative{y}{x} = f(Ax + By + C)
\]
can always be reduced to an equation with separable variables by means of the substitution \(u = Ax + By + C\) if \(B \neq 0\).

Method of solution:
\begin{enumerate}
  \item Let \(u = Ax + By + C \Rightarrow \derivative{u}{x} = A + B \derivative{y}{x}\).
        The equation becomes:
        \[
          \derivative{u}{x} = A + B \cdot f(u)
        \]
  \item Solve for \(u\) using another method (separable, linear or exact equations)
  \item Solve for \(y\) using \(y = \frac{u - Ax - C}{B}\)
\end{enumerate}


\subsection{Linear models}
\subsubsection*{Growth and decay}
For growth and decay, the general initial value problem is
\[
  \derivative{x}{t} = kx, x(t_0 ) = x_0
\]
where \(k\) is the constant of proportionality.

The general solution of this differential equation is \(x(t) = Ce^{kt}\).

When \(t_0 = 0\), \(C = x_0 \) and the solution of the differential equation becomes \(x(t) = x_0 e^{kt}\).
In order to get the value of \(k\), another value of \(x(t)\) needs to be known.

\paragraph*{Half-life}
The half-life is a measure of the stability of a radioactive substance.
It is the time it takes for one-half of the atoms in an initial amount \(A_0 \) to disintegrate.

In half-life value problems, the solution of the differential equation is \(A(t) = A_0 e^{kt}\) with \(k = -\frac{\ln 2}{t_{\mathrm{half-life}}}\).

One of the special value of half-life to know is the half-life of carbon - 14 which is \(t_{\mathrm{C-14}} = \SI[scientific-notation = false]{5730}{\mathrm{years}}\).


\subsubsection*{Newton's law of cooling and warming}
Newton's empirical law of cooling and warming states that the rate \(\derivative{T}{t}\) at which the temperature of a body changes is proportional to the difference between the temperature \(T\) of the body and the temperature \(T_m\) of the surroundings, i.e. the ambient temperature:
\[
  \derivative{T}{t} = k(T - T_m), T(t_0 ) = T_0
\]
where \(k\) is a constant of proportionality.

The general solution of this differential equation is \(T(t) = T_m + Ce^{kt}\).

When \(t_0 = 0\), \(C = T_0 - T_m\) and the solution of the differential equation becomes \(T(t) = T_m + (T_0 - T_m)e^{kt}\).
In order to get the value of \(k\), another value of \(T(t)\) needs to be known.


\subsubsection*{Series circuits}
\paragraph*{RL circuits}
The instantaneous voltage \(v\) in a RL circuit is described by the following differential equation:
\[
  L\derivative{I}{t} + RI = V
\]

\paragraph*{RC circuits}
The instantaneous voltage \(v\) in a RC circuit is described by the following differential equation:
\[
  R \derivative{Q}{t} + \frac{1}{C}Q = V
\]

\paragraph*{RLC circuits}
The instantaneous voltage \(v\) in a RLC circuit is described by the following differential equation:
\[
  L \nderivative{2}{Q}{t} + R \derivative{Q}{t} + \frac{1}{C}Q = V
\]


\subsection{Non-linear models}
\subsubsection*{Logistic equation}
Let an environment be capable of sustaining no more than a fixed number of \(P_\mathrm{max}\) individuals in its population.
The quantity \(P_\mathrm{max}\) is called the carrying capacity of the environment.
Let \(P_0 \) be the initial population.
Then, using a density-dependent model of population dynamics, we get the logistic equation as
\[
  \derivative{P}{t} = P(a - bP)
\]
where \(a = kP_\mathrm{max}\), \(b = k\) and \(k\) a constant of proportionality.

From this equation, it follows that
\begin{align*}
  P(t) & = \frac{aP_0 }{bP_0 + (a - bP_0 )e^{-at}}                                       \\
       & = \frac{P_0 P_\mathrm{max}}{P_0 + (P_\mathrm{max} - P_0 )e^{-kP_\mathrm{max}t}}
\end{align*}

To get the graph of \(P(t)\), the second derivative of \(P\) can be used:
\begin{align*}
  \nderivative{2}{P}{t} & = 2b^2 P\left( P - \frac{a}{b} \right)\left( P - \frac{a}{2b} \right)                                                         \\
                        & = 2\left( \frac{P_0 }{P_\mathrm{max}} \right)^2 P\left( P - P_\mathrm{max} \right)\left( P - \frac{P_\mathrm{max}}{2} \right)
\end{align*}


\subsubsection*{Modification of the logistic equation}
There exist many variations of the logistic equation such as
\[
  \derivative{P}{t} = P(a - bP) \pm h
\]
where \(h\) is a rate of increasing or decreasing in the population.

Another variation of the logistic equation, known as the Gompertz differential equation, is stated as:
\[
  \derivative{P}{t} = P(a - b\ln{P})
\]
This differential equation is used as a model in the study of the growth or decline of population, growth of solid tumors\dots


\subsubsection*{Chemical reactions}
Let a quantity \(a\) of chemical \(A\) and \(b\) of chemical \(B\).
If a reaction needs \(M\) part of \(A\) and \(N\) parts of \(B\) to form a quantity \(X\) of a chemical \(C\), then the quantity of \(A\) and \(B\) remaining at any time are, respectively,
\[
  a - \frac{M}{M + N}X \text{ and } b - \frac{N}{M + N}X
\]
The rate at which chemical \(C\) is produced can be governed by a second-order differential equation of the following
\[
  \derivative{X}{t} = k(\alpha - X)(\beta - X)
\]
with \(\alpha = a\frac{M + N}{M}\) and \(\beta = b\frac{M + N}{N}\).

Method of solution:
\begin{enumerate}
  \item Compute \(\alpha = a\frac{M + N}{M}\) and \(\beta = b\frac{M + N}{N}\)
  \item Solve the differential equation using separable equation and integration of partial fractions
  \item Solve the constant \(C\) of integration using \(X_0 = 0\)
  \item Solve for the constant \(k\) of proportionality of the reaction using a known value of \(X(t)\)
  \item The maximum quantity \(X_\mathrm{max}\) of chemical C is equal to the smallest value between \(\alpha\) and \(\beta\)
\end{enumerate}


\subsubsection*{Leaking tank}
TODO: leaking tank


\section{Higher-order differential equations}
\subsection{Theory of linear equations}
A linear differential equation is an equation of the form
\begin{multline*}
  a_n (x)\nderivative{n}{y}{x} + a_{n - 1}(x)\nderivative{n - 1}{y}{x} + \cdots \\
  + a_1 (x)\derivative{y}{x} + a_0 (x)y = g(x)
\end{multline*}


\subsubsection*{Initial-value problem}
For a linear equation, an \(n\)th-order initial-value problem (IVP) is a problem where the linear equation
\begin{multline*}
  a_n (x)\nderivative{n}{y}{x} + a_{n - 1}(x)\nderivative{n - 1}{y}{x} + \cdots \\
  + a_1 (x)\derivative{y}{x} + a_0 (x)y = g(x)
\end{multline*}
is solved using the initial conditions
\[
  y(x_0 ) = y_0 , y'(x_0 ) = y_1 , \dots, y^{(n - 1)}(x_0 ) = y_{n - 1}
\]

\paragraph*{Existence of a unique solution}
Let \(a_n(x)\), \(a_{n - 1}(x)\), \dots, \(a_1 (x)\), \(a_0 (x)\) and \(g(x)\) be continuous on an interval \(I\), and let \(a_n(x) \neq 0\) for every \(x\) in this interval.
If \(x = x_0 \) is any point in this interval, then a solution \(y(x)\) of the initial-value problem exists on the interval and is unique.


\subsubsection*{Boundary-value problem}
Another type of problem, called a boundary value problem (BVP) consists of solving a linear differential equation in which the dependent variable \(y\) or its derivatives are specified a different points.
The general boundary conditions for a two-point boundary-value problem are
\begin{align*}
  A_1 y(a) + B_1 y'(a) & = C_1 \\
  A_2 y(b) + B_2 y'(b) & = C_2 \\
\end{align*}


\subsubsection*{Homogeneous equations}
A linear \(n\)th order differential equation of the form
\begin{multline*}
  a_n (x)\nderivative{n}{y}{x} + a_{n - 1}(x)\nderivative{n - 1}{y}{x} + \cdots \\
  + a_1 (x)\derivative{y}{x} + a_0 (x)y = 0
\end{multline*}
is said to be homogeneous, whereas an equation
\begin{multline*}
  a_n (x)\nderivative{n}{y}{x} + a_{n - 1}(x)\nderivative{n - 1}{y}{x} + \cdots \\
  + a_1 (x)\derivative{y}{x} + a_0 (x)y = g(x)
\end{multline*}
with \(g(x) \neq 0\) is said to be nonhomogeneous.

In order to solve a nonhomogeneous linear equation, the associated homogeneous equation must first be solvable and solved.


\subsubsection*{Differential operator}
Differential operator \(D\) transforms differentiable functions into other functions:
\[
  \derivative{}{x} \left( \derivative{y}{x} \right) = \nderivative{2}{y}{x} = D(Dy) = D^2 y
\]
and in a general manner
\[
  \nderivative{n}{y}{x} = D^n y
\]
Any linear differential equation can be expressed in terms of \(D\).
\begin{example}
  \[
    y'' + 5y' + 6y = 5x - 3 \implies \left( D^2 + 5D + 6 \right)y = 5x - 3
  \]
\end{example}


\subsubsection*{Superposition principle}
\paragraph*{Linear dependence and independence}
A set of functions \(f_1 (x)\), \(f_2 (x)\), \dots, \(f_n (x)\) is said to be linearly dependent on an interval \(I\) if there exist constants \(c_1 \), \(c_2 \), \dots, \(c_n \) not all zero, such that
\[
  c_1 f_1 (x) + c_2 f_2 (x) + \cdots + c_n f_n (x) = 0
\]
for every \(x\) in the interval.
If the set of functions is not linearly dependent on the interval, it is said to be linearly independent.
Any set of \(n\) linearly independent solutions of the homogeneous linear differential equation on interval \(I\) is a fundamental set of solutions.

\paragraph*{Superposition principle for homogeneous equations}
Let \(y_1 \), \(y_2 \), \dots, \(y_k \) be solutions of a homogeneous \(n\)th-order differential equation on an interval \(I\).
Then, the linear combination
\[
  y = c_1 y_1 (x) + c_2 y_2 (x) + \cdots + c_k y_k (x)
\]
where \(c_1 \), \(c_2 \), \dots, \(c_k \) are arbitrary constants, is also a solution on the interval.


\subsubsection*{Solutions of differential equations}
\paragraph*{Wronskian}
Suppose each of the functions \(f_1 (x)\), \(f_2 (x)\), \dots, \(f_n (x)\) posses at least \(n - 1\) derivatives.
The determinant
\[
  W(f_1 , f_2 , \dots, f_n ) =
  \begin{vmatrix}
    f_1           & f_2           & \dots & f_n           \\
    f'_1          & f'_2          & \dots & f'_n          \\
    \vdots        & \vdots        &       & \vdots        \\
    f^{(n - 1)}_1 & f^{(n - 1)}_2 & \dots & f^{(n - 1)}_n
  \end{vmatrix}
\]
where the primes denotes derivatives, is called the Wronskian of the functions.

\paragraph*{Linearly independent solutions}
Let \(y_1 \), \(y_2 \), \dots, \(y_n \) be \(n\) solutions of a homogeneous \(n\)th-order differential equation on an interval \(I\).
Then, the set of solutions is linearly independent on \(I\) if and only if \(W(y_1 , y_2 , \dots, y_n ) \neq 0\) for every \(x\) in the interval.

The general solution of a homogeneous differential equation is
\[
  y = c_1 y_1 (x) + c_2 y_2 (x) + \cdots + c_n y_n (x)
\]
where \(y_1 \), \(y_2 \), \dots, \(y_n \) is a fundamental set of solutions and \(c_1 \), \(c_2 \), \dots, \(c_k \) are arbitrary constants.

The general solution of a nonhomogeneous differential equation is
\[
  y = c_1 y_1 (x) + c_2 y_2 (x) + \cdots + c_n y_n (x) + y_p
\]
where \(y_1 \), \(y_2 \), \dots, \(y_n \) is a fundamental set of solutions, \(c_1 \), \(c_2 \), \dots, \(c_k \) are arbitrary constants and \(y_p \) is a particular solution of the differential equation.


\subsubsection*{Another superposition principle}
\paragraph*{Superposition principle for nonhomogeneous equations}
Let \(y_{p_1 }\), \(y_{p_2 }\), \dots, \(y_{p_k }\) be \(k\) particular solutions of a nonhomogeneous linear\(n\)th-order differential equation on an interval \(I\) corresponding, in turn, to \(k\) distinct functions \(g_1 \), \(g_2 \), \dots, \(g_n \).
That is, suppose that \(y_{p_i }\) denotes a particular solution of the corresponding differential equation
\begin{multline*}
  a_n (x) y^{(n)} + a_{n - 1}(x) y^{(n - 1)} + \cdots \\
  + a_1 (x) y' + a_0 (x) y = g_i (x)
\end{multline*}
where \(i = 1, 2, \dots, k\).
Then
\[
  y_p (x) = y_{p_1 } + y_{p_2 } + \cdots + y_{p_k }
\]
is a particular solution of
\begin{multline*}
  a_n (x) y^{(n)} + a_{n - 1}(x) y^{(n - 1)} + \cdots + a_1 (x) y' + a_0 (x) y \\
  = g_1 (x) + g_2 (x) + \cdots + g_k (x)
\end{multline*}


\subsection{Homogeneous linear equations with constant coefficients}
An homogeneous linear equation with constant coefficients has the following form:
\[
  a_n y^{(n)} + a_{n - 1} y^{(n - 1)} + \cdots + a_1 y' + a_0 y = 0
\]
where the coefficients \(a_0 \), \(a_1 \), \dots, \(a_{n - 1}\), \(a_n \) are real constants and \(a_n \neq 0\).


The auxiliary equation of a homogeneous linear differential equation is used to solve the differential equation it is associated with:
\[
  a_n m^n + a_{n - 1}m^{n - 1} + \cdots + a_1 m + a_0 = 0
\]
where \(m\) is used to find a solution for the homogeneous linear differential equation.

The three possible cases for the value of \(m\) are:
\begin{description}
  \item[Case \Romannumeral{1}: distinct real roots] For each real root, there is a solution \(y = ce^{mx}\)
  \item[Case \Romannumeral{2}: repeated real roots] For each repeated root, there is a solution \(y = c_1 e^{mx} + c_2 xe^{mx} + \dots + c_k x^{k - 1}e^{mx}\), where \(m\) is a root of multiplicity \(k\) (i.e. \(k\) roots are equal to \(m\)).
  \item[Case \Romannumeral{3}: conjugate complex roots] For each conjugate complex root \(m = \alpha \pm i\beta\), there is a solution \(y = e^{\alpha x}(c_1 \cos{\beta x} + c_2 \sin{\beta x})\)
\end{description}

Method of solution:
\begin{enumerate}
  \item Build the auxiliary equation associated with the differential equation
  \item Solve for for all values of \(m\)
  \item Follow the rule for the three possible cases of \(m\)
  \item Sum all the solutions in order to get the full solution \(y_c \)
\end{enumerate}

The most difficult step to solve the auxiliary equation is to find some roots in order to reduce the equation to a factorized form with only 2nd order terms.

Method of solution:
\begin{enumerate}
  \item Find all the multiple (positive and negative) of the last coefficient \(a_0 \) as \(p\)
  \item Find all the multiple (positive and negative) of the first coefficient \(a_n \) as \(q\)
  \item Build the values \(\frac{p}{q}\) which are possibles solutions of the auxiliary equation
  \item Test the values \(\frac{p}{q}\) until finding one that solves the auxiliary equation
  \item Factor \(m - \frac{p}{q}\) of the equation using long division
  \item Repeat the process until having only 2nd order terms
\end{enumerate}


\subsubsection*{Two equations worth knowing}
Two differential equations
\[
  y'' + k^2 y = 0 \text{ and } y'' - k^2 y = 0, \quad k \in \realset
\]

The general solution of \(y'' + k^2 y = 0\) is
\[
  y_c = c_1 \cos{kx} + c_2 \sin{kx}
\]
and the general solution of  \(y'' - k^2 y = 0\) is
\[
  y_c = c_1 e^{kx} + c_2 e^{-kx} = c_3 \cosh{kx} + c_4 \sinh{kx}
\]


\subsection{Undetermined coefficients}
To solve a nonhomogeneous linear differential equation
\[
  a_n y^{(n)} + a_{n - 1} y^{(n - 1)} + \cdots + a_1 y' + a_0 y = g(x)
\]
two things must be done:
\begin{enumerate}
  \item Find the complementary function \(y_c \)
  \item Find any particular solution \(y_p \) of the nonhomogeneous equation
  \item Sum \(y_c \) and \(y_p \) ot get the general solution \(y = y_c + y_p \)
\end{enumerate}

The method of undetermined coefficients is limited to nonhomogeneous linear differential equations where
\begin{itemize}
  \item The coefficients \(a_0 \), \(a_1 \), \dots, \(a_{n - 1}\), \(a_n \) are real constants
  \item \(g(x)\) is a linear combination of functions of the type
        \begin{align*}
           & P(x) = a_n x^n + a_{n - 1} x^{n - 1} + \cdots + a_1 x + a_0 \\
           & P(x)e^{\alpha x}                                            \\
           & P(x)e^{\alpha x}\cos{\beta x}                               \\
           & P(x)e^{\alpha x}\sin{\beta x}
        \end{align*}
        where \(n \in \realset^+ \) and \((\alpha, \beta) \in \realset\)
\end{itemize}

There are several cases to find \(y_p \):
\begin{description}
  \item[Case \Romannumeral{1}] No function in \(y_p \) is found in \(y_c \)
  \item[Case \Romannumeral{2}] A function in \(y_p \) is found in \(y_c \)
\end{description}

The form of \(y_p \) can be found using the following table.
% 1cm = 10mm = 28pt = 1/2.54in
\begin{table}[h!] % Options: b (bottom), t (top), h (here), H (HERE!), ! (insist)
  \caption{Trial particular solutions}
  % \label{Trial particular solutions}
  \begin{center}
    \centering % Horizontal alignment of the table
    \begin{tabular}{ % Number of letter (l: left, c: center, r: right) = number of column
        ll
      }
      % Visible row border: \hline (needed for each row)
      % Visible column border: | next to tabular declaration (needed for each column)
      % Column separation: &, row separation: \\

      \(g(x)\)                            & Form of \(y_p \)                  \\
      \hline
      Constant                            & \(A\)                             \\
      \(a_n x^n + \cdots + a_1 x + a_0 \) & \(A x^n + \cdots + Y x + Z\)      \\
      \(a\cos{bx}\)                       & \(A\cos{bx} + B\sin{bx}\)         \\
      \(a\sin{bx}\)                       & \(A\cos{bx} + B\sin{bx}\)         \\
      \(ae^{bx}\)                         & \(Ae^{bx}\)                       \\
      \((a_n x^n + \cdots + a_0 )e^{bx}\) & \((A x^n + \cdots + Z)e^{bx}\)    \\
      \(ae^{bx}\cos{cx}\)                 & \(e^{bx}(A\cos{cx} + B\sin{cx})\) \\
    \end{tabular}
  \end{center}
\end{table}

\paragraph*{Multiplication rule}
If function of \(y_p \) contains terms that duplicate terms in \(y_c \), then that function must be multiplied by \(x^n \), where \(n\) is the smallest positive integer that eliminates that duplication.

Method of solution
\begin{enumerate}
  \item Find the form of \(y_p \)
        \begin{itemize}
          \item If no function in \(y_p \) is found in \(y_c \), then no more work is needed
          \item If a function in \(y_p \) is found in \(y_c \), then the multiplication rule must be applied
        \end{itemize}
  \item Solve for the unknowns of \(y_p \) using the differential equation and the derivatives of \(y_p \)
\end{enumerate}


\subsection{Variation of parameters}
Unlike the method of undetermined coefficients, the variation of parameters method is not limited to special function of \(g(x)\), not is it limited to differential equations with constant coefficients.

Method of solution:
\begin{enumerate}
  \item Rewrite the differential equation in its standard form (with a coefficient of 1 in front of the biggest derivative)
        \[
          y^{(n)} + P_{n - 1}(x) y^{(n - 1)} + \cdots + P_1 (x) y' + P_0 (x) y = f(x)
        \]
  \item Solve for \(y_c \), with
        \[
          y_c = c_1 y_1 + c_2 y_2 + \cdots + c_n y_n
        \]
        where \(y_1 \), \(y_2 \), \dots, \(y_n \) constitute a linearly independent set of solutions of the associated homogeneous differential equation.
        The particular solution \(y_p \) will be of the form
        \[
          y_p = u_1 y_1 + u_2 y_2 + \cdots + u_n y_n
        \]
        where \(u\) is a function of \(x\)
  \item Solve for \(u_1 \), \(u_2 \), \dots, \(u_n \) using their derivative and the following formula
        \[
          u'_k = \frac{W_k }{W} \text{, } k = 1, 2, \dots, n
        \]
        where \(W\) is the Wronskian of \(y_1 \), \(y_2 \), \dots, \(y_n \) and \(W_k \) is the determinant obtained by replacing the \(k\)th column of the Wronskian by the column \((0, 0, \dots, f(x))\)
\end{enumerate}


\subsection{Cauchy-Euler equations}
Any linear differential equation of the form
\[
  a_n x^n y^{(n)} + a_{n - 1}x^{n - 1}y^{(n - 1)} + \cdots + a_1 xy' + a_0 y = g(x)
\]
where the coefficients \(a_n \), \(a_{n - 1}\), \dots, \(a_0 \) are constants, is known as a Cauchy-Euler equation or an equidimensional equation.

The auxiliary equation of a Cauchy-Euler equation is used to solve the differential equation it is associated with.
In order to get the auxiliary equation, the substitution \(y = x^m \) is done, and the differential equation is set equal to 0.

\begin{example}
  For a second-order Cauchy-Euler equation, we would get
  \begin{alignat*}{2}
     &                           & am(m - 1) + bm + c  & = 0 \\
     & \Longleftrightarrow \quad & am^2 + (b - a)m + c & = 0
  \end{alignat*}
  where \(m\) is used to find a solution for the Cauchy-Euler equation.
\end{example}

The three possible cases for the value of \(m\) are:
\begin{description}
  \item[Case \Romannumeral{1}: distinct real roots] For each real root, there is a solution \(y = cx^m \)
  \item[Case \Romannumeral{2}: repeated real roots] For each repeated root, there is a solution \(y = c_1 x^m + c_2 x^m \ln{x} + \dots + c_k x^m (\ln{x})^{k-1}\), where \(m\) is a root of multiplicity \(k\) (i.e. \(k\) roots are equal to \(m\)).
  \item[Case \Romannumeral{3}: conjugate complex roots] For each conjugate complex root \(m = \alpha \pm i\beta\), there is a solution \(y = x^{\alpha}\left[ c_1 \cos(\beta\ln{x}) + c_2 \sin(\beta\ln{x}) \right]\)
\end{description}

Method of solution:
\begin{enumerate}
  \item Build the auxiliary equation associated with the Cauchy-Euler equation
  \item Solve for for all values of \(m\)
  \item Follow the rule for the three possible cases of \(m\)
  \item Sum all the solutions in order to get the full solution \(y_c \)
  \item Get \(y_p \) using the method of variation of parameters
\end{enumerate}

A generalization of this method can be made by replacing \(x\) by \(x - x_0 \).


\subsection{Nonlinear equations}
In order to solve nonlinear equations, several method exists.
One of them known as the reduction of order may be used to find solutions of special kinds of nonlinear second-order differential equations:
\begin{itemize}
  \item When the dependent variable \(y\) is missing and we have \(F(x, y', y'') = 0\)
  \item When the independent variable \(x\) is missing and we have \(F(y, y', y'') = 0\)
\end{itemize}

In each case, the differential equation can be reduced to a first-order equation by the substitution \(u = \derivative{y}{x}\).
In the first case, the differential equation becomes \(F(x, u, u') = 0\), where \(\derivative{y}{x} = u\) and \(\nderivative{2}{y}{x} = \derivative{u}{x}\).
In the second case, the differential equation becomes \(F(y, u, u') = 0\), where \(y\) is the new independent variable and \(\derivative{y}{x} = u\) and \(\nderivative{2}{y}{x} = u\derivative{u}{y}\).


\subsection{Linear Models: initial-value problems (IVP)}
\subsubsection*{Spring-mass systems: free undamped motion}
For a free undamped motion of a spring-mass system of spring constant \(k\) and mass \(m\), using Hooke's law (\(F = -ks\), where \(k \in \realset^*_+\)) and Newton's Second law (\(F = ma\)), the following second order differential equation can be obtained
\[
  \nderivative{2}{x}{t} + \omega^2 x = 0 \text{, where }\omega^2 = \frac{k}{m}
\]
This equation describes a simple harmonic motion, also called free undamped motion.

The solution of this differential equation is
\[
  x(t) = c_1 \cos{\omega t} + c_2 \sin{\omega t}
\]
The period of free vibrations is
\[
  T = \frac{2\pi}{\omega}
\]
and the frequency is
\[
  f = \frac{1}{T} = \frac{\omega}{2\pi}
\]

An alternative form of \(x(t)\) is used to find the amplitude of free vibrations \(A\) and the phase angle \(\phi\) of the system:
\begin{gather*}
  y = A\sin(\omega t + \phi) = A\cos(\omega t + \phi - \frac{\pi}{2}) \\
  \text{where } A = \sqrt{c_1 ^2 + c_2 ^2 } \text{ and } \phi = \arctan\left( \frac{c_1 }{c_2 } \right)
\end{gather*}

\paragraph*{Multiple spring systems}
If the springs are in parallel, we have \(k_{\mathrm{equivalent}} = k_1 + k_2 + \cdots\).
If the springs are in series, we have \(\frac{1}{k_{\mathrm{equivalent}}} = \frac{1}{k_1}+ \frac{1}{k_2} + \cdots\).

\paragraph*{Systems with variable spring constants}
The above solution does not apply for systems with variable spring constants, e.g. aging spring or environments with rapid temperature change.


\subsubsection*{Spring and mass systems: free damped motion}
For a free damped motion of a spring-mass system of spring constant \(k\), mass \(m\) and damping factor \(\beta\), the following second order differential equation can be obtained
\begin{gather*}
  \nderivative{2}{x}{t} + 2\lambda\derivative{x}{t} + \omega^2 x = 0 \\
  \text{where } 2\lambda = \frac{\beta}{m} \text{ and } \omega^2 = \frac{k}{m}
\end{gather*}
This equation describes a free damped motion.

There are three possible cases:
\begin{description}
  \item[Case \Romannumeral{1}: \(\lambda^2 - \omega^2 > 0\)] The system is said to be overdamped, it results in a smooth and nonoscillatory motion and the solution is
        \[
          x(t) = e^{-\lambda t} \left( c_1 e^{t\sqrt{\lambda^2 - \omega^2 }} + c_2 e^{-t\sqrt{\lambda^2 - \omega^2 }} \right)
        \]
  \item[Case \Romannumeral{2}: \(\lambda^2 - \omega^2 = 0\)] The system is said to be critically damped and the solution is
        \[
          x(t) = e^{-\lambda t} (c_1 + c_2 t)
        \]
  \item[Case \Romannumeral{3}: \(\lambda^2 - \omega^2 < 0\)] The system is said to be underdamped, it results in an oscillatory motion fading over time and the solution is
        \[
          x(t) = e^{-\lambda t} \left( c_1 \cos{t\sqrt{\omega^2 - \lambda^2 }} + c_2 \sin{t\sqrt{\omega^2 - \lambda^2 }} \right)
        \]
\end{description}

In case \Romannumeral{3} (\(\lambda^2 - \omega^2 < 0\)), an alternative form of \(x(t)\) is used to find the amplitude of free vibrations \(A\) and the phase angle \(\phi\) of the system:
\begin{gather*}
  y = A\sin(t\sqrt{\omega^2 - \lambda^2 } + \phi) \\
  \text{where } A = \sqrt{c_1 ^2 + c_2 ^2 } \text{ and } \phi = \arctan\left( \frac{c_1 }{c_2 } \right)
\end{gather*}


\subsubsection*{Driven motion}
The inclusion of another force \(f(t)\) in the formulation of Newton's second law gives the differential equation of driven of forced motion:
\begin{gather*}
  \nderivative{2}{x}{t} + 2\lambda\derivative{x}{t} + \omega^2 x = F(t) \\
  \text{where } 2\lambda = \frac{\beta}{m} \text{, } \omega^2 = \frac{k}{m} \text{ and } F(t) = \frac{f(t)}{m}
\end{gather*}

Method of solution:
\begin{enumerate}
  \item Find \(x_c \) of the homogeneous equation using the solutions found for the free damped motion
  \item Find \(x_p \) using the method of undetermined coefficients or variation of parameters
\end{enumerate}

Method of solution for driven motion without damping (pure resonance):
\begin{gather*}
  \nderivative{2}{x}{t} + \omega^2 x = F_0 \sin{\gamma t} \\
  \text{where } \omega^2 = \frac{k}{m}
\end{gather*}
\begin{enumerate}
  \item Find \(x_c \) of the homogeneous equation using the solutions found for the free undamped motion
  \item Find \(x_p \) using the method of undetermined coefficients or variation of parameters
  \item Determine the condition of pure resonance of the system by letting \(\gamma = \omega\)
\end{enumerate}

If the initial conditions are \(x_0 = 0\) and \(x'_0 = 0\), then for \(\gamma \neq \omega\), the solution is
\[
  x(t) = \frac{F_0 }{\omega\left( \omega^2 - \gamma^2 \right)}(- \gamma\sin{\omega t} + \omega\sin{\gamma t})
\]
and for \(\gamma = \omega\), the solution is
\[
  x(t) = \frac{F_0 }{2\omega^2 }\sin{\omega t} - \frac{F_0 }{2\omega}t\cos{\omega t}
\]


\subsubsection*{RLC-series circuits}
Using Kirchhoff's loop rule, the following linear second-order differential equation is found:
\[
  L\nderivative{2}{Q}{t} + R\derivative{Q}{t} + \frac{1}{C}Q = V(t)
\]

The equation can be solved using previous solution method such as auxiliary equation, undetermined coefficient or variation of parameters.


\section{Series solutions of linear equations}
\subsection{Solutions about ordinary points}
\subsubsection*{Review of power series}
A power series centered at \(a\) is an infinite series of the form
\[
  \sum_{n = 0}^{\infty}{c_n (x - a)^n } = c_0 + c_1 (x - a) + c_2 (x - a)^2 + \cdots
\]

Power series characteristics:
\begin{itemize}
  \item Convergence: the convergence of a power series can be checked by the ration test
        \[
          \limit{n}{\infty}{\left| \frac{c_{n + 1}(x - a)^{n + 1}}{c_n (x - a)^n } \right|} = |x - a|\limit{n}{\infty}{\left| \frac{c_{n + 1}}{c_n } \right|} = L
        \]
        \begin{description}
          \item[\(L < 1\)] The series converges absolutely
          \item[\(L = 1\)] The test is inconclusive
          \item[\(L > 1\)] The series diverges
        \end{description}
  \item Interval of convergence: the set of all real numbers for which the series converges
  \item Radius of convergence \(R\): the set of all real numbers for which the series converges, centered at a constant \(a\).
        The interval is then inside \(a - R\) and \(a + R\).
  \item Analytic: a function is analytic at a point \(a\) if a it can be represented by a power series centered at \(a\) with \(R \in \realset^+ \)
        \begin{example}
          \begin{alignat*}{2}
            e^x     & = \sum_ {n = 0}^{\infty}{\frac{x^n }{n!}}                      & = 1 + \frac{x}{1!} + \frac{x^2 }{2!} + \cdots    \\
            \sin{x} & = \sum_ {n = 0}^{\infty}{\frac{(- 1)^n x^{2n + 1}}{(2n + 1)!}} & = x - \frac{x^3 }{3!} + \frac{x^5 }{5!} - \cdots \\
            \sin{x} & = \sum_ {n = 0}^{\infty}{\frac{(- 1)^n x^{2n}}{(2n)!}}         & = 1 - \frac{x^2 }{2!} + \frac{x^4 }{4!} - \cdots
          \end{alignat*}
        \end{example}
\end{itemize}

Power series can be added if the power of \(x\) in each sum are "in phase" and if the summation indices start with the same number.
Method to add power series:
\begin{enumerate}
  \item Identify and check the current power of \(x\) for all the series
  \item Pull terms out of the series with the lowest power of \(x\) until it matches the power of the highest
  \item Introduce a new counter \(k = f(n)\) equal to the power of \(x\) in the series
  \item Replace the old counter \(n\) with the new one \(k\)
  \item Combine the series
\end{enumerate}


\subsubsection*{Power series solutions}
\paragraph*{Ordinary and singular points}
A point \(x_0 \) is said to be an ordinary point of the differential equation if all the coefficients in the standard form are analytic at \(x_0 \).
A point that is not an ordinary point is said to be a singular point of the equation.

\paragraph*{Existence of power series solutions}
If \(x = x_0 \) is an ordinary point of the differential equation, we can always find linearly independent solutions in the form of a power series centered at \(x_0 \).
A series solution converges at least on some interval defined by \(|x - x_0 | < R\), where \(R\) is the distance from \(x_0 \) to the closest singular point.

Method of solution (method of undetermined series coefficients at \(x = 0\)):
\begin{enumerate}
  \item Check for the existence of a solution at \(x = 0\)
  \item Set the solution as
        \begin{gather*}
          y = \sum_ {n = 0}^{\infty}{c_n x^n } \implies y' = \sum_ {n = 1}^{\infty}{c_n nx^{n-1}}\\
          \implies y'' = \sum_ {n = 2}^{\infty}{c_n n(n - 1)x^{n-2}}
        \end{gather*}
  \item Replace \(y\), \(y'\) and \(y''\) in the DE
  \item If a coefficient is non-polynomial, replace it with it's polynomial form (for \(\sin{x}\), \(\cos{x}\) and \(e^x \))
  \item Combine the series into one sum
  \item Apply the identity property (set all terms equal to 0)
  \item Get a recursion formula for the \(c\)'s
  \item Solve for the \(c\)'s with different values of \(k\)
  \item Combine the \(c\)'s into a unique \(y\) solution
  \item Factor out the common \(c\)'s
  \item Find a power series formula for each \(c\)
  \item The distinct power series are the independent solutions expanded about the ordinary point \(x = 0\)
\end{enumerate}


\section{Systems of linear DEs}
\subsection{Theory of linear systems}
The normal form of a first-order system of linear equations is:
\begin{align*}
  \derivative{x_1 }{t} & = a_{11}x_1 + a_{12}x_2 + \cdots + a_{1n}x_n + f_1 (t) \\
  \derivative{x_2 }{t} & = a_{21}x_1 + a_{22}x_2 + \cdots + a_{2n}x_n + f_2 (t) \\
  \derivative{x_n }{t} & = a_{n1}x_1 + a_{n2}x_2 + \cdots + a_{nn}x_n + f_n (t)
\end{align*}

This system can be represented by a matrix equation \(\vec{\mathrm{X}}' = \mathbf{A}\vec{\mathrm{X}} + \vec{\mathrm{F}}\).
If the system is homogeneous, its matrix form is then \(\vec{\mathrm{X}}' = \mathbf{A}\vec{\mathrm{X}}\).

\paragraph*{Existence of a unique solution}
Let the entries of the matrices \(\mathbf{A}(t)\) and \(\vec{\mathrm{F}}(t)\) be functions continuous on a common interval \(I\) that contains the point \(t_0 \).
Then there exists a unique solution of the initial-value problem on the interval.


\subsubsection*{Homogeneous systems}
\paragraph*{Superposition principle}
Let \(\vec{\mathrm{X}}_1 \), \(\vec{\mathrm{X}}_2 \), \dots, \(\vec{\mathrm{X}}_k \) be a set of solution vectors of a homogeneous system on an interval \(I\).
Then the linear combination
\[
  \vec{\mathrm{X}} = c_1 \vec{\mathrm{X}}_1 + c_2 \vec{\mathrm{X}}_2 + \cdots + c_k \vec{\mathrm{X}}_k
\]
where \(c_1 \), \(c_2 \), \dots, \(c_k \) are arbitrary constants, is also a solution of the system on the interval.

In order to check for linearly independent solutions, the Wronskian of \(\vec{\mathrm{X}}_1 \), \(\vec{\mathrm{X}}_2 \), \dots, \(\vec{\mathrm{X}}_k \) is computed:
\begin{itemize}
  \item If \(W \neq 0\), the solutions vectors are independent
  \item If \(W = 0\), the solutions vectors are dependent
\end{itemize}

\paragraph*{General solution of a homogeneous system}
Let \(\vec{\mathrm{X}}_1 \), \(\vec{\mathrm{X}}_2 \), \dots, \(\vec{\mathrm{X}}_n \) be a fundamental set of solutions, i.e a set independent solution vectors, of a homogeneous system on an interval \(I\).
Then the general solution of the system on the interval is
\[
  \vec{\mathrm{X}} = c_1 \vec{\mathrm{X}}_1 + c_2 \vec{\mathrm{X}}_2 + \cdots + c_n \vec{\mathrm{X}}_n
\]
where \(c_1 \), \(c_2 \), \dots, \(c_n \) are arbitrary constants.


\subsubsection*{Nonhomogeneous systems}
\paragraph*{General solution of a nonhomogeneous system}
The general solution \(\vec{\mathrm{X}}\) of a nonhomogeneous system is the sum of the general solution \(\vec{\mathrm{X}}_c \) of the associated homogeneous system and a particular solution \(\vec{\mathrm{X}}_p \):
\[
  \vec{\mathrm{X}} = \vec{\mathrm{X}}_c + \vec{\mathrm{X}}_p
\]


\subsection{Homogeneous linear systems}
The matrix form of a homogeneous linear system is \(\vec{\mathrm{X}}' = \mathbf{A}\vec{\mathrm{X}}\) and the general solution is of the form
\begin{gather*}
  \vec{\mathrm{X}} = c_1 \vec{\mathrm{X}}_1 + c_2 \vec{\mathrm{X}}_2 + \cdots \\
  \text{where } \vec{\mathrm{X}}_i = \vec{\mathrm{K}}_i e^{\lambda_i t}
\end{gather*}

This leads to solve this system as an eigenvalue and eigenvector problem, where the eigenvalues \(\lambda_i \) of \(\mathbf{A}\) are determined using the following characteristic equation:
\[
  \det(\mathbf{A} - \lambda\mathbf{I}) = 0
\]
From those eigenvalues, the corresponding eigenvectors \(\vec{\mathrm{K}}_i \) of \(\mathbf{A}\) can be determined using
\[
  (\mathbf{A} - \lambda\mathbf{I})\vec{\mathrm{K}} = \vec{0}
\]

Method of solution:
\begin{itemize}
  \item Determine the eigenvalues \(\lambda_i \) using \(\det(\mathbf{A} - \lambda\mathbf{I}) = 0\)
  \item For each eigenvalue, determine an eigenvector using \((\mathbf{A} - \lambda\mathbf{I})\vec{\mathrm{K}} = \vec{0}\)
  \item Apply the rule for one of the four possible cases:
        \begin{description}
          \item[Case \Romannumeral{1}] Distinct \(\lambda_i \)'s: the final answer is of the form
                \[
                  \vec{\mathrm{X}} = c_1 \vec{\mathrm{K}}_1 e^{\lambda_1 t} + c_2 \vec{\mathrm{K}}_2 e^{\lambda_2 t} + \cdots + c_n \vec{\mathrm{K}}_n e^{\lambda_n t}
                \]
          \item[Case \Romannumeral{2}] Repeated \(\lambda_i \)'s and distinct \(\vec{\mathrm{K}}_i \)'s: the final answer is of the form
                \[
                  \vec{\mathrm{X}} = c_1 \vec{\mathrm{K}}_1 e^{\lambda t} + c_2 \vec{\mathrm{K}}_2 e^{\lambda t} + \cdots + c_n \vec{\mathrm{K}}_1 e^{\lambda t}
                \]
          \item[Case \Romannumeral{3}] Repeated \(\lambda_i \)'s and \(\vec{\mathrm{K}}_i \)'s: the final answer is of the form
                \begin{gather*}
                  \vec{\mathrm{X}} = c_1 \vec{\mathrm{X}}_1 + c_2 \vec{\mathrm{X}}_2 + \cdots \\
                  \text{where } \vec{\mathrm{X}}_1 = e^{\lambda t} \text{, } \vec{\mathrm{X}}_2 = \vec{\mathrm{K}}te^{\lambda t} + \vec{\mathrm{P}}e^{\lambda t} \text{, until}\\
                  \vec{\mathrm{X}}_n = \vec{\mathrm{K}}\frac{t^{n - 1}}{(n - 1)!}e^{\lambda t} + \vec{\mathrm{P}}\frac{t^{n - 2}}{(n - 2)!}e^{\lambda t}  + \cdots + e^{\lambda t}
                \end{gather*}
          \item[Case \Romannumeral{4}] Complex \(\lambda_i \)'s: the final answer is of the form
                \begin{align*}
                  \vec{\mathrm{X}}                 & = c_1 \vec{\mathrm{X}}_1 + c_2 \vec{\mathrm{X}}_2                                                                                  \\
                  \text{where } \vec{\mathrm{X}}_1 & = e^{\alpha t}\left[ \Re\left( \vec{\mathrm{K}}_1 \right)\cos{\beta t} - \Im\left( \vec{\mathrm{K}}_1 \right)\sin{\beta t} \right] \\
                  \vec{\mathrm{X}}_2               & = e^{\alpha t}\left[ \Im\left( \vec{\mathrm{K}}_1 \right)\cos{\beta t} + \Re\left( \vec{\mathrm{K}}_1 \right)\sin{\beta t} \right] \\
                \end{align*}
        \end{description}
\end{itemize}

In case \Romannumeral{3}, to determine \(\vec{\mathrm{K}}\), \(\vec{\mathrm{P}}\), \dots, \(\vec{\mathrm{Q}}\), the following characteristic equations need to be used:
\begin{align*}
  (\mathbf{A} - \lambda\mathbf{I})\vec{\mathrm{K}} & = \vec{0}          \\
  (\mathbf{A} - \lambda\mathbf{I})\vec{\mathrm{P}} & = \vec{\mathrm{K}} \\
  (\mathbf{A} - \lambda\mathbf{I})\vec{\mathrm{Q}} & = \vec{\mathrm{P}}
\end{align*}


\section{Functions of a complex variable}
\subsection{Complex numbers}
\subsubsection*{Terminology}
A complex number is any number of the form  where \(a\) and \(b\) are real numbers and \(i\) is the imaginary number such that \(i^2 = -1\).

\paragraph*{Equality}
Complex numbers \(z_1 = a_1 + ib_1 \) and \(z_2 = a_2 + ib_2 \) are equal if
\[
  \Re(z_1 ) = \Re(z_2 ) \text{ and } \Im(z_1 ) = \Im(z_2 )
\]
\subsubsection*{Arithmetic operations}
\begin{description}
  \item[Addition] \(z_1 + z_2 = (a_1 + a_2 ) + i(b_1 + b_2 )\)
  \item[Subtraction] \(z_1 - z_2 = (a_1 - a_2 ) + i(b_1 - b_2 )\)
  \item[Multiplication] \(z_1 \cdot z_2 = (a_1 a_2 - b_1 b_2 ) + i(b_1 a_2 + b_2 a_1 )\)
  \item[Division] \(\frac{z_1 }{z_2 } = \left( \frac{a_1 a_2 + b_1 b_2 }{a_2 ^2 + b_2 ^2 } \right) + i\left( \frac{b_1 a_2 - b_2 a_1 }{a_2 ^2 + b_2 ^2 } \right)\)
  \item[Modulus] \(| z | = \sqrt{a^2 + b^2 }\)
\end{description}


\subsubsection*{Conjugate}
If \(z = a + ib\), then its conjugate is \(\bar{z} = a - ib\).

The arithmetic operations are the same and it can be shown that
\begin{align*}
  \overline{z_1 + z_2 }                       & = \bar{z}_1 + \bar{z}_2         \\
  \overline{z_1 - z_2 }                       & = \bar{z}_1 - \bar{z}_2         \\
  \overline{z_1 \cdot z_2 }                   & = \bar{z}_1 \cdot \bar{z}_2     \\
  \overline{\left( \frac{z_1 }{z_2 } \right)} & = \frac{\bar{z}_1 }{\bar{z}_2 }
\end{align*}

Then using a complex number \(z\) and its conjugate \(\bar{z}\), the following can be shown
\begin{align*}
  z + \bar{z}     & = 2a        \\
  z - \bar{z}     & = 2bi       \\
  z \cdot \bar{z} & = a^2 + b^2
\end{align*}



\subsection{Powers and roots}
\subsubsection*{Polar form of complex numbers}
A complex number \(z\) can be expressed in polar coordinates \((r,\theta)\) as
\[
  z = r(\cos\theta + i\sin\theta) = re^{i\theta}
\]
where \(r\) is the modulus of \(z\) and \(\theta\) is its argument, written as \(r = |z|\) and \(\theta = \arg{z}\).


Multiplication and division with polar form:
\begin{alignat*}{2}
  z_1 \cdot z_2     & = r_1 r_2 \left[ \cos(\theta_1 + \theta_2 ) + i\sin(\theta_1 + \theta_2 ) \right]          & = & r_1 r_2 e^{i(\theta_1 + \theta_2 )}          \\
  \frac{z_1 }{z_2 } & = \frac{r_1 }{r_2 }\left[ \cos(\theta_1 - \theta_2 ) + i\sin(\theta_1 - \theta_2 ) \right] & = & \frac{r_1 }{r_2 }e^{i(\theta_1 - \theta_2 )}
\end{alignat*}


\subsubsection*{Powers of \(z\)}
\paragraph*{Integer powers of \(z\)}
The polar form of a complex number is useful to manage powers of \(z\) more easily, as the formula is
\[
  z^n = r^n \left( \cos n\theta + i\sin n\theta \right), \quad n \in \integerset
\]

\paragraph*{Roots of \(z\)}
A number \(w\) is said to be a \(n\)th root of a nonzero complex number \(z\) if \(w^n = z\).
The \(n\) \(n\)th roots of a nonzero complex number \(z = r(\cos\theta + i\sin\theta) = re^{i\theta}\) are given by
\begin{gather*}
  \begin{aligned}
    w_k & = \sqrt[n]{r}\left[ \cos\left( \frac{\theta + 2k\pi}{n} \right) + i\sin\left( \frac{\theta + 2k\pi}{n} \right) \right] \\
        & = \sqrt[n]{r}e^{i\frac{\theta + 2k\pi}{n}}
  \end{aligned} \\
  \text{where } k = 0, 1, 2, \dots, n - 1
\end{gather*}
\end{document}