\documentclass[10pt, twocolumn]{article}

%%%%%%%%%%%%%%%%%%%%%%%%%%%%%%%%%%%%%%%%%%%%%%%%%%%%%%%%%%%%%%%%%%%%%%%%%%%%%%%
%%%% Cover page
\title{ENGR 242: Statics}
\date{\today}
\author{Anthony Bourboujas}

\makeatletter
\let\Title\@title
\let\Author\@author
\let\Date\@date
\makeatother

%%%%%%%%%%%%%%%%%%%%%%%%%%%%%%%%%%%%%%%%%%%%%%%%%%%%%%%%%%%%%%%%%%%%%%%%%%%%%%%
%%%% Preamble
%%%%%%%%%%%%%%%%%%%%%%%%%%%%%%%%%%%%%%%%%%%%%%%%%%%%%%%%%%%%%%%%%%%%%%%%%%%%%%%
%%%% Packages
\usepackage[utf8x]{inputenc} % Accept different input encodings
\usepackage[T1]{fontenc} % Standard package for selecting font encodings
\usepackage{lmodern} % Font name; classic: lmodern
\usepackage[english]{babel} % Multilingual support for LaTeX
% \usepackage{abstract} % Control the typesetting of the abstract environment
\usepackage{amsmath} % AMS mathematical facilities for LaTeX
\usepackage{amssymb} % TeX fonts from the American Mathematical Society
\usepackage{amsthm} % Typesetting theorems (AMS style)
\usepackage{array} % Extending the array and tabular environments
% \usepackage[backend=biber,style=ieee,sorting=none]{biblatex}
\usepackage{bold-extra} % Use bold small caps and typewriter fonts
\usepackage{cellspace} % Ensure minimal spacing for table cells
\usepackage{chemformula} % Command for typesetting chemical formulas and reactions
% \usepackage{colortbl} % Add colour to LaTeX tables
\usepackage{comment} % Selectively include/exclude portions of text
\usepackage{csquotes} % Context sensitive quotation facilities
% \usepackage[en-US,showdow]{datetime2} % Formats for dates, times and time zones
% \usepackage{diagbox} % Table heads with diagonal lines
\usepackage{enumitem} % Control layout of itemize, enumerate, description
\usepackage{esint} % Extended set of integrals for Computer Modern
\usepackage{graphicx} % Enhanced support for graphics
% \usepackage{listings} % Typeset source code listings using LaTeX
% \usepackage{lipsum} % Easy access to the Lorem Ipsum dummy text
\usepackage{mathrsfs} % Support for using RSFS fonts in maths
% \usepackage{matlab-prettifier} % Pretty-print Matlab source code
\usepackage{moreverb} % Extended verbatim
\usepackage{multicol} % Intermix single and multiple columns
\usepackage{multirow} % Create tabular cells spanning multiple rows
% \usepackage{pgfplots} % Plots
% \usepackage{pgfplotstable} % Loads, rounds, format and post-processes numerical tables (generates table from CSV)
% \usepackage{pdfpages} % Include PDF document in LaTeX
% \usepackage{rotating} % Rotation tools, including rotated full-page floats with sidewaysfigure
\usepackage[scr]{rsfso} % A mathematical calligraphic font based on rsfs
\usepackage{setspace} % Set space between lines
\usepackage{soul} % Hyphenation for letterspacing, underlining, and more
\usepackage{threeparttable} % Tables with captions and notes all the same width
% \usepackage{verbatim} % Reimplementation of and extensions to LaTeX verbatim
\usepackage{wrapfig} % Produces figures which text can flow around
\usepackage{xcolor} % Driver-independent color extensions for LaTeX
\usepackage{xurl} % Verbatim with URL-sensitive line breaks, allow URL breaks at any alphanumerical character

%%%%%%%%%%%%%%%%%%%%%%%%%%%%%%%%%%%%%%%%%%%%%%%%%%%%%%%%%%%%%%%%%%%%%%%%%%%%%%%
%%%% Lengths
% 1cm = 10mm = 28pt = 1/2.54in
% 1ex = height of a lowercase 'x' in the current font
% 1em = width of an uppercase 'M' in the current font

%%%% Spacing in math mode
% \!                         = -3/18em
% \,                         = 3/18em
% \:                         = 4/18em
% \;                         = 5/18em
% \ (space after backslash!) = space in normal text
% \quad                      = 1em
% \qquad                     = 2em

% \setlength{\baselineskip}{1em} % Vertical distance between lines in a paragraph
% \renewcommand{\baselinestretch}{1.0} % A factor multiplying \baslineskip
\setlength{\columnsep}{0.75cm} % Distance between columns
% \setlength{\columnwidth}{} % The width of a column
\setlength{\columnseprule}{1pt} % The width of the vertical ruler between columns
% \setlength{\evensidemargin}{} % Margin of even pages, commonly used in two-sided documents such as books
% \setlength{\linewidth}{} % Width of the line in the current environment.
% \setlength{\oddsidemargin}{} % Margin of odd pages, commonly used in two-sided documents such as books
% \setlength{\paperwidth}{} % Width of the page
% \setlength{\paperheight}{} % Height of the page
\setlength{\parindent}{0cm} % Paragraph indentation
\setlength{\parskip}{6pt} % Vertical space between paragraphs
% \setlength{\tabcolsep}{} % Separation between columns in a table (tabular environment)
% \setlength{\textheight}{} % Height of the text area in the page
% \setlength{\textwidth}{} % Width of the text area in the page
% \setlength{\topmargin}{} % Length of the top margin
\setlist{
  %%%% Vertical spacing
  topsep = 0pt,
  partopsep = 0pt,
  parsep = 0pt,
  itemsep = 0pt,
  %%%% Horizontal spacing
  leftmargin = 0.5cm,
  rightmargin = 0cm,
  % listparindent = 0cm,
  % labelwidth = 0cm,
  % labelsep = 0cm,
  % itemindent = 0cm
}
\addtolength{\cellspacetoplimit}{2pt}
\addtolength{\cellspacebottomlimit}{2pt}

%%%%%%%%%%%%%%%%%%%%%%%%%%%%%%%%%%%%%%%%%%%%%%%%%%%%%%%%%%%%%%%%%%%%%%%%%%%%%%%
%%%% Page layout
\usepackage{layout} % View the layout of a document
\usepackage{geometry} % Flexible and complete interface to document dimensions
% 1cm = 10mm = 28pt = 1/2.54in
% ex = height of a lowercase 'x' in the current font
% em = width of an uppercase 'M' in the current font
\geometry{
  a4paper,
  top         = 1cm,
  bottom      = 1cm,
  left        = 1.5cm,
  right       = 1.5cm,
  includehead = true,
  includefoot = true,
  landscape   = false, % Paper orientation
  twoside     = false,
}
% \geometry{showframe} % Show paper outline for the text area and page

%%%%%%%%%%%%%%%%%%%%%%%%%%%%%%%%%%%%%%%%%%%%%%%%%%%%%%%%%%%%%%%%%%%%%%%%%%%%%%%
%%%% Header and footer style
\usepackage{fancyhdr} % Extensive control of page headers and footers in LaTeX
\pagestyle{fancy}
% Options: \leftmark (chapter title), \rightmark(section title), \thepage (page number), \thechapter(chapter number), \thesection (section number)
\lhead{\thetitle}
\chead{}
\rhead{}
\lfoot{}
\cfoot{\thepage}
\rfoot{}

%%%%%%%%%%%%%%%%%%%%%%%%%%%%%%%%%%%%%%%%%%%%%%%%%%%%%%%%%%%%%%%%%%%%%%%%%%%%%%%
%%%% URL insertion settings
\usepackage{hyperref} % Extensive support for hypertext in LaTeX
\definecolor{black}{RGB}{0, 0, 0} % rgb(0, 0, 0)
\definecolor{blue}{RGB}{0, 0, 255} % rgb(0, 0, 255)
\hypersetup{
  % unicode            = true,
  pdftitle           = {\thetitle},
  pdfauthor          = {\theauthor},
  % pdfsubject       = {},
  %%%% Reference
  % bookmarks          = true,
  bookmarksnumbered  = true,
  bookmarksopen      = true, % Open the bookmarks
  bookmarksopenlevel = 2, % Open until 1 level (section)
  %%%% Bookmarks
  breaklinks         = true,
  pdfborder          = {0 0 0},
  % backref            = true, % Add links into bibliography
  % pagebackref        = true,
  % hyperindex         = true, % Add links into index
  %%%% Color
  colorlinks         = true,
  linkcolor          = black, % Internal links color
  citecolor          = black,
  urlcolor           = blue, % Hyperlinks color
  filecolor          = black,
}

\usepackage{varioref} % Intelligent page reference
\usepackage[capitalise,noabbrev]{cleveref}
\usepackage{prettyref} % Make label references "self-identity" with \prettyref{#1}
\newrefformat{cha}{chapter \textbf{\nameref{#1}} \vpageref{#1}} % {chapter \textbf{\nameref{#1}} on page \pageref{#1}}
\newrefformat{sec}{section \textbf{\nameref{#1}} \vpageref{#1}} % {section \textbf{\nameref{#1}} on page \pageref{#1}}
% \newrefformat{fig}{\vref{#1}} % {Figure \ref{#1} on page \pageref{#1}}
% \newrefformat{tab}{\vref{#1}} % {Table \ref{#1} on page \pageref{#1}}
% \newrefformat{eqn}{\vref{#1}}
% \newrefformat{lis}{\emph{\nameref{#1}} \vpageref{#1}}

%%%%%%%%%%%%%%%%%%%%%%%%%%%%%%%%%%%%%%%%%%%%%%%%%%%%%%%%%%%%%%%%%%%%%%%%%%%%%%%
%%%% Physics units settings
% Dependencies
\usepackage{booktabs} % Publication quality tables in LaTeX
\usepackage{caption} % Customizing captions in floating environments
\usepackage{helvet} % Load Helvetica, scaled
\usepackage{cancel} % Place lines through maths formulae

\usepackage{siunitx} % A comprehensive (SI) units package
\sisetup{
  exponent-product     = \cdot, % Symbol between number and power of ten
  group-minimum-digits = 5, % Number of digits when 3 digits separation appear
  % inter-unit-product   = \cdot, % Symbol between units (when several units are used)
  output-complex-root  = \ensuremath{i}, % How i math should be seen
  % prefixes-as-symbols  = false, % Translate prefixes (kilo, centi, milli, micro,...) into a power of ten
  separate-uncertainty = true, % Write uncertainty with +-
  scientific-notation  = engineering,
}

%%%%%%%%%%%%%%%%%%%%%%%%%%%%%%%%%%%%%%%%%%%%%%%%%%%%%%%%%%%%%%%%%%%%%%%%%%%%%%%
%%%% Theorems and proofs
\numberwithin{equation}{section}
% \makeatletter
% \g@addto@macro\th@remark{\thm@headpunct{:}}
% \makeatother
\theoremstyle{remark}
\newtheorem*{example}{Example}
\newtheorem*{remark}{Remark}

%%%%%%%%%%%%%%%%%%%%%%%%%%%%%%%%%%%%%%%%%%%%%%%%%%%%%%%%%%%%%%%%%%%%%%%%%%%%%%%
%%%% User-defined environments
% Remove the space before the enumerate and itemize environments
\let\oldenumerate\enumerate % Keep a copy of \enumerate (or \begin{enumerate})
\let\endoldenumerate\endenumerate % Keep a copy of \endenumerate (or \end{enumerate})
\renewenvironment{enumerate}{
  \begin{oldenumerate}
    \vspace{-6pt}
    }{
  \end{oldenumerate}
}

\let\olditemize\itemize % Keep a copy of \itemize (or \begin{itemize})
\let\endolditemize\enditemize % Keep a copy of \enditemize (or \end{itemize})
\renewenvironment{itemize}{
  \begin{olditemize}
    \vspace{-6pt}
    }{
  \end{olditemize}
}

\let\olddescription\description % Keep a copy of \description (or \begin{description})
\let\endolddescription\enddescription % Keep a copy of \enddescription (or \end{description})
\renewenvironment{description}{
  \begin{olddescription}
    \vspace{-6pt}
    }{
  \end{olddescription}
}

%%%%%%%%%%%%%%%%%%%%%%%%%%%%%%%%%%%%%%%%%%%%%%%%%%%%%%%%%%%%%%%%%%%%%%%%%%%%%%%
%%%% User-defined commands
\newcommand{\Romannumeral}[1]{\MakeUppercase{\romannumeral #1}} % Capital roman numbers
% \newcommand{\gui}[1]{\og #1 \fg{}} % French quotation marks
\renewcommand{\thefootnote}{[\arabic{footnote}]}

%%% Figure command
%% Include SVG files
\newcommand{\executeiffilenewer}[3]{
  \ifnum\pdfstrcmp{\pdffilemoddate{#1}}
    {\pdffilemoddate{#2}}>0
    {\immediate\write18{#3}}\fi
}
\newcommand{\includesvg}[1]{
  \executeiffilenewer{#1.svg}{#1.pdf}
  {
    % Inkscape must be installed in PATH and the user must include '--shell-escape' in the build arguments
    inkscape #1.svg --export-type=pdf --export-latex
  }
  \input{#1.pdf_tex}
}

%%% Math commands
%% Tables (requires cellspace package)
\newcolumntype{L}{>{\(\displaystyle}Cl<{\)}} % Column type for left-aligned math column
\newcolumntype{D}{>{\(\displaystyle}Cc<{\)}} % Column type for centered math column

%% Functions
\newcommand{\constant}{\mathrm{constant}} % Constant
\newcommand{\abs}[1]{\left| #1 \right|} % Absolute function
\newcommand{\erf}[1]{\mathrm{erf} \left( #1 \right)} % Error function
\newcommand{\erfc}[1]{\mathrm{erfc} \left( #1 \right)} % Complementary error function
\newcommand{\unitstep}[1]{\,\mathcal{U}\left( #1 \right)} % Unit step function
\newcommand{\diracdelta}[2]{\,\delta_{#1}\left( #2 \right)} % Dirac delta function


%% Derivatives and integrals
\newcommand{\diff}[2]{\mathrm{d}^{#1} #2} % Letter 'd' of differentials
\newcommand{\diffint}[1]{\,\diff{}{#1}} % Differential with a space for integrals
\newcommand{\derivative}[2]{\frac{\diff{}{#1}}{\diff{}{#2}}} % Derivative
\newcommand{\nderivative}[3]{\frac{\diff{#1}{#2}}{\diff{}{#3^{#1}}}} % Derivative of degree n
\newcommand{\partialderivative}[2]{\frac{\partial #1}{\partial #2}} % Partial derivative
\newcommand{\npartialderivative}[3]{\frac{\partial^{#1} #2}{\partial #3^{#1}}} % Partial derivative of degree n
\newcommand{\direcderivative}[2]{D_{\vec{#1}}\,#2} % Directional derivative

\newcommand{\Laplace}[1]{\mathcal{L}\left\{ #1 \right\}} % Laplace transform notation
\newcommand{\invLaplace}[1]{\mathcal{L}^{-1}\left\{ #1 \right\}} % Inverse Laplace transform notation

%% Set
\newcommand{\set}[3]{\mathbb{#1}_{#2}^{#3}} % Set of numbers
\newcommand{\integerset}{\mathbb{Z}} % Set of integer numbers (compatibility)
\newcommand{\realset}{\mathbb{R}} % Set of real numbers (compatibility)

%% Limits
\newcommand{\limit}[3]{\lim_{#1 \to #2}{#3}} % Limit from a point to another
\newcommand{\rlimit}[3]{\lim_{#1 \to #2^{+}}{#3}} % Right limit from a point to another
\newcommand{\llimit}[3]{\lim_{#1 \to #2^{-}}{#3}} % Left imit from a point to another
\newcommand{\modulus}[1]{\,\left[ #1 \right]} % Modulus notation

%% Vectors
\newcommand{\ivec}{\hat{\mathrm{i}}} % i vector
\newcommand{\jvec}{\hat{\mathrm{j}}} % j vector
\newcommand{\kvec}{\hat{\mathrm{k}}} % k vector
\renewcommand{\Vec}[1]{\overrightarrow{#1}} % Vector notation for expression with more than one letter
\newcommand{\norm}[1]{\left\| #1 \right\|} % Norm notation for expression with just one letter
\newcommand{\normvec}[1]{\left\| \vec{#1} \right\|} % Norm notation for expression with just one letter
\newcommand{\Normvec}[1]{\left\| \Vec{#1} \right\|} % Norm notation for expression with more than one letter
\newcommand{\comp}[2]{\mathrm{comp}_{\vec{#2}}\vec{#1}} % Components
\newcommand{\proj}[2]{\mathrm{proj}_{\vec{#2}}\vec{#1}}
\newcommand{\grad}[1]{\vec{\nabla}#1} % Gradient notation
\newcommand{\frames}[2]{\left( #1 \right)_{#2}} % Frame definition

\newcommand{\curl}[1]{\mathrm{curl}\,\vec{#1}} % Curl of a vector field
\newcommand{\divergence}[1]{\mathrm{div}\,\vec{#1}} % Divergence of a vector field


%%%%%%%%%%%%%%%%%%%%%%%%%%%%%%%%%%%%%%%%%%%%%%%%%%%%%%%%%%%%%%%%%%%%%%%%%%%%%%%
%%%% New document specific commands
% \renewcommand{\contentsname}{Table of contents} % Change the table of contents title

%%%%%%%%%%%%%%%%%%%%%%%%%%%%%%%%%%%%%%%%%%%%%%%%%%%%%%%%%%%%%%%%%%%%%%%%%%%%%%%
%%%% Beginning of the document
\begin{document}
\maketitle % Insert the cover page
% \tableofcontents
% \layout % Show a drawing of page layout
% \renewcommand{\abstractname}{} % Change the abstract title

\section{Introduction}
Newton's laws:
\begin{description}
  \item[First law:] \(\sum{\vec{F}} = \vec{0}\)
  \item[Second law:] \(\sum{\vec{F}} = m\vec{a}\)
  \item[Third law:] \(\vec{F}_{A \to B} = - \vec{F}_{B \to A}\)
\end{description}

Newton's law of gravitation:
\[
  \norm{\vec{F}_{A \to B}} = \norm{\vec{F}_{B \to A}} = G \frac{m_A m_B}{r^2}
\]
\[
  \begin{array}{|l}
    F (\si{\newton}) \text{: attraction force}                         \\
    G = \SI{6.674 e-11}{\metre\cubed\per\kilo\gram\per\second\squared} \\
    m (\si{\kilo\gram}) \text{: mass of the object}                    \\
    r (\si{\metre}) \text{: distance between the objects}
  \end{array}
\]

\section{Statics of rigid bodies}
\subsection{Concepts for rigid bodies}
\subsubsection{Principle of transmissibility}
A force \(\vec{F}\) can be moved anywhere on its line of action without affecting the conditions of equilibrium or motion.

\subsubsection{Moments}
A moment is a measurement of how much a force acting ona rigid body causes its rotation.

The moment of a force about a point is
\[
  \vec{M}_B = \vec{M}_A + \Vec{BA} \times \vec{F}
\]
\subsection{Moment of a couple}
Two forces \(\vec{F}\) and \(- \vec{F}\) having the same magnitude and opposite direction are said to form a couple.
The sum of these forces is \(\vec{0}\), but the sum of the moments about a given point is different from \(\vec{0}\).

The value of a couple is \(\normvec{M} = d\normvec{F}\) where \(d\) is the smallest distance between the 2 forces.

\subsection{Moment of a force about an axis}
\subsubsection{Moment about a given axis}
The value of a moment of a force \(\vec{F}\) with a point of application \(A\) about an arbitrary axis \(BC\) is
\[
  \norm{\vec{M}_{BC}} = \frac{\Vec{BC}}{\Normvec{BC}} \bullet \left( \Vec{BA} \times \vec{F} \right)
\]

\subsection{Reduction of a system of forces}
A system of forces can be reduced to a single force \(\vec{R}\) and moment \(\vec{M}_O(\vec{R})\):
\begin{align*}
  \vec{R} = \sum{\vec{F}} &  & \vec{M}_O(\vec{R}) = \sum{\vec{M}_O(\vec{F})}
\end{align*}

To find the equivalent system, the forces can be moved from their position to a point \(O\) only if their moments about \(O\) are added.
A pure moment (couple) can be moved anywhere without affecting the system.

\subsection{Further reduction of a system of forces}
A system of forces can be reduced to a single force if the force and its moment are mutually perpendicular.
This condition is satisfied if:
\begin{itemize}
  \item Forces are coplanar (2D case)
  \item Forces are concurrent
  \item Forces are parallel to the same axis
\end{itemize}

\subsection{Reduction to a wrench}
A force-moment system can be reduced to a wrench, a force \(\vec{R}\) and a moment \(\vec{M}\) along \(\vec{R}\).
The ratio \(p = \frac{\normvec{M}}{\normvec{R}}\) is called the pitch of the wrench, and we have
\[
  \vec{M} = \mathrm{proj}_{\vec{R}}\vec{M}_O = \frac{\vec{R} \bullet \vec{M}_O(\vec{R})}{\normvec{R}^2}\vec{R}
\]

\subsection{Equilibrium of rigid bodies}
The equations of equilibrium for a rigid body are
\begin{align*}
  \sum{\vec{F}} = \vec{0} &  & \sum{\vec{M}_O(\vec{F})} = \vec{0}
\end{align*}

\subsection{Equilibrium in 3D}
\subsubsection{Reactions at supports and connections}
\emph{See Pascal Klinguer paper for a comprehensive table with the reactions at supports and connections.}

The basic thing is to put a 0 on the free components.

\subsubsection{Steps for solving}
\begin{enumerate}
  \item Isolate the body
  \item Define the basis
  \item Express all forces and moments in components
  \item Apply equilibrium equations \(\sum{\vec{F}} = \vec{0}\) and \(\sum{\vec{M}_O(\vec{F})} = \vec{0}\)
\end{enumerate}

\section{Centroids and centers of gravity}
\subsection{Planar centers of gravity and centroids}
In this section, we will consider that the body is homogeneous (uniform density \(\rho\)) and have uniform thickness \(t\).

In order to find the centroid, we need to find the area of the plate:
\[
  A = \iint{\diffint{A}} \\
\]

Next, we introduce the first moments of the shape as:
\begin{align*}
  Q_y = M_y = \iint{x \diffint{A}} &  & Q_x = M_x = \iint{y \diffint{A}}
\end{align*}

Finally, we have the centroid location \((\bar{x}, \bar{y})\) is at:
\begin{align*}
  \bar{x} = \frac{Q_y}{A} &  & \bar{y} = \frac{Q_x}{A}
\end{align*}

The negative signs are determined using a table of sign, where the negative sign appears in the negative axis and for holes.

\subsection{Wires}
For a wire, a similar equation can be derived:
\begin{align*}
  \bar{x} = \frac{Q_y}{L} = \frac{\int{x \diffint{L}}}{L} &  & \bar{y} = \frac{Q_x}{L} = \frac{\int{y \diffint{L}}}{L}
\end{align*}
where
\[
  L = \int{\norm{\vec{r}'} \diffint{t}} = \int{\sqrt{\left[ x'(t)^2 \right] + \left[ y'(t)^2 \right]} \diffint{t}}
\]

\subsection{Application of centroids}
\subsubsection{Distributed loads on beams}
One of the application of centroids is to find the concentrated load which is equivalent to the given distributed load.
It is used to to get reaction forces, but not internal forces and deflections.

\subsubsection{Forces on submerged surfaces}
The forces on submerged surface are distributed along the length of the body.
Using the the formula of pressure, we have
\[
  \vec{W} = t \rho h \vec{g}
\]
\[
  \begin{array}{|l}
    \vec{W} (\si{\newton}) \text{: weight of the body}                      \\
    t (\si{\metre}) \text{: thickness of the body}                          \\
    \rho (\si{\kilo\gram\per\metre\cubed}) \text{: density of the body}     \\
    h (\si{\metre}) \text{: vertical distance from the surface to the body} \\
    \vec{g} = (0, 0, -9.81)\si{\metre\per\second\squared} \text{ (on Earth)}
  \end{array}
\]


For forces exerted by a liquid on a curved body:
\begin{enumerate}
  \item Isolate the volume of liquid above the body
  \item Determine the forces exerted on it (including weight of the liquid)
  \item The force that the body exerts on the water has the same magnitude and line of action but an opposite direction.
\end{enumerate}

\subsection{Center of gravity and centroids of volumes}
\subsubsection{3D center of gravity and centroids}
The mass a solid is given by
\[
  m = \iiint{\rho(x, y, z) \diffint{V}}
\]

The first moments of the solid about the coordinate planes indicated by the subscripts are given by
\begin{align*}
  M_{xy} & = \iiint{z\rho(x, y, z) \diffint{V}} \\
  M_{xz} & = \iiint{y\rho(x, y, z) \diffint{V}} \\
  M_{yz} & = \iiint{x\rho(x, y, z) \diffint{V}}
\end{align*}

The coordinates of the center of mass \(G\) of the solid are given by
\begin{align*}
  x_G = \frac{M_{yz}}{m} &  & y_G = \frac{M_{xz}}{m} &  & z_G = \frac{M_{xy}}{m}
\end{align*}

\section{Analysis of structures}
\subsection{Equilibrium of a two-force body}
A rigid body in equilibrium subjected to two forces is called a "two-force body".
The two forces have the same magnitude and line of action, but an opposite direction.

Therefore, if a member is subjected to two forces, the internal forces are at the joints and must have the same line of action crossing the joints.

\subsection{Categories of structures}
\begin{description}
  \item[Trusses:] they are stationary and used to support loads, they are fully constrained and exclusively made of two-force members
  \item[Frames:] they are stationary and used to support loads, and they always have at least one multi-force member
  \item[Mechanisms:] they are designed to transmit and modify forces, they have moving parts and they always have at least one multi-force member
\end{description}

\subsection{Analysis of trusses}
Trusses are made of straight members connected at joints.
Bolded or welded connections are assumed to be pinned together.
It is usually assumed that members are pinned, therefore, the forces acting at each end are reduced to a force and no moment.
Only two-force members are considered.

\subsubsection{Zero-force members}
If there are two members at a joint with no external forces and the two members are not aligned, then these two members are zero-force members.

If there are three members at a joint with no external forces and two are aligned, then the third member is a zero-force members.

\subsubsection{Analysis of trusses by method of joints}
The line of actions of all internal forces are known, the analysis is reduced to finding their magnitude and determining their stress: either "compression" or "tension".

\begin{enumerate}
  \item Isolate the entire truss as a rigid body
  \item Find reaction forces using equilibrium equations
  \item Isolate each joint
  \item Find internal reaction forces using equilibrium at each joint
\end{enumerate}

\subsubsection{Analysis of trusses by method of sections}
If the force in on member or very few members are needed, the method of sections is more efficient.

\begin{enumerate}
  \item Isolate the entire truss as a rigid body
  \item Find reaction forces using equilibrium
  \item Cut the truss through the members of interest
  \item Represent the forces of the cut members
  \item Equilibrium of one fo the cut truss (the one with less external forces)
\end{enumerate}

\subsection{Analysis of frames and mechanisms}
\subsubsection{Frames}
Frames are stationary and used to support loads, and they always have at least one multi-force member.

\begin{enumerate}
  \item Isolate the entire frame as a rigid body
  \item Find reaction forces using equilibrium
  \item Isolate each member
  \item Find internal forces using equilibrium
\end{enumerate}

\subsubsection{Mechanisms}
Mechanisms are designed to transmit and/or modify forces, they have moving parts and they always have at least one multi-force member.

\begin{enumerate}
  \item Isolate the entire mechanism as a rigid body
  \item Find reaction forces using equilibrium
  \item Isolate each member
  \item Find internal forces using equilibrium
\end{enumerate}

\subsection{Internal forces in members}
For a straight two-force member, the internal force is equivalent to axial forces.

For a not straight member or a multi-force member, the internal forces are not limited to producing just compression or tension, they also produce shear force \(\vec{V}\) and bending moment \(\vec{M}\).

\subsection{Beams}
\subsubsection{Types of loading and support}

\subsubsection{Shear and bending moment in a beam}
Steps to analyze a beam:
\begin{enumerate}
  \item Isolate the entire beam as a rigid body
  \item Find reaction forces using equilibrium
  \item Cut the beam at the desired location
  \item Find the internal shearing force \(\vec{V}\) and the bending moment \(\vec{M}\) using the sign convention
  \item Plot the values of the shearing force \(\vec{V}\) and the bending moment \(\vec{M}\) with respect to a distance \(x\) to get the shear and bending diagrams
\end{enumerate}

\paragraph*{Sign convention}
At first, shearing force \(\vec{V}\) is assumed to be directed down on the left side of the cut, and up on the right side of the cut.
The bending moment \(\vec{M}\) is assumed to be directed counterclockwise on the left side of the cut, and clockwise on the right side of the cut.

\subsubsection{Relation between loads, shear and bending moment}
When there are continuous loads \(w\), the following relation can be used:
\begin{align*}
  \derivative{V}{x} = - w &  & \derivative{M}{x} = V
\end{align*}

\section{Friction}
\subsection{Friction equations}
Static equation:
\[
  \vec{F}_S = \mu_S \vec{N}
\]
\[
  \begin{array}{|l}
    \vec{F}_S (\si{\newton}) \text{: static friction force} \\
    \mu_S \text{: coefficient of static friction}           \\
    \vec{N} (\si{\newton}) \text{: normal force}
  \end{array}
\]

Kinetic equation:
\[
  \vec{F}_K = \mu_K \vec{N}
\]
\[
  \begin{array}{|l}
    \vec{F}_K (\si{\newton}) \text{: kinetic friction force} \\
    \mu_K \text{: coefficient of kinetic friction}           \\
    \vec{N} (\si{\newton}) \text{: normal force}
  \end{array}
\]

\subsection{Friction on a horizontal surface}
\begin{description}
  \item[Case \Romannumeral{1}:] No friction \(\implies\) no horizontal force \(\implies\) no opposing friction force
  \item[Case \Romannumeral{2}:] No motion \(\implies\) external force \(\vec{P}\) not high enough to overcome static friction force \(\vec{F}_S\)
  \item[Case \Romannumeral{3}:] Motion impending (imminent) \(\implies\) applied forces such that the body is just about to slide \(\implies\) \(\vec{F} = \vec{F}_S\)
  \item[Case \Romannumeral{4}:] Motion \(\implies\) \(\vec{F} = \vec{F}_K\)
\end{description}

\subsection{Angles of friction}
\begin{description}
  \item[Case \Romannumeral{1}:] Motion impending \(\implies\) \(\mu_S = \tan\phi_S = \frac{\norm{\vec{F}_S}}{\normvec{N}}\)
  \item[Case \Romannumeral{2}:] Motion \(\implies\) \(\mu_K = \tan\phi_K = \frac{\norm{\vec{F}_K}}{\normvec{N}}\)
\end{description}

\subsection{Problems involving friction}
\begin{itemize}
  \item All applied forces are given and coefficients of frictions are known \(\implies\) determine whether the body will remain at rest or slide
  \item Applied forces are known and the body is in impending motion \(\implies\) determine friction force, normal force and coefficients of friction
  \item Coefficients of friction are known and the body is in impending motion \(\implies\) determine one of the external forces that will cause impending motion
\end{itemize}

\subsection{\(P_\mathrm{min}\) and \(P_\mathrm{max}\) to equilibrium}
For a block with a weight \(\vec{W}\) on an inclined surface, pushed by a force \(\vec{P}\), we have:
\[
  P_\mathrm{max, min} = \normvec{W} \sin\theta \pm \mu_S \normvec{N} \\
\]
where \(P_\mathrm{min}\) and \(P_\mathrm{max}\) are the boundaries in which the block will not move.

\section{Moment of inertia}
\subsection{Area moment of inertia}
\subsubsection{Calculation of the area moment of inertia}
In the 2D case, we have:
\begin{align*}
  I_x = \iint{y^2 \diffint{A}} &  & I_y = \iint{x^2 \diffint{A}}
\end{align*}

\subsubsection{Poplar moment of inertia}
When dealing with torsion, the polar moment of inertia is generally more important:
\begin{align*}
  J_O & = \iint{r^2 \diffint{A}} = \iint{x^2 + y^2 \diffint{A}} \\
      & = I_x + I_y
\end{align*}

\subsubsection{Radius of gyration of an area}
To get the same \(I\) between a blob and a strip, the strip should be placed at the radius of gyration \(K\):
\begin{align*}
  K_x = \sqrt{\frac{I_x}{A}} &  & K_y = \sqrt{\frac{I_y}{A}} &  & K_O = \sqrt{\frac{J_O}{A}}
\end{align*}

\subsection{Parallel-axis theorem and composite areas}
Often, complex shapes can be broken into a sum of simple shapes.
Since we need the moment of inertia of each simple shapes with respect to the same axis, we need to use the parallel-axis theorem.

The moment of inertia of the area \(A\) about a non-centroidal axis can be obtained using
\begin{align*}
  I_x = I_{x_O} + A {d_x}^2 &  & I_y = I_{y_O} + A {d_y}^2
\end{align*}
where \(d_x\) is the distance between the two \(x\)-axis and \(d_y\) is the distance between the two \(y\)-axis.

Hence, the moment of inertia's of a composite chape made of several areas is
\begin{align*}
  I_x & = \sum{I_{x_i}} + \sum{A {d_x}^2} \\
  I_y & = \sum{I_{y_i}} + \sum{A {d_y}^2}
\end{align*}

The change in the radius of gyration is \(K^2 = {K_O}^2 + d^2\)

\subsection{Mass moment of inertia}
\subsubsection{Moment of inertia of a simple mass}
The mass moment of inertia is defined as:
\[
  I = \int{r^2 \diffint{m}}
\]
\[
  \begin{array}{|l}
    I (\si{\kilo\gram\metre\squared}) \text{: mass moment of inertia about an axis} \\
    r (\si{\metre}) \text{: distance for the center of mass to the axis}            \\
  \end{array}
\]

We can express \(I\) in terms of the standard \(xyz\)-basis:
\begin{align*}
  I_x & = \int{y^2 + z^2 \diffint{m}} \\
  I_y & = \int{x^2 + z^2 \diffint{m}} \\
  I_z & = \int{x^2 + y^2 \diffint{m}}
\end{align*}

\subsubsection{Parallel-axis theorem for mass moments of inertia}
\begin{align*}
  I_{x_O} & = I_{x_G} + m({y_O}^2 + {z_O}^2) \\
  I_{y_O} & = I_{y_G} + m({x_O}^2 + {z_O}^2) \\
  I_{z_O} & = I_{z_G} + m({x_O}^2 + {y_O}^2) \\
\end{align*}
where \(G\) is the center of mass and \(O\) is the destination point.


\subsubsection{Mass moment of inertia of a 3D body by integration}
The general formula for the mass moment of inertia of a 3D body is:
\begin{align*}
  I_x & = \iiint{(y^2 + z^2 )\rho(x, y, z) \diffint{V}} \\
  I_y & = \iiint{(x^2 + z^2 )\rho(x, y, z) \diffint{V}} \\
  I_z & = \iiint{(x^2 + y^2 )\rho(x, y, z) \diffint{V}}
\end{align*}

\subsection{Transformation of moment of inertia}
If we want to find the axis that generate the maximum and minimum moment of inertia, we first need to define the product of inertia:
\begin{align*}
  I_{xy} = \iint{xy \diffint{A}} &  & I_{xy} = I_{x_G y_G} + x_G y_G A
\end{align*}

Now, we can obtain Mohr's circles, which let use get the minimum and maximum moments of inertia for a body:
\[
  I_\mathrm{max, min} = I_\mathrm{average} \pm R = \frac{I_x + I_y}{2} \pm \sqrt{\left( \frac{I_x - I_y}{2} \right)^2 + {I_{xy}}^2}
\]
\end{document}