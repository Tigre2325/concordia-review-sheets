\documentclass[10pt, twocolumn]{article}

%%%%%%%%%%%%%%%%%%%%%%%%%%%%%%%%%%%%%%%%%%%%%%%%%%%%%%%%%%%%%%%%%%%%%%%%%%%%%%%
%%%% Cover page
\title{ENGR 311: Transform Calculus and Partial Differential Equations}
\date{\today}
\author{Anthony Bourboujas}

\makeatletter
\let\Title\@title
\let\Author\@author
\let\Date\@date
\makeatother

%%%%%%%%%%%%%%%%%%%%%%%%%%%%%%%%%%%%%%%%%%%%%%%%%%%%%%%%%%%%%%%%%%%%%%%%%%%%%%%
%%%% Preamble
%%%%%%%%%%%%%%%%%%%%%%%%%%%%%%%%%%%%%%%%%%%%%%%%%%%%%%%%%%%%%%%%%%%%%%%%%%%%%%%
%%%% Packages
\usepackage[utf8x]{inputenc} % Accept different input encodings
\usepackage[T1]{fontenc} % Standard package for selecting font encodings
\usepackage{lmodern} % Font name; classic: lmodern
\usepackage[english]{babel} % Multilingual support for LaTeX
% \usepackage{abstract} % Control the typesetting of the abstract environment
\usepackage{amsmath} % AMS mathematical facilities for LaTeX
\usepackage{amssymb} % TeX fonts from the American Mathematical Society
\usepackage{amsthm} % Typesetting theorems (AMS style)
\usepackage{array} % Extending the array and tabular environments
\usepackage{cellspace} % Ensure minimal spacing for table cells
\usepackage{chemformula} % Command for typesetting chemical formulas and reactions
% \usepackage{colortbl} % Add colour to LaTeX tables
% \usepackage{comment} % Selectively include/exclude portions of text
% \usepackage[en-US,showdow]{datetime2} % Formats for dates, times and time zones
\usepackage{enumitem} % Control layout of itemize, enumerate, description
\usepackage{esint} % Extended set of integrals for Computer Modern
\usepackage{graphicx} % Enhanced support for graphics
% \usepackage{lipsum} % Easy access to the Lorem Ipsum dummy text
\usepackage{mathrsfs} % Support for using RSFS fonts in maths
\usepackage{moreverb} % Extended verbatim
\usepackage{multicol} % Intermix single and multiple columns
\usepackage{multirow} % Create tabular cells spanning multiple rows
\usepackage[scr]{rsfso} % A mathematical calligraphic font based on rsfs
\usepackage{setspace} % Set space between lines
% \usepackage{soul} % Hyphenation for letterspacing, underlining, and more
\usepackage{threeparttable} % Tables with captions and notes all the same width
\usepackage{wrapfig} % Produces figures which text can flow around
\usepackage{xcolor} % Driver-independent color extensions for LaTeX

%%%%%%%%%%%%%%%%%%%%%%%%%%%%%%%%%%%%%%%%%%%%%%%%%%%%%%%%%%%%%%%%%%%%%%%%%%%%%%%
%%%% Lengths
% 1cm = 10mm = 28pt = 1/2.54in
% 1ex = height of a lowercase 'x' in the current font
% 1em = width of an uppercase 'M' in the current font

%%%% Spacing in math mode
% \!                         = -3/18em
% \,                         = 3/18em
% \:                         = 4/18em
% \;                         = 5/18em
% \ (space after backslash!) = space in normal text
% \quad                      = 1em
% \qquad                     = 2em

% \setlength{\baselineskip}{1em} % Vertical distance between lines in a paragraph
% \renewcommand{\baselinestretch}{1.0} % A factor multiplying \baslineskip
\setlength{\columnsep}{0.75cm} % Distance between columns
% \setlength{\columnwidth}{} % The width of a column
\setlength{\columnseprule}{1pt} % The width of the vertical ruler between columns
% \setlength{\evensidemargin}{} % Margin of even pages, commonly used in two-sided documents such as books
% \setlength{\linewidth}{} % Width of the line in the current environment.
% \setlength{\oddsidemargin}{} % Margin of odd pages, commonly used in two-sided documents such as books
% \setlength{\paperwidth}{} % Width of the page
% \setlength{\paperheight}{} % Height of the page
\setlength{\parindent}{0cm} % Paragraph indentation
\setlength{\parskip}{6pt} % Vertical space between paragraphs
% \setlength{\tabcolsep}{} % Separation between columns in a table (tabular environment)
% \setlength{\textheight}{} % Height of the text area in the page
% \setlength{\textwidth}{} % Width of the text area in the page
% \setlength{\topmargin}{} % Length of the top margin
\setlist{
  %%%% Vertical spacing
  topsep = 0pt,
  partopsep = 0pt,
  parsep = 0pt,
  itemsep = 0pt,
  %%%% Horizontal spacing
  leftmargin = 0.5cm,
  rightmargin = 0cm,
  % listparindent = 0cm,
  % labelwidth = 0cm,
  % labelsep = 0cm,
  % itemindent = 0cm
}
\addtolength{\cellspacetoplimit}{2pt}
\addtolength{\cellspacebottomlimit}{2pt}

%%%%%%%%%%%%%%%%%%%%%%%%%%%%%%%%%%%%%%%%%%%%%%%%%%%%%%%%%%%%%%%%%%%%%%%%%%%%%%%
%%%% Page layout
\usepackage{layout} % View the layout of a document
\usepackage{geometry} % Flexible and complete interface to document dimensions
% 1cm = 10mm = 28pt = 1/2.54in
% ex = height of a lowercase 'x' in the current font
% em = width of an uppercase 'M' in the current font
\geometry{
  a4paper,
  top         = 1cm,
  bottom      = 1cm,
  left        = 1.5cm,
  right       = 1.5cm,
  includehead = true,
  includefoot = true,
  landscape   = false, % Paper orientation
  twoside     = false,
}
% \geometry{showframe} % Show paper outline for the text area and page

%%%%%%%%%%%%%%%%%%%%%%%%%%%%%%%%%%%%%%%%%%%%%%%%%%%%%%%%%%%%%%%%%%%%%%%%%%%%%%%
%%%% Header and footer style
\usepackage{fancyhdr} % Extensive control of page headers and footers in LaTeX
\pagestyle{fancy}
% Options: \leftmark (chapter title), \rightmark(section title), \thepage (page number), \thechapter(chapter number), \thesection (section number)
\lhead{\Title}
\chead{}
\rhead{}
\lfoot{}
\cfoot{\thepage}
\rfoot{}

%%%%%%%%%%%%%%%%%%%%%%%%%%%%%%%%%%%%%%%%%%%%%%%%%%%%%%%%%%%%%%%%%%%%%%%%%%%%%%%
%%%% URL insertion settings
\usepackage{hyperref} % Extensive support for hypertext in LaTeX
\definecolor{black}{RGB}{0, 0, 0} % rgb(0, 0, 0)
\definecolor{blue}{RGB}{0, 0, 255} % rgb(0, 0, 255)
\hypersetup{
  % unicode            = true,
  pdftitle           = {\Title},
  pdfauthor          = {\Author},
  % pdfsubject       = {},
  %%%% Reference
  % bookmarks          = true,
  bookmarksnumbered  = true,
  bookmarksopen      = true, % Open the bookmarks
  bookmarksopenlevel = 2, % Open until 1 level (section)
  %%%% Bookmarks
  breaklinks         = true,
  pdfborder          = {0 0 0},
  % backref            = true, % Add links into bibliography
  % pagebackref        = true,
  % hyperindex         = true, % Add links into index
  %%%% Color
  colorlinks         = true,
  linkcolor          = black, % Internal links color
  citecolor          = black,
  urlcolor           = blue, % Hyperlinks color
  filecolor          = black,
}

\usepackage{varioref} % Intelligent page reference
\usepackage[capitalise,noabbrev]{cleveref}
\usepackage{prettyref} % Make label references "self-identity" with \prettyref{#1}
\newrefformat{cha}{chapter \textbf{\nameref{#1}} \vpageref{#1}} % {chapter \textbf{\nameref{#1}} on page \pageref{#1}}
\newrefformat{sec}{section \textbf{\nameref{#1}} \vpageref{#1}} % {section \textbf{\nameref{#1}} on page \pageref{#1}}
% \newrefformat{fig}{\vref{#1}} % {Figure \ref{#1} on page \pageref{#1}}
% \newrefformat{tab}{\vref{#1}} % {Table \ref{#1} on page \pageref{#1}}
% \newrefformat{eqn}{\vref{#1}}
% \newrefformat{lis}{\emph{\nameref{#1}} \vpageref{#1}}

%%%%%%%%%%%%%%%%%%%%%%%%%%%%%%%%%%%%%%%%%%%%%%%%%%%%%%%%%%%%%%%%%%%%%%%%%%%%%%%
%%%% Physics units settings
% Dependencies
\usepackage{booktabs} % Publication quality tables in LaTeX
\usepackage{caption} % Customizing captions in floating environments
\usepackage{helvet} % Load Helvetica, scaled
\usepackage{cancel} % Place lines through maths formulae

\usepackage{siunitx} % A comprehensive (SI) units package
\sisetup{
  exponent-product     = \cdot, % Symbol between number and power of ten
  group-minimum-digits = 5, % Number of digits when 3 digits separation appear
  % inter-unit-product   = \cdot, % Symbol between units (when several units are used)
  output-complex-root  = \ensuremath{i}, % How i math should be seen
  % prefixes-as-symbols  = false, % Translate prefixes (kilo, centi, milli, micro,...) into a power of ten
  separate-uncertainty = true, % Write uncertainty with +-
  scientific-notation  = engineering,
}

%%%%%%%%%%%%%%%%%%%%%%%%%%%%%%%%%%%%%%%%%%%%%%%%%%%%%%%%%%%%%%%%%%%%%%%%%%%%%%%
%%%% Theorems and proofs
\numberwithin{equation}{section}
% \makeatletter
% \g@addto@macro\th@remark{\thm@headpunct{:}}
% \makeatother
\theoremstyle{remark}
\newtheorem*{example}{Example}
\newtheorem*{remark}{Remark}

%%%%%%%%%%%%%%%%%%%%%%%%%%%%%%%%%%%%%%%%%%%%%%%%%%%%%%%%%%%%%%%%%%%%%%%%%%%%%%%
%%%% User-defined environments
% Remove the space before the enumerate and itemize environments
\let\oldenumerate\enumerate % Keep a copy of \enumerate (or \begin{enumerate})
\let\endoldenumerate\endenumerate % Keep a copy of \endenumerate (or \end{enumerate})
\renewenvironment{enumerate}{
  \begin{oldenumerate}
    \vspace{-6pt}
    }{
  \end{oldenumerate}
}

\let\olditemize\itemize % Keep a copy of \itemize (or \begin{itemize})
\let\endolditemize\enditemize % Keep a copy of \enditemize (or \end{itemize})
\renewenvironment{itemize}{
  \begin{olditemize}
    \vspace{-6pt}
    }{
  \end{olditemize}
}

\let\olddescription\description % Keep a copy of \description (or \begin{description})
\let\endolddescription\enddescription % Keep a copy of \enddescription (or \end{description})
\renewenvironment{description}{
  \begin{olddescription}
    \vspace{-6pt}
    }{
  \end{olddescription}
}

%%%%%%%%%%%%%%%%%%%%%%%%%%%%%%%%%%%%%%%%%%%%%%%%%%%%%%%%%%%%%%%%%%%%%%%%%%%%%%%
%%%% User-defined commands
\newcommand{\Romannumeral}[1]{\MakeUppercase{\romannumeral #1}} % Capital roman numbers
% \newcommand{\gui}[1]{\og #1 \fg{}} % French quotation marks
\renewcommand{\thefootnote}{[\arabic{footnote}]}

%%% Figure command
%% Include SVG files
\newcommand{\executeiffilenewer}[3]{
  \ifnum\pdfstrcmp{\pdffilemoddate{#1}}
    {\pdffilemoddate{#2}}>0
    {\immediate\write18{#3}}\fi
}
\newcommand{\includesvg}[1]{
  \executeiffilenewer{#1.svg}{#1.pdf}
  {
    % Inkscape must be installed in PATH and the user must include '--shell-escape' in the build arguments
    inkscape #1.svg --export-type=pdf --export-latex
  }
  \input{#1.pdf_tex}
}

%%% Math commands
%% Tables (requires cellspace package)
\newcolumntype{L}{>{\(\displaystyle}Cl<{\)}} % Column type for left-aligned math column
\newcolumntype{D}{>{\(\displaystyle}Cc<{\)}} % Column type for centered math column

%% Functions
\newcommand{\constant}{\mathrm{constant}} % Constant
\newcommand{\abs}[1]{\left| #1 \right|} % Absolute function
\newcommand{\erf}[1]{\mathrm{erf} \left( #1 \right)} % Error function
\newcommand{\erfc}[1]{\mathrm{erfc} \left( #1 \right)} % Complementary error function
\newcommand{\unitstep}[1]{\,\mathcal{U}\left( #1 \right)} % Unit step function
\newcommand{\diracdelta}[2]{\,\delta_{#1}\left( #2 \right)} % Dirac delta function


%% Derivatives and integrals
\newcommand{\diff}[2]{\mathrm{d}^{#1} #2} % Letter 'd' of differentials
\newcommand{\diffint}[1]{\,\diff{}{#1}} % Differential with a space for integrals
\newcommand{\derivative}[2]{\frac{\diff{}{#1}}{\diff{}{#2}}} % Derivative
\newcommand{\nderivative}[3]{\frac{\diff{#1}{#2}}{\diff{}{#3^{#1}}}} % Derivative of degree n
\newcommand{\partialderivative}[2]{\frac{\partial #1}{\partial #2}} % Partial derivative
\newcommand{\npartialderivative}[3]{\frac{\partial^{#1} #2}{\partial #3^{#1}}} % Partial derivative of degree n
\newcommand{\direcderivative}[2]{D_{\vec{#1}}\,#2} % Directional derivative

\newcommand{\Laplace}[1]{\mathcal{L}\left\{ #1 \right\}} % Laplace transform notation
\newcommand{\invLaplace}[1]{\mathcal{L}^{-1}\left\{ #1 \right\}} % Inverse Laplace transform notation

%% Set
\newcommand{\set}[3]{\mathbb{#1}_{#2}^{#3}} % Set of numbers
\newcommand{\integerset}{\mathbb{Z}} % Set of integer numbers (compatibility)
\newcommand{\realset}{\mathbb{R}} % Set of real numbers (compatibility)

%% Limits
\newcommand{\limit}[3]{\lim_{#1 \to #2}{#3}} % Limit from a point to another
\newcommand{\rlimit}[3]{\lim_{#1 \to #2^{+}}{#3}} % Right limit from a point to another
\newcommand{\llimit}[3]{\lim_{#1 \to #2^{-}}{#3}} % Left imit from a point to another
\newcommand{\modulus}[1]{\,\left[ #1 \right]} % Modulus notation

%% Vectors
\newcommand{\ivec}{\hat{\mathrm{i}}} % i vector
\newcommand{\jvec}{\hat{\mathrm{j}}} % j vector
\newcommand{\kvec}{\hat{\mathrm{k}}} % k vector
\renewcommand{\Vec}[1]{\overrightarrow{#1}} % Vector notation for expression with more than one letter
\newcommand{\norm}[1]{\left\| #1 \right\|} % Norm notation for expression with just one letter
\newcommand{\normvec}[1]{\left\| \vec{#1} \right\|} % Norm notation for expression with just one letter
\newcommand{\Normvec}[1]{\left\| \Vec{#1} \right\|} % Norm notation for expression with more than one letter
\newcommand{\comp}[2]{\mathrm{comp}_{\vec{#2}}\vec{#1}} % Components
\newcommand{\proj}[2]{\mathrm{proj}_{\vec{#2}}\vec{#1}}
\newcommand{\grad}[1]{\vec{\nabla}#1} % Gradient notation
\newcommand{\frames}[2]{\left( #1 \right)_{#2}} % Frame definition

\newcommand{\curl}[1]{\mathrm{curl}\,\vec{#1}} % Curl of a vector field
\newcommand{\divergence}[1]{\mathrm{div}\,\vec{#1}} % Divergence of a vector field


%%%%%%%%%%%%%%%%%%%%%%%%%%%%%%%%%%%%%%%%%%%%%%%%%%%%%%%%%%%%%%%%%%%%%%%%%%%%%%%
%%%% Beginning of the document
\begin{document}
\maketitle % Insert the cover page
% \tableofcontents
% \layout % Show a drawing of page layout
% \renewcommand{\abstractname}{} % Change the abstract title
\section{Laplace transform}
\subsection{Definition of the Laplace transform}
If \(f(t)\) is defined for \(t \geqslant 0\), then the improper integral \(\int_0^\infty{K(s, t) f(t) \diffint{t}}\), is defined as a limit:
\[
  \int_0^\infty{K(s, t) f(t) \diffint{t}} = \limit{b}{\infty}{\int_0^b{K(s, t) f(t) \diffint{t}}}
\]

If the limit exists, the integral is said to exist or to be convergent; if the limit does not exist the integral does not exist and is said to be divergent.
The limit will generally exist for only certain values of the variable \(s\).
The choice \(K(s, t) = e^{-st}\) gives us an especially important integral transform called Laplace transform and written
\[
  \Laplace{f(t)} = F(s) = \int_0^\infty{e^{-st} f(t) \diffint{t}}
\]
where \(t\) is in the time domain defined for \(\set{R}{+}{}\) and \(s\) in the frequency domain.
The \(s\)-plane is not a real plane, but a complex plane.

The formula for integration by parts used in this class is:
\begin{multline*}
  \int{f_1(t) f_2(t) \diffint{t}} = f_1(t) \int{f_2(t) \diffint{t}} - \\
  \int{ \left( \derivative{}{t} [f_1(t)] \int{f_2(t) \diffint{t}} \right) \diffint{t}}
\end{multline*}


\subsubsection{Laplace transforms of some basic functions}
\begin{align*}
  \Laplace{1}         & = \frac{1}{s}                               \\
  \Laplace{t^n}       & = \frac{n!}{s^{n + 1}}, n \in \set{N}{+}{*} \\
  \Laplace{e^{at}}    & = \frac{1}{s - a}, a \in \set{R}{}{}        \\
  \Laplace{\sin{kt}}  & = \frac{k}{s^2 + k^2}                       \\
  \Laplace{\cos{kt}}  & = \frac{s}{s^2 + k^2}                       \\
  \Laplace{\sinh{kt}} & = \frac{k}{s^2 - k^2}                       \\
  \Laplace{\cosh{kt}} & = \frac{s}{s^2 - k^2}                       \\
\end{align*}


\subsubsection{Sufficient conditions for existence}
A function \(f\) is said to be of exponential order if there exist constants \(c > 0\), \(M > 0\), and \(T > 0\) such that \(\abs{f(t)} \leqslant Me^{ct}\) for all \(t > T\) (meaning for \(t\) big enough, the graph of \(f(t)\) does not grow faster than the graph of \(Me^{ct}\)).
If \(f(t)\) is piecewise continuous on the interval \(\set{R}{+}{} \) and of exponential order, then \(\Laplace{f(t)}\) exists for \(s > c\).


\subsubsection{Linearity of Laplace transform}
Linear transform:
\[
  \Laplace{\alpha f(t) + \beta g(t)} = \alpha \Laplace{f(t)} + \beta \Laplace{g(t)}
\]


\subsection{The inverse transform and transforms of derivatives}
If \(F(s)\) represents the Laplace transform of a function \(f(t)\), then \(f(t)\) is the inverse Laplace transform of \(F(s)\):
\[
  f(t) = \invLaplace{F(s)}
\]

Inverse linear transform:
\[
  \invLaplace{\alpha F(s) + \beta G(s)} = \alpha \invLaplace{F(s)} + \beta \invLaplace{G(s)}
\]


\subsubsection{Inverse Laplace transforms of some basic functions}
\begin{align*}
  \invLaplace{\frac{1}{s}}          & = 1                         \\
  \invLaplace{\frac{n!}{s^{n + 1}}} & = t^n, n \in \set{N}{+}{*}  \\
  \invLaplace{\frac{1}{s - a}}      & = e^{at}, a \in \set{R}{}{} \\
  \invLaplace{\frac{k}{s^2 + k^2}}  & = \sin{kt}                  \\
  \invLaplace{\frac{s}{s^2 + k^2}}  & = \cos{kt}                  \\
  \invLaplace{\frac{k}{s^2 - k^2}}  & = \sinh{kt}                 \\
  \invLaplace{\frac{s}{s^2 - k^2}}  & = \cosh{kt}                 \\
\end{align*}


\subsubsection{Transforms of derivatives}
If \(f\), \(f'\),\dots{},\(f^{(n - 1)}\) are continuous on \(\set{R}{+}{}\) and are of exponential order and if \(f^{(n)}(t)\) is piecewise continuous on \(\set{R}{+}{}\), then
\begin{multline*}
  \Laplace{f^{(n)}(t)} = s^n F(s) - s^{n - 1} f(0) - \\
  s^{n - 2} f'(0) - \cdots - f^{(n - 1)}(0)
\end{multline*}
where \(F(s) = \Laplace{f(t)}\).

\begin{remark}
  The Laplace transform of a linear differential equation with constant coefficients becomes an algebraic equation in \(Y(s)\), which will be used to solve differential equations.
\end{remark}


\subsubsection{The unit step function}
The unit step function \(\unitstep{t - a}\) is defined to be
\[
  \unitstep{t - a} =
  \begin{cases}
    0, & 0 \leqslant t < a \\
    1, & t \geqslant a     \\
  \end{cases}
\]

The unit step function is used to write a piecewise-defined function in a compact form:
\begin{align*}
  f(t)      & =
  \begin{cases}
    g(t), & 0 \leqslant t < a \\
    h(t), & a \leqslant t < b \\
    i(t), & t \geqslant b
  \end{cases}                                                 \\
  \iff f(t) & = g(t)\left[ \unitstep{t - 0} - \unitstep{t - a} \right]      \\
            & + h(t)\left[ \unitstep{t - a} - \unitstep{t - b} \right]      \\
            & + i(t)\left[ \unitstep{t - b} - \unitstep{t - \infty} \right] \\
            & + i(t)\left[ \unitstep{t - b}\right]                          \\
  \iff f(t) & = g(t) + \unitstep{t - a}\left[ h(t) - g(t) \right]           \\
            & + \unitstep{t - b}\left[ i(t) - h(t) \right]
\end{align*}


\subsection{Translation theorems}
\subsubsection{Frequency shift theorem}
If \(\Laplace{f(t)} = F(s)\) and \(a \in \set{R}{}{}\), then
\[
  \Laplace{e^{at}f(t)} = F(s - a)
\]


\subsubsection{Time shift theorem}
If \(\Laplace{f(t)} = F(s)\) and \(a > 0\), then
\[
  \Laplace{f(t - a) \unitstep{t - a}} = e^{-as}F(s)
\]
This leads to the Laplace transform of the unit step function being:
\[
  \Laplace{\unitstep{t - a}} = \frac{e^{-as}}{s}
\]

An alternative form of the time shift theorem is
\[
  \Laplace{g(t) \unitstep{t - a}} = e^{-as}\Laplace{g(t + a)}
\]


\subsection{Additional operational properties}
\subsubsection{Derivatives of transforms}
If the function \(f\) is piecewise continuous on \(\set{R}{+}{}\), of exponential order, and \(\Laplace{f(t)} = F(s)\), then for \(n \in \set{N}{}{*}\) and \(s > c\)
\[
  \Laplace{t^n f(t)} = (-1)^n \nderivative{n}{F(s)}{s}
\]


\subsubsection{Convolution}
If functions \(f\) and \(g\) are piecewise continuous on \(\set{R}{+}{}\), then the convolution of \(f\) and \(g\), denoted by the symbol \(f * g\), is a function defined by the integral
\[
  f * g = \int_0^t{f(\tau) g(t - \tau) \diffint{\tau}}
\]

\paragraph{Convolution theorem}
If the functions \(f\) and \(g\) are piecewise continuous on \(\set{R}{+}{}\), of exponential order and \(\Laplace{f(t)} = F(s)\) and \(\Laplace{g(t)} = G(s)\), then for \(s > c\):
\[
  \Laplace{f * g} = \Laplace{f(t)} \Laplace{g(t)} = F(s) G(s)
\]


\subsubsection{Transform of an integral}
Using the convolution theorem with \(g(t) = 1\), the Laplace transform of the integral of \(f\) is
\[
  \Laplace{\int_0^\tau{f(\tau) \diffint{\tau}}} = \frac{F(s)}{s}
\]


\subsubsection{Periodic function}
If a periodic function \(f\) has a period \(T \in \set{R}{+}{*}\), then \(f(t + T) = f(t)\).

If \(f(t)\) is piecewise continuous on \(\set{R}{+}{}\), of exponential order, and periodic with period \(T\), then
\[
  \Laplace{f(t)} = \frac{1}{1 - e^{-sT}} \int_0^T{e^-st f(t) \diffint{t}}
\]


\subsection{The dirac delta function}
\subsubsection{The unit impulse function}
The unit impulse function is defined as:
\[
  \diracdelta{a}{t - t_0} = \begin{cases}
    0,             & 0 \leqslant t < t_0 - a       \\
    \dfrac{1}{2a}, & t_0 - a \leqslant t < t_0 + a \\
    0,             & t \geqslant t_0 + a
  \end{cases}
\]


\subsubsection{The dirac delta function}
The dirac delta function is defined as:
\[
  \diracdelta{}{t - t_0} = \limit{a}{0}{\diracdelta{a}{t - t_0}} =
  \begin{cases}
    \infty, & t = t_0    \\
    0,      & t \neq t_0
  \end{cases}
\]

\begin{remark}
  The integral of the dirac delta function is:
  \[
    \int_0^\infty{\diracdelta{}{t - t_0} \diffint{t}} = 1
  \]
\end{remark}

The Laplace transform of the dirac delta function is:
\[
  \Laplace{\diracdelta{}{t - t_0}} = e^{-st_0}
\]


\subsection{Systems of linear differential equations}
When initial conditions are specified, the Laplace transform reduces a system of linear differential equations to a set of simultaneous algebraic equations in the transformed functions.

This is applicable to coupled springs, electrical networks, double pendulums\dots{}


\section{Orthogonal functions and fourier series}
\subsection{Orthogonal functions}
\subsubsection{The inner product}
The inner product, also called dot product, of vectors \(\vec{u} = (u_1,u_2,u_3)\) and \(\vec{v} = (v_1,v_2,v_3)\) is a scalar defined as
\[
  \left( \vec{u}, \vec{v} \right) = u_1 v_1 + u_2 v_2 + u_3 v_3 = \vec{u} \bullet \vec{v}
\]

Properties of the inner product:
\begin{align}
  \left( \vec{u}, \vec{v} \right)           & = \left( \vec{v}, \vec{u} \right)                                   \\
  \left( k\vec{u}, \vec{v} \right)          & = k \left( \vec{v}, \vec{u} \right), k \in \set{R}{}{}              \\
  \left( \vec{u}, \vec{u} \right)           & = 0 \iff \vec{u} = \vec{0}                                          \\
  \left( \vec{u} + \vec{v}, \vec{w} \right) & = \left( \vec{u}, \vec{w} \right) + \left( \vec{v}, \vec{w} \right)
\end{align}


The inner product of functions on the interval \([a,b]\) is defined as
\[
  (f_1, f_2) = \int_a^b{f_1(x)f_2(x) \diffint{x}}
\]


\subsubsection{Orthogonality}
Two vectors or functions are said to be orthogonal if their inner product is zero.
A set of real-valued functions \(\{\phi_0(x), \phi_1(x), \phi_2(x), \dots\}\) is said to be orthogonal on \([a,b]\) if
\[
  (\phi_m, \phi_n) = \int_a^b{\phi_m(x)\phi_n(x) \diffint{x}} = 0, m \neq n, \{m,n\} \in \set{N}{}{}
\]

The norm \(\normvec{u}\) of a vector \(u\) can be expressed using the inner product:
\[
  \normvec{u} = \sqrt{\left( \vec{u}, \vec{u} \right)}
\]

Similarly, the norm of a function \(\phi_n\) in an orthogonal set \(\{\phi_n(x)\}\), \(n \in \set{N}{}{}\), is
\[
  \norm{\phi_n(x)} = \sqrt{\int_a^b{{\phi_n(x)^2} \diffint{x}}}
\]

If \(\{\phi_n(x)\}\) is an orthogonal set of functions on the interval \([a,b]\) with the property that \(\norm{\phi_n(x)} = 1\) for all \(n \in \set{N}{}{}\), then \(\{\phi_n(x)\}\) is said to be an orthonormal set on the interval.

An orthogonal set is said to be a complete set if the only continuous function orthogonal to each member of the set is the zero function \(f(x) = 0\).

A set of real-valued functions \(\{\phi_n(x)\}\), \(n \in \set{N}{}{}\) is said to be orthogonal with respect to a weight function \(w(x)\) on \([a,b]\) if
\[
  \int_a^b{w(x) \phi_m(x)\phi_n(x) \diffint{x}} = 0, m \neq n, \{m,n\} \in \set{N}{}{}
\]

A real valued function \(f(x)\) is said to be periodic with \(T \neq 0\) if \(f(x + T) = f(x)\) for all \(x\) in the domain of \(x\).
The smallest value of \(T\) which satisfies this property is the fundamental period.


\subsubsection{Gram-Schmidt orthogonalization process}
For a linearly independent set of real-valued functions that are continuous on an interval \([a,b]\), an orthogonal can be made from this set using the Gram-Schmidt process.
The Gram-Schmidt process for turning the set of linearly independent real-valued functions \({f_n(x)}\) \(n \in \set{N}{}{}\) into the orthogonal set \({\phi_n(x)}\) \(n \in \set{N}{}{}\) is
\begin{align*}
  \phi_0(x) & = f_0(x)                                                                                                       \\
  \phi_1(x) & = f_1(x) - \frac{(f_1,\phi_0)}{\norm{\phi_0(x)}^2}\phi_0(x)                                                    \\
  \phi_2(x) & = f_2(x) - \frac{(f_2,\phi_0)}{\norm{\phi_0(x)}^2}\phi_0(x) - \frac{(f_2,\phi_1)}{\norm{\phi_1(x)}^2}\phi_1(x) \\
  \phi_n(x) & = f_n(x) - \sum_{k = 0}^{n - 1}{\frac{(f_n,\phi_k)}{\norm{\phi_k(x)}^2}\phi_k(x)}                              \\
\end{align*}


\subsubsection{Orthogonal series expansion}
An orthogonal series expansion of \(f\) or a generalized Fourier series is
\[
  f(x) = \sum_{n = 0}^\infty{\frac{\int_a^b{f(x)\phi_n(x) \diffint{x}}}{\norm{\phi_n(x)}^2}\phi_n(x)} = \sum_{n = 0}^\infty{\frac{(f,\phi_n)}{\norm{\phi_n(x)}^2}\phi_n(x)}
\]


\subsection{Fourier series}
The Fourier series of a function \(f\) defined on the interval \((-p,p)\)is given by
\[
  f(x) = \frac{a_0}{2} + \sum_{n = 1}^\infty \left( a_n \cos{\frac{n \pi}{p}x} + b_n \sin{\frac{n \pi}{p} x} \right)
\]
where
\begin{align*}
  a_0 & = \frac{1}{p} \int_{-p}^p{f(x) \diffint{x}}                         \\
  a_n & = \frac{1}{p} \int_{-p}^p{f(x) \cos{\frac{n \pi}{p}x} \diffint{x}}  \\
  b_n & = \frac{1}{p} \int_{-p}^p{f(x) \sin{\frac{n \pi}{p} x} \diffint{x}}
\end{align*}

Conditions of convergence where \(f(x)\) and \(f'(x)\)are piecewise continuous on the interval \((-p,p)\):
\begin{itemize}
  \item At a point of continuity: the Fourier series converges to \(f(x)\).
  \item At a point of discontinuity: the Fourier series converges to \(\frac{f(x^-) + f(x^+)}{2}\), where \(f(x^-) = \llimit{a}{x}{f(a)}\) and \(f(x^+) = \rlimit{a}{x}{f(a)}\).
\end{itemize}


\subsection{Fourier cosine and sine series}
\subsubsection{Odd and even functions}
\begin{description}
  \item[Odd:] a function is said to be an odd function of \(x\) if \(f(-x) = -f(x)\) (symmetry about the origin);
  \item[Even:] a function is said to be an even function of \(x\) if \(f(-x) = f(x)\) (symmetry about the \(y\)-axis).
\end{description}

Properties of even and odd functions:
\begin{itemize}
  \item The product of two even functions is even.
  \item The product of two odd functions is even.
  \item The product of an even function and an odd function is odd.
  \item The sum (difference) of two even functions is even.
  \item The sum (difference) of two odd functions is odd.
  \item If \(f\) is even, then
        \[
          \int_{-a}^a{f(x) \diffint{x}} = 2 \int_0^a{f(x) \diffint{x}}
        \]
  \item If \(f\) is odd, then
        \[
          \int_{-a}^a{f(x) \diffint{x}} = 0
        \]
\end{itemize}

Special Fourier series:
\begin{description}
  \item[Even function:] The Fourier series of an even function \((-p, p)\) is the cosine series (\(b_n = 0\)):
        \[
          f(x) = \frac{a_0}{2} + \sum_{n = 1}^\infty{a_n \cos{\frac{n \pi}{p}x}}
        \]
        where
        \[
          a_0 = \frac{2}{p} \int_0 ^p{f(x) \diffint{x}}
          \qquad a_n = \frac{2}{p} \int_0 ^p{f(x) \cos{\frac{n \pi}{p}x} \diffint{x}}
        \]
  \item[Odd function:] The Fourier series of an odd function \((-p, p)\) is the sine series (\(a_0 = a_n = 0\)):
        \[
          f(x) = \sum_{n = 1}^\infty {b_n \sin{\frac{n \pi}{p} x}}
        \]
        where
        \[
          b_n = \frac{2}{p} \int_0 ^p{f(x) \sin{\frac{n \pi}{p} x} \diffint{x}}
        \]
\end{description}


\subsubsection{Half-range expansion}
If the range of a function is given on \((0, L)\) rather than \((-p,p)\), an arbitrary definition of \(f\) on \((-L, 0)\) can be made:
\begin{enumerate}
  \item Reflect the graph of the function about the \(y\)-axis onto \((-L, 0)\) so the function is even on \((-L, L)\).
        This results in doing the half-range cosine expansion of \(f\) by taking \(p = L\) and using the cosine series.
  \item Reflect the graph of the function about the origin onto \((-L, L)\) so the function is odd on \((-L, L)\).
        This results in doing the half-range sine expansion of \(f\) by taking \(p = L\) and using the sine series.
  \item Define \(f\) on \((-L, L)\) by identity reflection (translation), \(f(x) = f(x + L)\).
        By taking \(p = \frac{L}{2}\) and using the classic Fourier series, this results in doing the periodic extension of \(f\) .
\end{enumerate}


\section{Boundary-value problems in rectangular coordinates}
\subsection{Separable partial differential equations (PDEs)}
The general form of a linear second-order partial differential equation is
\[
  A\npartialderivative{2}{u}{x} + B \partialderivative{^2 u}{x \partial y} + C \npartialderivative{2}{u}{y} + D \partialderivative{u}{x} + E \partialderivative{u}{y} + F u = G
\]
where \(A\), \(B\), \(C\), \(D\), \(E\), \(F\) and \(G\) are constants in \(\set{R}{}{}\) or functions of \(x\) and \(y\)

\begin{remark}
  If \(G = 0\), the equation is called a homogeneous equation; otherwise, it is nonhomogeneous .
\end{remark}

A solution of a linear PDE is a function of the two independent variables possessing all partial derivatives occurring in the equation and satisfying the equation in some region of the \(xy\)-plane.


\subsubsection{Separation of variables}
The method of separation of variables can be used to find a particular solution \(u\) of the form of a product of function of \(x\) and \(y\), where
\begin{gather*}
  u(x,y) = X(x) Y(y) \\
  \implies \partialderivative{u}{x} = X'Y \qquad \partialderivative{u}{y} = XY' \\
  \implies \npartialderivative{2}{u}{x} = X''Y \qquad \npartialderivative{2}{u}{y} = XY'' \qquad \partialderivative{^2 u}{x \partial y} = X'Y'
\end{gather*}
With this assumption, is is sometimes possible to reduce a linear PDE in two variables to two ODEs.

Separation of variables is not a general method for finding particular solutions as some linear partial differential equations are simply not separable.


\subsubsection{Superposition principle}
If \(u_1\), \(u_2\), \dots{}, \(u_k\) are solutions of a homogeneous linear partial differential equation, then the linear combination
\[
  u = c_1 u_1 + c_2 u_2 + \cdots + c_k u_k = \sum_{i = 1}^k{c_i u_i}
\]
where \(c_i\), \(i = 1, 2, \dots, k\) are constants in \(\set{R}{}{}\), is also a solution.


\subsubsection{Classification of equations}
The general linear second-order partial differential equation
\[
  A\npartialderivative{2}{u}{x} + B \partialderivative{^2 u}{x \partial y} + C \npartialderivative{2}{u}{y} + D \partialderivative{u}{x} + E \partialderivative{u}{y} + F u = G
\]
where \(A\), \(B\), \(C\), \(D\), \(E\), \(F\) and \(G\) are constants in \(\set{R}{}{}\), is said to be
\begin{align*}
   & \textbf{hyperbolic} \text{ if} & B^2 - 4AC & > 0 \\
   & \textbf{parabolic} \text{ if}  & B^2 - 4AC & = 0 \\
   & \textbf{elliptic} \text{ if}   & B^2 - 4AC & < 0
\end{align*}


\subsection{Classical PDEs and boundary-value problems}
The classical second-order partial differential equations which have important applications and mathematical physics are:
\begin{description}
  \item[The one-dimensional heat equation:]
        \[
          k \npartialderivative{2}{u}{x} = \partialderivative{u}{t}, \quad k \in\set{R}{+}{*}
        \]
  \item[The one-dimensional wave equation:]
        \[
          a^2 \npartialderivative{2}{u}{x} = \npartialderivative{2}{u}{t}
        \]
  \item[The two-dimensional Laplace's equation:]
        \[
          \npartialderivative{2}{u}{x} + \npartialderivative{2}{u}{y} = 0
        \]
\end{description}

Since the heat and wave equations depend on \(t\), there exists initial conditions at \(t = 0\). \newline
Moreover, one of the following boundary conditions can be used to define the behavior at boundaries:
\begin{description}
  \item[Dirichlet condition:] \(u\).
  \item[Neumann condition:] \(\partialderivative{u}{x}\).
  \item[Robin condition:] \(\partialderivative{u}{x} + hu\).
\end{description}


\subsubsection{Solving the classical PDEs and boundary-value problems}
\paragraph{Sturm-Liouville problem types}
Sturm-Liouville \Romannumeral{1}:
\[
  u''(x) + \lambda u(x) = 0, \quad u(0) = 0 \text{ and } u(L) = 0
\]
Non-trivial solution exists only for \(\lambda = \alpha^2 > 0\): the non-trivial solution is \(u_n(x) = \sin\alpha x = \sin\frac{n \pi}{L}x\).

Sturm-Liouville \Romannumeral{2}:
\[
  u''(x) + \lambda u(x) = 0, \quad u'(0) = 0 \text{ and } u'(L) = 0
\]
Non-trivial solutions exist for
\begin{description}
  \item[\(\lambda = 0\):] the non-trivial solution is \(u_0(x) = 1\).
  \item[\(\lambda = \alpha^2 > 0\):] the non-trivial solution is \(u_n(x) = \cos\frac{n \pi}{L}x\).
\end{description}

Sturm-Liouville \Romannumeral{3}:
\[
  u''(x) + \lambda u(x) = 0, \quad u(0) = 0 \text{ and } u'(L) = 0
\]
Non-trivial solution exists only for \(\lambda = \alpha^2 > 0\): the non-trivial solution is \(u_n(x) = \sin\alpha x = \sin\frac{(2n - 1) \pi}{2L}x\).

Sturm-Liouville \Romannumeral{4}:
\[
  u''(x) + \lambda u(x) = 0, \quad u'(0) = 0 \text{ and } u(L) = 0
\]
Non-trivial solution exists only for \(\lambda = \alpha^2 > 0\): the non-trivial solution is \(u_n(x) = \cos\alpha x = \cos\frac{(2n - 1) \pi}{2L}x\).




\paragraph{General method steps} \label{sec:general-method}
\begin{enumerate}
  \item Apply the change of variable \(u(x,y) = X(x)Y(y)\).
  \item Substitute the partial derivatives by the product of derivatives of \(X(x)\) and \(Y(y)\).
  \item Write the equation in fraction form with all \(X(x)\) and its derivatives on one side and all \(Y(y)\) and its derivatives one the other side.
  \item Equal those fractions to the separation variable \(-\lambda\).
  \item Write the ordinary differential equations using \(-\lambda\).
  \item Solve for \(X(x)\) and \(Y(y)\) using their ordinary differential equation and \(-\lambda\)
        \begin{olditemize}
          \item In the case of a first-order linear ordinary differential equation of the form \(X'(x) + P(x)X = f(x)\), the solution is:
          \begin{equation*}
            \begin{split}
              & X(x) = e^{-\int{P(x) \diffint{x}}} \int{e^{\int{P(x) \diffint{x}}}f(x) \diffint{x}} \\
              & f(x) = 0 \implies X(x) = Ce^{-\int{P(x) \diffint{x}}}, \quad C \in \set{R}{}{*}
            \end{split}
          \end{equation*}
          \item In the case of a second-order linear ordinary differential equation of the form \(X'' + P(\lambda) X = 0\), it must be solved for non-trivial solutions (\(X(x) \neq 0\)) with 3 cases in which the constants \(C_i\) are solved the homogeneous boundary conditions (or use Sturm-Liouville problem types):
          \begin{olddescription}
            \item[\(P(\lambda) = 0\):] \(X(x) = C_1 x + C_2, \quad \{C_1,C_2\} \in \set{R}{}{}\)
            \item[\(P(\lambda) = -\alpha^2 < 0\):] \(X(x) = C_3 \cosh\alpha x + C_4 \sinh\alpha x, \quad \{C_3,C_4\} \in \set{R}{}{}\)
            \item[\(P(\lambda) = \alpha^2< 0\):] \(X(x) = C_5 \cos\alpha x + C_6 \sin\alpha x, \quad \{C_5,C_6\} \in \set{R}{}{}\) (Do not forget that \(\sin n\pi = 0\) and \(\cos (2n + 1) \frac{\pi}{2} = 0\), \(n \in \set{N}{}{}\))
          \end{olddescription}
          \item In the case of another form of second-order ordinary differential equation, use other methods such as the auxiliary equation method.
        \end{olditemize}
  \item Form the product solution \(u_n(x,t) = X(x)Y(y)\).
  \item Combine the constants \(C_i\) as \(A_0\), \(A_n\) and \(B_n\) (functions of \(n\)).
  \item Apply the superposition principle \(u(x,y) = \sum_{n = 1}^\infty{u_n(x,y)}\).
  \item Use initial and boundary conditions and Fourier series expansion to solve for \(A_0\), \(A_n\) and \(B_n\).
  \item Obtain the complete solution \(u(x,t) = X(x)Y(y)\) by substituting \(A_0\), \(A_n\) and \(B_n\) into the solution.
\end{enumerate}


\subsection{Heat equation}
Suppose a rod of length \(L\) has a circular cross-sectional area \(A\) and coincides with the \(x\)-acis on the interval \([0, L]\).
The following assumption are made:
\begin{itemize}
  \item The flow of heat within the rod takes place only in the \(x\)-direction.
  \item The lateral, or curved, surface of the rod is insulated, meaning no heat escapes from this surface.
  \item Not heat is being generated withing the rod.
  \item The rod is homogeneous, meaning its density \(\rho\) is constant.
  \item The specific heat \(\gamma\) and thermal conductivity \(K\) of the material of the rod are constants.
\end{itemize}

The governing equation for this problem is:
\[
  k \npartialderivative{2}{u}{x} = \partialderivative{u}{t}, \quad k > 0, 0 \leqslant x \leqslant L, t \geqslant 0
\]
which is subjected to the following conditions:
\begin{description}
  \item[Boundary conditions:] \(u(0,t) = 0\), \(u(L,t) = 0\), \(t \geqslant 0\).
  \item[Initial conditions:] \(u(x,0) = f(x)\), \(0 \leqslant x \leqslant L\).
\end{description}

Using the general method steps for this problem, the complete solution of the one-dimensional heat equation is:
\[
  u(x,t) = \sum_{n = 1}^\infty \left( \frac{2}{L} \int_0^L{f(x) \sin\frac{n \pi}{L} x \diffint{x}} \right) e^{-k t \left( \frac{n\pi}{L} \right)^2} \sin\frac{n \pi}{L} x \\
\]


\subsection{Wave equation}
Consider a string of length \(L\) stretched taut between two point on the \(x\)-axis at \(x = 0\) and \(x = L\).
The following assumptions are made:
\begin{itemize}
  \item The string is perfectly flexible.
  \item The string is homogeneous, meaning its density \(\rho\) is constant.
  \item The displacement \(u\) are small compared to the length of the string.
  \item The slope of the curve is small at all points.
  \item The tension \(\vec{T}\) acts tangent to the string, and its magnitude \(T\) is the same at all points.
  \item The tension is large compared with the force of gravity.
  \item No other external forces act on the string.
\end{itemize}

The governing equation for this problem is:
\[
  a^2 \npartialderivative{2}{u}{x} = \npartialderivative{2}{u}{t}, \quad 0 \leqslant x \leqslant L, t \geqslant 0
\]
which is subjected to the following conditions:
\begin{description}
  \item[Boundary conditions:] \(u(0,t) = 0\), \(u(L,t) = 0\), \(t \geqslant 0\).
  \item[Initial conditions:] \(u(x,0) = f(x)\), \(\left. \partialderivative{u}{t} \right|_{t = 0} = g(x)\), \(0 \leqslant x \leqslant L\)
\end{description}

Using the general method steps for this problem, the complete solution of the one-dimensional wave equation is:
\begin{multline*}
  u(x,t) = \sum_{n = 1}^\infty \left[ \left( \frac{2}{L} \int_0^L{f(x) \sin\frac{n \pi}{L} x \diffint{x}} \right) \cos\frac{n \pi a}{L}t \right. \\
  \left.+ \left( \frac{2}{n \pi a} \int_0^L{g(x) \sin\frac{n \pi}{L} x \diffint{x}} \right) \sin\frac{n \pi a}{L}t \right] \sin\frac{n \pi}{L} x
\end{multline*}


\subsection{Laplace's equation}
Laplace's equation in two and three dimensions occurs in time-independent problems involving potentials such as electrostatic, gravitational and velocity in fluid mechanics.
A solution of Laplace's equation can also be interpreted as a steady-state temperature distribution.

Suppose we wish to find the steady-state temperature \(u(x,y)\) in a rectangular plate whose vertical edges are insulated and whose lower and upper edges are maintained at temperatures 0 and \(f(x)\)

The governing equation for this problem is:
\[
  \npartialderivative{2}{u}{x} + \npartialderivative{2}{u}{y} = 0, \quad 0 \leqslant x \leqslant a, 0 \leqslant y \leqslant b
\]
which is subjected to the following conditions:
\begin{description}
  \item[Boundary conditions:] \(\left. \partialderivative{u}{x} \right|_{x = 0} = 0\), \(\left. \partialderivative{u}{x} \right|_{x = a} = 0\), \(0 \leqslant y \leqslant b\).
  \item[Initial conditions:] \(u(x,0) = 0\), \(u(x,b) = f(X)\), \(0 \leqslant x \leqslant a\)
\end{description}

Using the general method steps for this problem, the complete solution of the two-dimensional Laplace's equation is:
\begin{multline*}
  u(x,y) = \frac{y}{ab} \int_0^a{f(x) \diffint{x}} \\
  + \frac{2}{a} \sum_{n = 1}^\infty{\frac{1}{\sinh\frac{n \pi b}{a}} \int_0^a{f(x) \cos\frac{n \pi}{a} \diffint{x}}} \sinh\frac{n \pi}{a}y \cos\frac{n \pi}{a}x
\end{multline*}


\subsubsection{Dirichlet problem}
A Dirichlet problem is a boundary-value problem in which a solution to an elliptic partial differential equation within a region \(R\) is seek, such that all boundary conditions are Dirichlet conditions.

A Dirichlet problem for a rectangle can be solved by separation of variables when homogeneous boundary conditions are specified on two parallel boundaries.
If the boundary conditions are nonhomogeneous on at least two perpendicular sides, the boundary-value problem is split into two problems, each of which has a homogeneous boundary conditions on parallel boundaries.


\subsection{Nonhomogeneous boundary-value problems}
A boundary-value problem is said to be nonhomogeneous if either the partial differential equation or the boundary conditions are nonhomogeneous.

The method of separation of variable does not always work.


\subsubsection{Time-independent}
A change of the dependent variable is employed
\[
  u(x,t) = v(x,t) + \psi(x)
\]
This transforms a nonhomogeneous partial differential equation boundary value problem into one involving an ordinary differential equation and the another involving a homogeneous partial differential equation, solvable by separation of variables.


\paragraph{General method steps}
\begin{enumerate}
  \item Obtain the governing equation, and the boundary and initial conditions from the problem statement.
  \item Apply the change of dependent variable \(u(x,t) = v(x,t) + \psi(x)\).
  \item Substitute the change of dependent variable into the governing equation, and the boundary and initial conditions.
  \item Separate the problem into 2 sub-problems (set the boundary conditions in terms of \(v(x,t)\) at 0 and express the ones of \(\psi(x)\) accordingly).
  \item Solve the ordinary differential equation for \(\psi(x)\) using the boundary conditions.
  \item Solve the partial differential equation for \(v(x,t)\) using the general method (\vref{sec:general-method}).
  \item Obtain the complete solution using \(u(x,t) = v(x,t) + \psi(x)\).
\end{enumerate}


\subsubsection{Time-dependent}
A change of the dependent variable is employed
\begin{align*}
  u(x,t)                    & = v(x,t) + \psi(x,t)                                  \\
  \text{ where: } \psi(x,t) & = u(0,t) + \frac{x}{a} \left[ u(a,t) - u(0,t) \right]
\end{align*}
This transforms a nonhomogeneous partial differential equation boundary value problem into a nonhomogeneous partial differential equation with homogeneous boundary and initial conditions.


\paragraph{General method steps}
\begin{enumerate}
  \item Obtain the governing equation, and the boundary and initial conditions from the problem statement.
  \item Apply the change of variable \(u(x,t) = v(x,t) + \psi(x,t)\).
  \item Find \(\psi(x,t) = u(0,t) + \frac{x}{a} \left[ u(a,t) - u(0,t) \right]\).
  \item Substitute the change of dependent variable into the governing equation, and the boundary and initial conditions.
  \item Write \(v\) terms on the left and \(\psi\) terms on the right.
  \item Solve for \(v(x,t)\) (since \(\psi(x,t)\) is already known) by assuming \(v(x,t)\) and the \(\psi\) term can be written as Fourier sine or cosine series for a fixed \(t\) (the coefficients are in terms of \(t\) and \(n\) only inside the infinite sums as \(x\) appears in the sine/cosine term).
  \item Solve for the coefficient \(\psi_n(t)\) of the \(\psi\) term.
  \item Find the partial derivatives of \(v(x,t)\) in their Fourier sine/cosine expansion form.
  \item Substitute those sine/cosine expansions into the governing equation as well as the one of the \(\psi\) term.
  \item Equate the coefficients \(v_n(t)\) and \(\psi_n(t)\) of the \(\psi\) term, providing that the sine/cosine terms are the same.
  \item Solve for \(v_n(t)\) (solve constants using boundary and initial conditions).
  \item Obtain the complete solution using \(u(x,t) = v(x,t) + \psi(x,t)\).
\end{enumerate}
\end{document}