\documentclass[10pt, twocolumn]{article}

%%%%%%%%%%%%%%%%%%%%%%%%%%%%%%%%%%%%%%%%%%%%%%%%%%%%%%%%%%%%%%%%%%%%%%%%%%%%%%%
%%%% Cover page
\title{ENGR 243: Dynamics}
\date{\today}
\author{Anthony Bourboujas}

\makeatletter
\let\Title\@title
\let\Author\@author
\let\Date\@date
\makeatother

%%%%%%%%%%%%%%%%%%%%%%%%%%%%%%%%%%%%%%%%%%%%%%%%%%%%%%%%%%%%%%%%%%%%%%%%%%%%%%%
%%%% Preamble
%%%%%%%%%%%%%%%%%%%%%%%%%%%%%%%%%%%%%%%%%%%%%%%%%%%%%%%%%%%%%%%%%%%%%%%%%%%%%%%
%%%% Packages
\usepackage[utf8x]{inputenc} % Accept different input encodings
\usepackage[T1]{fontenc} % Standard package for selecting font encodings
\usepackage{lmodern} % Font name; classic: lmodern
\usepackage[english]{babel} % Multilingual support for LaTeX
% \usepackage{abstract} % Control the typesetting of the abstract environment
\usepackage{amsmath} % AMS mathematical facilities for LaTeX
\usepackage{amssymb} % TeX fonts from the American Mathematical Society
\usepackage{amsthm} % Typesetting theorems (AMS style)
\usepackage{array} % Extending the array and tabular environments
% \usepackage[backend=biber,style=ieee,sorting=none]{biblatex}
\usepackage{bold-extra} % Use bold small caps and typewriter fonts
\usepackage{cellspace} % Ensure minimal spacing for table cells
\usepackage{chemformula} % Command for typesetting chemical formulas and reactions
% \usepackage{colortbl} % Add colour to LaTeX tables
\usepackage{comment} % Selectively include/exclude portions of text
\usepackage{csquotes} % Context sensitive quotation facilities
% \usepackage[en-US,showdow]{datetime2} % Formats for dates, times and time zones
% \usepackage{diagbox} % Table heads with diagonal lines
\usepackage{enumitem} % Control layout of itemize, enumerate, description
\usepackage{esint} % Extended set of integrals for Computer Modern
\usepackage{graphicx} % Enhanced support for graphics
% \usepackage{listings} % Typeset source code listings using LaTeX
% \usepackage{lipsum} % Easy access to the Lorem Ipsum dummy text
\usepackage{mathrsfs} % Support for using RSFS fonts in maths
% \usepackage{matlab-prettifier} % Pretty-print Matlab source code
\usepackage{moreverb} % Extended verbatim
\usepackage{multicol} % Intermix single and multiple columns
\usepackage{multirow} % Create tabular cells spanning multiple rows
% \usepackage{pgfplots} % Plots
% \usepackage{pgfplotstable} % Loads, rounds, format and post-processes numerical tables (generates table from CSV)
% \usepackage{pdfpages} % Include PDF document in LaTeX
% \usepackage{rotating} % Rotation tools, including rotated full-page floats with sidewaysfigure
\usepackage[scr]{rsfso} % A mathematical calligraphic font based on rsfs
\usepackage{setspace} % Set space between lines
\usepackage{soul} % Hyphenation for letterspacing, underlining, and more
\usepackage{threeparttable} % Tables with captions and notes all the same width
% \usepackage{verbatim} % Reimplementation of and extensions to LaTeX verbatim
\usepackage{wrapfig} % Produces figures which text can flow around
\usepackage{xcolor} % Driver-independent color extensions for LaTeX
\usepackage{xurl} % Verbatim with URL-sensitive line breaks, allow URL breaks at any alphanumerical character

%%%%%%%%%%%%%%%%%%%%%%%%%%%%%%%%%%%%%%%%%%%%%%%%%%%%%%%%%%%%%%%%%%%%%%%%%%%%%%%
%%%% Lengths
% 1cm = 10mm = 28pt = 1/2.54in
% 1ex = height of a lowercase 'x' in the current font
% 1em = width of an uppercase 'M' in the current font

%%%% Spacing in math mode
% \!                         = -3/18em
% \,                         = 3/18em
% \:                         = 4/18em
% \;                         = 5/18em
% \ (space after backslash!) = space in normal text
% \quad                      = 1em
% \qquad                     = 2em

% \setlength{\baselineskip}{1em} % Vertical distance between lines in a paragraph
% \renewcommand{\baselinestretch}{1.0} % A factor multiplying \baslineskip
\setlength{\columnsep}{0.75cm} % Distance between columns
% \setlength{\columnwidth}{} % The width of a column
\setlength{\columnseprule}{1pt} % The width of the vertical ruler between columns
% \setlength{\evensidemargin}{} % Margin of even pages, commonly used in two-sided documents such as books
% \setlength{\linewidth}{} % Width of the line in the current environment.
% \setlength{\oddsidemargin}{} % Margin of odd pages, commonly used in two-sided documents such as books
% \setlength{\paperwidth}{} % Width of the page
% \setlength{\paperheight}{} % Height of the page
\setlength{\parindent}{0cm} % Paragraph indentation
\setlength{\parskip}{6pt} % Vertical space between paragraphs
% \setlength{\tabcolsep}{} % Separation between columns in a table (tabular environment)
% \setlength{\textheight}{} % Height of the text area in the page
% \setlength{\textwidth}{} % Width of the text area in the page
% \setlength{\topmargin}{} % Length of the top margin
\setlist{
  %%%% Vertical spacing
  topsep = 0pt,
  partopsep = 0pt,
  parsep = 0pt,
  itemsep = 0pt,
  %%%% Horizontal spacing
  leftmargin = 0.5cm,
  rightmargin = 0cm,
  % listparindent = 0cm,
  % labelwidth = 0cm,
  % labelsep = 0cm,
  % itemindent = 0cm
}
\addtolength{\cellspacetoplimit}{2pt}
\addtolength{\cellspacebottomlimit}{2pt}

%%%%%%%%%%%%%%%%%%%%%%%%%%%%%%%%%%%%%%%%%%%%%%%%%%%%%%%%%%%%%%%%%%%%%%%%%%%%%%%
%%%% Page layout
\usepackage{layout} % View the layout of a document
\usepackage{geometry} % Flexible and complete interface to document dimensions
% 1cm = 10mm = 28pt = 1/2.54in
% ex = height of a lowercase 'x' in the current font
% em = width of an uppercase 'M' in the current font
\geometry{
  a4paper,
  top         = 1cm,
  bottom      = 1cm,
  left        = 1.5cm,
  right       = 1.5cm,
  includehead = true,
  includefoot = true,
  landscape   = false, % Paper orientation
  twoside     = false,
}
% \geometry{showframe} % Show paper outline for the text area and page

%%%%%%%%%%%%%%%%%%%%%%%%%%%%%%%%%%%%%%%%%%%%%%%%%%%%%%%%%%%%%%%%%%%%%%%%%%%%%%%
%%%% Header and footer style
\usepackage{fancyhdr} % Extensive control of page headers and footers in LaTeX
\pagestyle{fancy}
% Options: \leftmark (chapter title), \rightmark(section title), \thepage (page number), \thechapter(chapter number), \thesection (section number)
\lhead{\thetitle}
\chead{}
\rhead{}
\lfoot{}
\cfoot{\thepage}
\rfoot{}

%%%%%%%%%%%%%%%%%%%%%%%%%%%%%%%%%%%%%%%%%%%%%%%%%%%%%%%%%%%%%%%%%%%%%%%%%%%%%%%
%%%% URL insertion settings
\usepackage{hyperref} % Extensive support for hypertext in LaTeX
\definecolor{black}{RGB}{0, 0, 0} % rgb(0, 0, 0)
\definecolor{blue}{RGB}{0, 0, 255} % rgb(0, 0, 255)
\hypersetup{
  % unicode            = true,
  pdftitle           = {\thetitle},
  pdfauthor          = {\theauthor},
  % pdfsubject       = {},
  %%%% Reference
  % bookmarks          = true,
  bookmarksnumbered  = true,
  bookmarksopen      = true, % Open the bookmarks
  bookmarksopenlevel = 2, % Open until 1 level (section)
  %%%% Bookmarks
  breaklinks         = true,
  pdfborder          = {0 0 0},
  % backref            = true, % Add links into bibliography
  % pagebackref        = true,
  % hyperindex         = true, % Add links into index
  %%%% Color
  colorlinks         = true,
  linkcolor          = black, % Internal links color
  citecolor          = black,
  urlcolor           = blue, % Hyperlinks color
  filecolor          = black,
}

\usepackage{varioref} % Intelligent page reference
\usepackage[capitalise,noabbrev]{cleveref}
\usepackage{prettyref} % Make label references "self-identity" with \prettyref{#1}
\newrefformat{cha}{chapter \textbf{\nameref{#1}} \vpageref{#1}} % {chapter \textbf{\nameref{#1}} on page \pageref{#1}}
\newrefformat{sec}{section \textbf{\nameref{#1}} \vpageref{#1}} % {section \textbf{\nameref{#1}} on page \pageref{#1}}
% \newrefformat{fig}{\vref{#1}} % {Figure \ref{#1} on page \pageref{#1}}
% \newrefformat{tab}{\vref{#1}} % {Table \ref{#1} on page \pageref{#1}}
% \newrefformat{eqn}{\vref{#1}}
% \newrefformat{lis}{\emph{\nameref{#1}} \vpageref{#1}}

%%%%%%%%%%%%%%%%%%%%%%%%%%%%%%%%%%%%%%%%%%%%%%%%%%%%%%%%%%%%%%%%%%%%%%%%%%%%%%%
%%%% Physics units settings
% Dependencies
\usepackage{booktabs} % Publication quality tables in LaTeX
\usepackage{caption} % Customizing captions in floating environments
\usepackage{helvet} % Load Helvetica, scaled
\usepackage{cancel} % Place lines through maths formulae

\usepackage{siunitx} % A comprehensive (SI) units package
\sisetup{
  exponent-product     = \cdot, % Symbol between number and power of ten
  group-minimum-digits = 5, % Number of digits when 3 digits separation appear
  % inter-unit-product   = \cdot, % Symbol between units (when several units are used)
  output-complex-root  = \ensuremath{i}, % How i math should be seen
  % prefixes-as-symbols  = false, % Translate prefixes (kilo, centi, milli, micro,...) into a power of ten
  separate-uncertainty = true, % Write uncertainty with +-
  scientific-notation  = engineering,
}

%%%%%%%%%%%%%%%%%%%%%%%%%%%%%%%%%%%%%%%%%%%%%%%%%%%%%%%%%%%%%%%%%%%%%%%%%%%%%%%
%%%% Theorems and proofs
\numberwithin{equation}{section}
% \makeatletter
% \g@addto@macro\th@remark{\thm@headpunct{:}}
% \makeatother
\theoremstyle{remark}
\newtheorem*{example}{Example}
\newtheorem*{remark}{Remark}

%%%%%%%%%%%%%%%%%%%%%%%%%%%%%%%%%%%%%%%%%%%%%%%%%%%%%%%%%%%%%%%%%%%%%%%%%%%%%%%
%%%% User-defined environments
% Remove the space before the enumerate and itemize environments
\let\oldenumerate\enumerate % Keep a copy of \enumerate (or \begin{enumerate})
\let\endoldenumerate\endenumerate % Keep a copy of \endenumerate (or \end{enumerate})
\renewenvironment{enumerate}{
  \begin{oldenumerate}
    \vspace{-6pt}
    }{
  \end{oldenumerate}
}

\let\olditemize\itemize % Keep a copy of \itemize (or \begin{itemize})
\let\endolditemize\enditemize % Keep a copy of \enditemize (or \end{itemize})
\renewenvironment{itemize}{
  \begin{olditemize}
    \vspace{-6pt}
    }{
  \end{olditemize}
}

\let\olddescription\description % Keep a copy of \description (or \begin{description})
\let\endolddescription\enddescription % Keep a copy of \enddescription (or \end{description})
\renewenvironment{description}{
  \begin{olddescription}
    \vspace{-6pt}
    }{
  \end{olddescription}
}

%%%%%%%%%%%%%%%%%%%%%%%%%%%%%%%%%%%%%%%%%%%%%%%%%%%%%%%%%%%%%%%%%%%%%%%%%%%%%%%
%%%% User-defined commands
\newcommand{\Romannumeral}[1]{\MakeUppercase{\romannumeral #1}} % Capital roman numbers
% \newcommand{\gui}[1]{\og #1 \fg{}} % French quotation marks
\renewcommand{\thefootnote}{[\arabic{footnote}]}

%%% Figure command
%% Include SVG files
\newcommand{\executeiffilenewer}[3]{
  \ifnum\pdfstrcmp{\pdffilemoddate{#1}}
    {\pdffilemoddate{#2}}>0
    {\immediate\write18{#3}}\fi
}
\newcommand{\includesvg}[1]{
  \executeiffilenewer{#1.svg}{#1.pdf}
  {
    % Inkscape must be installed in PATH and the user must include '--shell-escape' in the build arguments
    inkscape #1.svg --export-type=pdf --export-latex
  }
  \input{#1.pdf_tex}
}

%%% Math commands
%% Tables (requires cellspace package)
\newcolumntype{L}{>{\(\displaystyle}Cl<{\)}} % Column type for left-aligned math column
\newcolumntype{D}{>{\(\displaystyle}Cc<{\)}} % Column type for centered math column

%% Functions
\newcommand{\constant}{\mathrm{constant}} % Constant
\newcommand{\abs}[1]{\left| #1 \right|} % Absolute function
\newcommand{\erf}[1]{\mathrm{erf} \left( #1 \right)} % Error function
\newcommand{\erfc}[1]{\mathrm{erfc} \left( #1 \right)} % Complementary error function
\newcommand{\unitstep}[1]{\,\mathcal{U}\left( #1 \right)} % Unit step function
\newcommand{\diracdelta}[2]{\,\delta_{#1}\left( #2 \right)} % Dirac delta function


%% Derivatives and integrals
\newcommand{\diff}[2]{\mathrm{d}^{#1} #2} % Letter 'd' of differentials
\newcommand{\diffint}[1]{\,\diff{}{#1}} % Differential with a space for integrals
\newcommand{\derivative}[2]{\frac{\diff{}{#1}}{\diff{}{#2}}} % Derivative
\newcommand{\nderivative}[3]{\frac{\diff{#1}{#2}}{\diff{}{#3^{#1}}}} % Derivative of degree n
\newcommand{\partialderivative}[2]{\frac{\partial #1}{\partial #2}} % Partial derivative
\newcommand{\npartialderivative}[3]{\frac{\partial^{#1} #2}{\partial #3^{#1}}} % Partial derivative of degree n
\newcommand{\direcderivative}[2]{D_{\vec{#1}}\,#2} % Directional derivative

\newcommand{\Laplace}[1]{\mathcal{L}\left\{ #1 \right\}} % Laplace transform notation
\newcommand{\invLaplace}[1]{\mathcal{L}^{-1}\left\{ #1 \right\}} % Inverse Laplace transform notation

%% Set
\newcommand{\set}[3]{\mathbb{#1}_{#2}^{#3}} % Set of numbers
\newcommand{\integerset}{\mathbb{Z}} % Set of integer numbers (compatibility)
\newcommand{\realset}{\mathbb{R}} % Set of real numbers (compatibility)

%% Limits
\newcommand{\limit}[3]{\lim_{#1 \to #2}{#3}} % Limit from a point to another
\newcommand{\rlimit}[3]{\lim_{#1 \to #2^{+}}{#3}} % Right limit from a point to another
\newcommand{\llimit}[3]{\lim_{#1 \to #2^{-}}{#3}} % Left imit from a point to another
\newcommand{\modulus}[1]{\,\left[ #1 \right]} % Modulus notation

%% Vectors
\newcommand{\ivec}{\hat{\mathrm{i}}} % i vector
\newcommand{\jvec}{\hat{\mathrm{j}}} % j vector
\newcommand{\kvec}{\hat{\mathrm{k}}} % k vector
\renewcommand{\Vec}[1]{\overrightarrow{#1}} % Vector notation for expression with more than one letter
\newcommand{\norm}[1]{\left\| #1 \right\|} % Norm notation for expression with just one letter
\newcommand{\normvec}[1]{\left\| \vec{#1} \right\|} % Norm notation for expression with just one letter
\newcommand{\Normvec}[1]{\left\| \Vec{#1} \right\|} % Norm notation for expression with more than one letter
\newcommand{\comp}[2]{\mathrm{comp}_{\vec{#2}}\vec{#1}} % Components
\newcommand{\proj}[2]{\mathrm{proj}_{\vec{#2}}\vec{#1}}
\newcommand{\grad}[1]{\vec{\nabla}#1} % Gradient notation
\newcommand{\frames}[2]{\left( #1 \right)_{#2}} % Frame definition

\newcommand{\curl}[1]{\mathrm{curl}\,\vec{#1}} % Curl of a vector field
\newcommand{\divergence}[1]{\mathrm{div}\,\vec{#1}} % Divergence of a vector field


%%%%%%%%%%%%%%%%%%%%%%%%%%%%%%%%%%%%%%%%%%%%%%%%%%%%%%%%%%%%%%%%%%%%%%%%%%%%%%%
%%%% Beginning of the document
\begin{document}
\maketitle % Insert the cover page
% \tableofcontents
% \layout % Show a drawing of page layout
% \renewcommand{\abstractname}{} % Change the abstract title

\part{Kinematics}
\section{Particles}
Dynamics use equations that dictate the evolution of a system usually over time.
From a mechanics perspective, dynamics is divided in two parts:
\begin{description}
  \item[Kinematics:] pure motion, displacement, velocity, acceleration, where the effects of mass and forces are ignored
  \item[Kinetics:] the effects of mass, inertia and forces are considered
\end{description}

A particle is an object with a mass but no dimension.
We use particles because engineering is all about approximation and simplification of a system and they let us simplify systems.


\subsection{Rectilinear motion}
\subsubsection{Position, velocity and acceleration}
Important variable notations:
\begin{itemize}
  \item Position \(x\)
  \item Velocity \(v = \dot{x}\)
  \item Acceleration \(a = \dot{v} = \ddot{x}\)
  \item Origin \(O\)
  \item Time \(t\)
\end{itemize}

\paragraph{Position \(x\)}
To express position, the basis need to be defined (origin and axis).
Position is usually a function of time such as \(x = 6t^2 - t^3\).

\paragraph{Velocity \(v\)}
The velocity is the change of position over time.
\begin{itemize}
  \item Average velocity: \(v_\mathrm{avg} = \frac{\Delta x}{\Delta t}\)
  \item Instantaneous velocity: \(v = \derivative{x}{t}\)
\end{itemize}

\paragraph{Acceleration \(a\)}
The acceleration is the change of velocity over time.
\begin{itemize}
  \item Average acceleration: \(a_\mathrm{avg} = \frac{\Delta v}{\Delta t}\)
  \item Instantaneous acceleration: \(a = \derivative{v}{t} = \nderivative{2}{x}{t} = v \derivative{v}{x}\)
\end{itemize}

Determining increasing and decreasing velocity (acceleration and deceleration)
\begin{itemize}
  \item If \(a\) and \(v\) have the same signs, then \(v\) increases with time
  \item If \(a\) and \(v\) have opposite signs, then \(v\) is decreasing with time
\end{itemize}

\begin{remark}
  The derivative of acceleration is called "jerk" \(j\) and \(j = \nderivative{3}{x}{t} = \dddot{x}\)
\end{remark}


\subsubsection{Determining the motion of a particle}
In this part, we will figure out how a particle moves given the initial conditions.
Thus, we use integration with respect to time, letting us to get velocity from acceleration and position from velocity.

\paragraph{Acceleration as a function of time}
Let us first compute the velocity \(v\).
Given acceleration \(a = f(t) = \derivative{v}{t}\), we will find the velocity \(v\) and the position \(x\):
\[
  v = \int{a \diffint{t}} = \int{f(t) \diffint{t}}
\]

To define a system uniquely, we use definite integrals with integration limits (get rid of the \(+ C\)):
\[
  v_2 - v_1 = \int_{t_1}^{t_2}{a \diffint{t}} = \int_{t_1}^{t_2}{f(t) \diffint{t}}
\]

By setting lower limits as initial conditions (i.e. \(t_1 = t_0\) and \(v_1 = v_0\)), and setting upper limits as variables (i.e. \(t_2 = t\) and \(v_2 = v\)), we get:
\[
  v = \int_{t_0}^t {a \diffint{t}} + v_0 = \int_{t_0}^t {f(t) \diffint{t}} + v_0
\]

Then, we look at the position \(x\).
Similarly to the velocity, we have:
\begin{align*}
  x & = \int_{t_0}^t {v \diffint{t}} + x_0                                           \\
    & = \int_{t_0}^t {\left( \int_{t_0}^t {a \diffint{t}} + v_0 \right) \diffint{t}} \\
    & = \iint_{t_0}^t {a \diffint{t}^2} + v_0 \left( t - t_0 \right) + x_0
\end{align*}

\paragraph{Acceleration as a function of position}
Recall that \(a = v \derivative{v}{x}\), which leads to get the velocity as:
\[
  v^2 = 2\int_{x_0}^x {a \diffint{x}} + {v_0}^2
\]

\paragraph{Acceleration as a function of velocity}
Again using \(a = v \derivative{v}{x}\), we have
\[
  \int{\diffint{x}} = \int{\frac{v}{a} \diffint{v}}
\]

Problem solving strategy:
\begin{enumerate}
  \item Define the coordinate system (origin and coordinate axes)
  \item Write down the assumptions, the knowns and the unknowns
  \item Determine the high lever problem solving sequence (the general steps to solve)
  \item Solve the problem
\end{enumerate}


\subsection{Special cases and relative motion}
\subsubsection{Gravity}
Assumptions:
\begin{itemize}
  \item constant acceleration
  \item object is in free fall
\end{itemize}
\[
  a = -g = -\SI{9.81}{\metre\per\second} = -32.2 ft/s^2
\]

\paragraph{Function of time \(t\)}
\begin{align*}
  a(t) & = -g = \derivative{v}{t}                          \\
  v(t) & = -g(t - t_0) + v_0                               \\
  x(t) & = -\frac{1}{2} g (t - t_0)^2 + v_0(t - t_0) + x_0
\end{align*}

\paragraph{Function of postion \(x\)}
\begin{align*}
  a(x) & = -g = v \derivative{v}{x}     \\
  v(x) & = \sqrt{{v_0}^2 - 2g(x - x_0)}
\end{align*}


\subsubsection{Motion of several particles}
The relative position of \(B\) with respect to a point \(A\) is \(x_{B/A} = x_B - x_A\).
The relative velocity of \(B\) with respect to point \(A\) is \(v_{B/A} = v_B - v_A\).
The relative acceleration of \(B\) with respect to point \(A\) is \(a_{B/A} = a_B - a_A\).

\begin{remark}
  Understanding the signs:
  \begin{itemize}
    \item \(x_{B/A} > 0\) means \(B\) is on the positive side of \(A\)
    \item \(v_{B/A} > 0\) means \(B\) is moving in the positive direction compared to \(A\)
    \item \(q_{B/A} > 0\) means \(B\) is accelerating in the positive direction compared to \(A\)
  \end{itemize}
\end{remark}
Sometimes, the motion of particles are dependent on each other (ex: pulleys, gears).
This is called \emph{dependent motion} and can be related mathematically by the constraint equation.

In a system with 2 pulleys, there is only one degree of freedom in the system: \(x_A + 2x_B = l\), where \(l\) is the length of the rope.
In a system with 4 pulleys, there are two degree of freedom in the system, implying:
\begin{align*}
  2x_A + 2x_B + x_C = l \\
  2v_A + 2v_B + v_C = 0 \\
  2a_A + 2a_B + a_C = 0
\end{align*}

\addtocounter{subsection}{1}
\subsection{Curvilinear motion of particles}
In order to describe a motion in more than one dimension, we use the position vector \(\vec{p}\).
The velocity vector \(\vec{v}\) is always tangent to the path of the particle.
Now, we can better define the speed as \(\normvec{v} = \sqrt{x^2 + y^2 + z ^2}\).
The effect of acceleration is now more important: it can change the velocity's direction \emph{and/or} magnitude.

\begin{remark}
  By bringing the velocity vectors to a common origin, they form a curve called hodograph and the acceleration is tangent to this curve.
\end{remark}

Combining motion in more that a dimension and relative motion, we have the following relations:
\begin{align*}
  \vec{p}_B & = \vec{p}_A + \vec{p}_{B/A} \\
  \vec{v}_B & = \vec{v}_A + \vec{v}_{B/A} \\
  \vec{a}_B & = \vec{a}_A + \vec{a}_{B/A}
\end{align*}
This only works when there no rotation of the axes at \(A\).


\subsubsection{Projectile motion}
For a projectile motion where air resistance in negligible, we have on the \(x\)-axis
\[
  x = v_{x,0}(t - t_0) + x_0
\]
and on the \(y\)-axis (\(a_y = -g\))
\begin{align*}
  y     & = \frac{1}{2}a_y(t - t_0)^2 + v_{y,0}(t - t_0) + y_0 \\
  v_y   & = a_y(t - t_0) + v_{y,0}                             \\
  v_y^2 & = 2a_y(y - y_0) + {v_{y,0}}^2
\end{align*}


\subsection{Non-rectangular components}
\subsubsection{Tangential and normal components}
Tangential and normal unit vectors \(\hat{\mathrm{e}}_t\) and \(\hat{\mathrm{e}}_n\) can be defined at the position of the particle.
Using these, we have
\begin{align*}
  \vec{v} & = v\hat{\mathrm{e}}_t                                                                                                      \\
  \vec{a} & = a_t\hat{\mathrm{e}}_t + a_n\hat{\mathrm{e}}_n = \derivative{v}{t}\hat{\mathrm{e}}_t + \frac{v^2}{\rho}\hat{\mathrm{e}}_n
\end{align*}
where \(\rho\) is the radius of curvature of the path.


\subsubsection{Radial and transverse components}
\paragraph{Polar coordinates}
Radial and transverse components are use in the polar coordinates system so the position is a function of the radius \(r\) and the angle \(\theta\).

The radial component is defined by the unit vector \(\hat{\mathrm{e}}_r\) which defined the radial direction, always pointing from the origin to the particle.
The transverse component is defined by the unit vector \(\hat{\mathrm{e}}_\theta\) and is always normal to \(\hat{e}_r\).

Thus we get
\begin{align*}
  \vec{p} & = r\hat{\mathrm{e}}_r                                                                                                                       \\
  \vec{v} & = \dot{r}\hat{\mathrm{e}}_r + r\dot{\theta}\hat{\mathrm{e}}_\theta                                                                          \\
  \vec{a} & = \left( \ddot{r} - r\dot{\theta}^2 \right)\hat{\mathrm{e}}_r + \left( r\ddot{\theta} + 2\dot{r}\dot{\theta} \right)\hat{\mathrm{e}}_\theta
\end{align*}

\paragraph{Cylindrical coordinates}
In cylindrical coordinates, we simply have
\begin{align*}
  \vec{p} & = r\hat{\mathrm{e}}_r + z\kvec                                                                                                                              \\
  \vec{v} & = \dot{r}\hat{\mathrm{e}}_r + r\dot{\theta}\hat{\mathrm{e}}_\theta + \dot{z}\kvec                                                                           \\
  \vec{a} & = \left( \ddot{r} - r\dot{\theta}^2 \right)\hat{\mathrm{e}}_r + \left( r\ddot{\theta} + 2\dot{r}\dot{\theta} \right)\hat{\mathrm{e}}_\theta + \ddot{z}\kvec
\end{align*}


\section{Rigid bodies}
A rigid body can be analysed as a group of particles that have fixed relative position.

% 1cm = 10mm = 28pt = 1/2.54in
\begin{table}[h!] % Options: b (bottom), t (top), h (here), ! (insist)
  \caption{Difference particle vs. rigid body}
  % \label{tab:}
  \begin{center}
    \centering % Horizontal alignment of the table
    \begin{tabular}{ % Number of letter (l: left, c: center, r: right) = number of column
        c|c|c
      }
      % Visible row border: \hline (needed for each row)
      % Visible column border: | next to tabular declaration (needed for each column)
      % Column separation: &, row separation: \\

                     & Particle & Rigid body \\
      \hline
      Mass           & Yes      & Yes        \\
      \hline
      Dimension      & No       & Yes        \\
      \hline
      Translation    & Yes      & Yes        \\
      \hline
      Rotation       & No       & Yes        \\
      \hline
      Changes shapes & -        & No
    \end{tabular}
  \end{center}
\end{table}


\subsection{Translation and fixed-axis rotation of rigid bodies}
Rectilinear translation:
\begin{itemize}
  \item all particles have the same path
  \item the path is a straight line
  \item the orientation is maintained
\end{itemize}

Curvilinear translation:
\begin{itemize}
  \item all particle have the same path
  \item the path is a curved line
  \item the orientation is maintained
\end{itemize}

Fixed-axis rotation:
\begin{itemize}
  \item Has a single axis of rotation
  \item not all particle have the same path
  \item paths are circular about the axis of rotation
\end{itemize}


\subsubsection{Translation}
The relative position between to particles in the rigid body is constant.
This implies all points have the same velocities and accelerations, which equal to the velocity and acceleration of the body.


\subsubsection{Rotation about a fixed-axis}
Every particles have a constant radius between their position and the axis of rotation.
Also, the angle \(\phi\) between the position \(\vec{p}\) and the axis of rotation is constant.

The relation between arc length \(s\) and the angle \(\theta\) in a circle is \(s = R\theta = \normvec{p}\sin(\phi)\theta\) with R the radius from the axis of rotation to \(P\).

The velocity of the particle is \(v = R\dot{\theta} = \normvec{p}\sin(\phi)\dot{\theta}\).

Angular components:
\begin{align*}
  \vec{\omega} & = \vec{\dot{\theta}}                       \\
  \vec{\alpha} & = \vec{\dot{\omega}} = \vec{\ddot{\theta}}
\end{align*}

Linear components:
\begin{align*}
  \vec{v} & = \vec{\omega} \times \vec{p}                                                                                                                              \\
  \vec{a} & = \vec{\alpha} \times \vec{p} + \vec{\omega} \times \vec{v} = \vec{\alpha} \times \vec{p} + \vec{\omega} \times \left( \vec{\omega} \times \vec{p} \right)
\end{align*}
where \(\vec{\alpha} \times \vec{p}\) is the tangential acceleration and \(\vec{\omega} \times \vec{v}\) is the normal acceleration.

In the special case of planar movement, we have
\begin{align*}
  a_N & = r\omega^2 \text{ pointing to the center of rotation}     \\
  a_T & = r\alpha \text{ perpendicular to the normal acceleration}
\end{align*}


\subsubsection{Equations defining rotation}
By assuming constant acceleration \(\alpha\)
\begin{align*}
  \omega(t)      & = \alpha (t - t_0) + \omega_0                                   \\
  \theta(t)      & = \frac{1}{2} \alpha (t - t_0)^2 + \omega_0(t - t_0) + \theta_0 \\
  \omega(\theta) & = \sqrt{{\omega_0}^2 + 2\alpha(\theta - \theta_0)}
\end{align*}


\subsection{General plane motion - velocity}
The general plane motion can be break down into translational and rotational components.
The motion breakdown can be non-unique, it depends on the reference point that is chosen.

Breaking down plane motion in general:
\begin{enumerate}
  \item Pick a reference point \(A_1\)
  \item Translate body such that point \(A_1\) reaches its desired position \(A_2\)
  \item Rotate about \(A_2\) until point \(B_1'\) reach the desired position \(B_2\)
\end{enumerate}

For the general plane motion, the velocity between points on a rigid body is
\[
  \vec{v}_B = \vec{v}_A + \vec{\omega} \times \vec{p}_{B/A}
\]

The angular velocity \(\omega\) of a rigid body in plane motion is independent of the reference point, meaning the angular velocity of the entire body is \(\omega\) no matter where we are.
Moreover, any rigid bodies that are in contact with each other must have the same absolute velocity.


\subsection{Instantaneous center of rotation}
Another way to represent motion is using the instantaneous center of rotation (instantaneous center of zero velocity): there exists a point \(C\) where the generalized motion of a rigid body can be described purely as a rotation about that point.
This is a shortcut for calculating angular velocity of rigid body or linear velocities of points on the rigid body.

At \(C\), we would have
\[
  \omega = \frac{v_A}{\Normvec{AC}}
\]

A pure translation in this representation is represented with \(C\) at infinity, and an angular velocity \(\omega = 0\).
Instantaneous center of rotation representation is restricted ot rigid bodies since they do not deform, and it is only valid for calculating instantaneous velocities, but not the acceleration.
The path draw by the instantaneous center of rotation is called "space centrode".


\subsection{General plane motion - acceleration}
As with velocity, the general plane motion can be broken down into translation and rotation, but with acceleration, we use the normal and tangential acceleration because the path is a circle, and when working with circles, those acceleration are easier to use.

By choosing \(A\) as the reference point, we still use \(\vec{a}_B = \vec{a}_A + \vec{a}_{B/A}\), but now we have
\[
  \vec{a}_{B/A} = \vec{\alpha}_{AB} \times \vec{p}_{B/A} + \vec{\omega}_{AB} \times \left( \vec{\omega}_{AB} \times \vec{p}_{B/A} \right)
\]
meaning, in the end, we can get \(\vec{a}_B\) in general plane motion:
\[
  \vec{a}_B = \vec{a}_A + \vec{\alpha}_{AB} \times \vec{p}_{B/A} + \vec{\omega}_{AB} \times \left( \vec{\omega}_{AB} \times \vec{p}_{B/A} \right)
\]


\subsubsection{Bodies in contact}
Pin-connected bodies (rotary joints) must have the same absolute acceleration at the point of connection.
For meshed gears, they have the same tangential acceleration at the point of contact, but their normal acceleration is different.


\subsubsection{Four-bar linkages}
The purpose of four-bar linkages is to convert a uniform rotary motion into other cyclical motion:
\begin{itemize}
  \item ground/fixed link which does not move
  \item crank which fully rotates 360° (usually the input and sometimes the output)
  \item coupler which is the link joining other link (usually non-circular motion)
  \item Rocker which has a non 360° periodic motion
\end{itemize}


\subsection{Motion with respect to rotating frames}
\subsubsection{Frames of reference, coordinate frames}
A frame of reference is a coordinate frames that are not necessarily fixed.
In planar motion, the 2 types of coordinate frames are rectangular or polar frames, and in space motion, the 3 types of coordinate frames are rectangular, cylindrical and spherical frames.

Let \(\vec{Q}(t)\) a vector changing with time, \(R\) a rotating frame of angular velocity \(\Omega\) to which the unit vectors \(\ivec\), \(\jvec\) and \(\kvec\) are attached, and \(F\) a fixed frame.
Thus with respect to the rotating frame \(R\), we have:
\begin{align*}
  \frames{\vec{Q}}{R}                 & = Q_x \ivec + Q_y \jvec + Q_z \kvec                   \\
  \frames{\derivative{\vec{Q}}{t}}{R} & = \dot{Q_x} \ivec + \dot{Q_y} \jvec + \dot{Q_z} \kvec
\end{align*}

But with respect to the fixed frame \(F\):
\begin{align*}
  \frames{\vec{Q}}{F}                 & = Q_x \ivec + Q_y \jvec + Q_z \kvec                                                                                                       \\
  \frames{\derivative{\vec{Q}}{t}}{F} & = \dot{Q_x} \ivec + \dot{Q_y} \jvec + \dot{Q_z} \kvec + Q_x \derivative{\ivec}{t} + Q_y \derivative{\jvec}{t} + Q_z \derivative{\kvec}{t} \\
                                      & = \left( \dot{\vec{Q}} \right)_R + \vec{\Omega} \times \vec{Q}
\end{align*}


\subsubsection{Plane motion of a partical relative to a rotating frame}
Let a particle \(P\), a rotating frame \(R\) and a fixed frame \(F\) both centered at \(O\), then
\[
  \vec{v}_P = \vec{v}_O + \vec{\Omega} \times \vec{p}_{P/O} + \frames{\vec{v}_{P/O}}{R}
\]
\[
  \begin{array}{|l}
    \vec{v}_P [\si{\metre\per\second}] \text{: velocity of } P \text{ in } F                                            \\
    \vec{v}_O [\si{\metre\per\second}] \text{: velocity of } O \text{ in } F                                            \\
    \vec{\Omega} [\si{\radian\per\second}] \text{: angular velocity of } R                                              \\
    \vec{p}_{P/O} [\si{\metre}] \text{: position of } P \text{ with respect to } O                                      \\
    \frames{\vec{v}_{P/O}}{R} [\si{\metre\per\second}] \text{: velocity of } P \text{ with respect to } O \text{ in } R \\
  \end{array}
\]


\subsubsection{Acceleration in rotating frames}
Let a particle \(P\), a rotating frame \(R\) and a fixed frame \(F\) both centered at \(O\), then
\begin{multline*}
  \vec{a}_P = \vec{a}_O + \dot{\vec{\Omega}} \times \vec{p}_{P/O} + \vec{\Omega} \times \left( \vec{\Omega} \times \vec{p}_{P/O} \right) \\
  + 2\vec{\Omega} \times \frames{\vec{v}_{P/O}}{R} + \frames{\vec{a}_{P/O}}{R}
\end{multline*}
\[
  \begin{array}{|l}
    \vec{a}_P [\si{\metre\per\second}] \text{: acceleration of } P \text{ in } F                                        \\
    \vec{a}_O [\si{\metre\per\second}] \text{: acceleration of } O \text{ in } F                                        \\
    \vec{\Omega} [\si{\radian\per\second}] \text{: angular velocity of } R                                              \\
    \vec{p}_{P/O} [\si{\metre}] \text{: position of } P \text{ with respect to } O                                      \\
    \frames{\vec{v}_{P/O}}{R} [\si{\metre\per\second}] \text{: velocity of } P \text{ with respect to } O \text{ in } R \\
    \frames{\vec{a}_{P/O}}{R} [\si{\metre\per\second}] \text{: acceleration of } P                                      \\
    \text{ with respect to } O \text{ in } R                                                                            \\
  \end{array}
\]


\subsubsection{Coriolis acceleration}
Coriolis acceleration \(\vec{a}_\mathrm{coriolis}\) is due to the change of velocity when a particle change its radius of motion:
\[
  \vec{a}_\mathrm{coriolis} = 2\vec{\Omega} \times \frames{\vec{v}_{P/O}}{R}
\]
\[
  \begin{array}{|l}
    \vec{a}_\mathrm{coriolis} [\si{\metre\per\second}] \text{: coriolis acceleration}                                   \\
    \vec{\Omega} [\si{\radian\per\second}] \text{: angular velocity of } R                                              \\
    \frames{\vec{v}_{P/O}}{R} [\si{\metre\per\second}] \text{: velocity of } P \text{ with respect to } O \text{ in } R \\
  \end{array}
\]

\part{Kinetics: Newton's Second Law in general plane motion}
\section{Particles}
\subsection{Newton's second law and linear momentum}
Newton's second law states that \(\sum{\vec{F}} = m \vec{a}\).
In statics, we had \(\vec{a} = \vec{0}\) and thus Newton's second law became Newton's first law.

By defining linear momentum as \(\vec{L} = m \vec{v}\), then Newton's second law becomes  \(\sum{\vec{F}} = \dot{\vec{L}}\).


\subsection{Angular momentum and orbital motion}
The angular momentum of a rotating object measures how hard it is to stop such rotating object when it is in motion.
We define angular momentum as
\[
  \vec{H}_O = \vec{p}_{P/O} \times m\vec{v}_{P/O}
\]
\[
  \begin{array}{|l}
    \vec{H}_O [\si{unit}] \text{: angular momentum at } O                                     \\
    \vec{p}_{P/O} [\si{\metre}] \text{: position of } P \text{ with respect to } O            \\
    m [\si{\kilogram}] \text{: mass of the rotating object}                                   \\
    \vec{v}_{P/O} [\si{\metre\per\second}] \text{: velocity of } P \text{ with respect to } O \\
  \end{array}
\]

% Comparison table


\subsubsection{Components of angular momentum}
In cartesian coordinates, we have \(\vec{H} = \vec{p} \times m \vec{v}\), while in polar coordinates, the angular momentum is \(H = m r^2 \dot{\theta}\) in the direction perpendicular to the plane.


\subsubsection{Newton's second law for angular motion}
Newton's second law for angular motion is \(\sum{\vec{M}_O} = \dot{\vec{H}}_O\), which is very similar to Newton's second law for linear motion \(\sum{\vec{F}} = \dot{\vec{L}}\).


\subsubsection{Central force}
A net force \(\vec{F}\) applied on a particle \(P\) directed towards point is called a central force.
This force does not change the angular momentum because \(\vec{F}\) does not create any moment around \(O\).


\subsubsection{Newton's law of gravitation}
Newton's law of universal gravitation states that:
Two particles of masses \(M\) and \(m\) at a distance \(r\) from each other have a mutual attraction  of equal and opposite forces \(\vec{F}\) and \(-\vec{F}\) directed along the line joining the particles.
Thus, we have
\[
  F = G \frac{m_1 m_2}{r^2}
\]
\[
  \begin{array}{|l}
    F [\si{\newton}] \text{: gravitation force}                      \\
    G = \SI{66.73 e-12}{\metre\squared\per\kilogram\per\sec\squared} \\
    m [\si{\kilogram}] \text{: mass of the objects}                  \\
    r [\si{\metre}] \text{: distance between the objects}
  \end{array}
\]

Using this equation, we can find that on Earth, \(GM = gR^2\) where \(R = \SI{6.373 e+6}{\metre}\) is the Earth radius and \(g = \SI{9.81}{\metre\per\second\squared}\).


\section{Rigid bodies}
\subsection{Kinetics of rigid bodies}
Now, we will take the body shape into account when analyzing effect of forces on a mass.
\(G\) is defined as the center of mass, and we will use the following equations:
\begin{align*}
  \sum{\vec{F}}   & = \dot{\vec{L}} = m\vec{a}_G \\
  \sum{\vec{M}_G} & = \dot{\vec{H}}_G
\end{align*}


\subsubsection{Angular momentum of a rigid body in plane motion}
The angular momentum of a rigid body in plane motion can be defined using its mass moment of inertia \(I_G\):
\[
  \vec{H}_G = I_G \vec{\omega}
\]
which implies that in plane motion, \(\sum{\vec{M}_G} = I_G \vec{\alpha}\).


\paragraph{Mass moment of inertia \(I\)}
For rigid bodies, the shape and reference point influences the moment of inertia.

For any arbitrary point \(P\), the parallel axis theorem states \(I_P = I_G + md^2\) where \(d\) is the perpendicular (shortest) distance between \(P\) and \(G\).


\paragraph{Radius of gyration}
The radius of gyration is defined as \(k = \sqrt{\frac{I_G}{m}}\) and is the distance at which a point with same mass would give the same moment of inertia \(I = k^2 m\).

The parallel axis theorem in this case becomes \({k_P}^2 = {k_G}^2 + d^2\).


\subsubsection{Kinetics of rigid bodies in plane motion}
Here, only motion \(xy\)-plane and rotation about \(z\)-axis are considered.
The motion of the rigid body is completely defined by the resultant force and resultant moment about \(G\) acting on the body.
Thus, in plane motion, we have the following equations at the center of gravity \(G\):
\begin{align*}
   & \text{In } \vec{x}:\sum{F_x} = m a_{G,x}  \\
   & \text{In } \vec{y}:\sum{F_y} = m a_{G,y}  \\
   & \text{In } \vec{z}:\sum{M_G} = I_G \alpha
\end{align*}


\subsubsection{Using other points of reference}
We can choose any arbitrary point \(P\) of reference to sum moments.
About \(G\): \(\sum{\vec{M}_G} = I_G \vec{\alpha}\)
About \(P\): \(\sum{\vec{M}_P} = \sum{\vec{M}_G} + \vec{p}_{G/P} \times m\vec{a}_G\)


\subsubsection{Static friction and kinetic friction}
Friction is a resistive force that opposes motion and it depends on the normal force \(\vec{N}\).
\begin{description}
  \item[Static friction:] the object is not moving (\(v = 0\)): \(F_s \leqslant \mu_s N\) and the direction is opposite to the applied force.
  \item[Kinetic friction:] the object is moving:  \(F_k = \mu_k N\) and the direction is opposite to velocity.
\end{description}


\paragraph{Rolling motion}
\begin{description}
  \item[Rolling, no sliding:] \(F_s \leqslant \mu_s N\) and \(a_G = r\alpha\)
  \item[Rolling, sliding impending:] \(F_s = \mu_s N\) and \(a_G = r\alpha\)
  \item[Rolling, sliding:] \(F_k = \mu_k N\) and \(a_G\), \(\alpha\) independent
\end{description}


\subsection{Constrained plane motion}
\paragraph{Unbalanced disk}
For an unbalanced disk, the center of mass \(G\) and the centroid \(O\) are not at the same position, thus:
\[
  \vec{a}_G = \vec{a}_O + \vec{\alpha}_{OG} \times \vec{p}_{G/O} + \vec{\omega}_{OG} \times \left( \vec{\omega}_{OG} \times \vec{p}_{G/O} \right)
\]


\part{Kinetics: energy method in general plane motion}
\section{Particles}
\subsection{Work and energy}
Work \(W\) is the energy transferred to or from an object via the application of force along a displacement.
It is a scalar: positive when the force is in the same the direction of motion and negative when the force is in the opposite direction of motion.
\[
  W_{1 \to 2} = \int_1^2{\vec{F} \bullet \diffint{\vec{p}}}
\]

\subsubsection{Principle of work and energy}
In an absolute reference frame, using Newton's Second Law, we can get that the work of a tangential force along a displacement of a mass is:
\[
  W_{1 \to 2} = \frac{1}{2}m \left( {v_2}^2 - {v_1}^2 \right) = E_\mathrm{kinetic,2} - E_\mathrm{kinetic,1}
\]

\subsubsection{Power}
Power is defined as the time rate at which work is done:
\[
  P = \dot{W} = \derivative{W}{t} = \vec{F} \bullet \vec{v}
\]

\begin{remark}
  1 hp = 550 \(\mathrm{ft} \cdot \mathrm{lb} \cdot \mathrm{s}^{-1}\) = 550 \(\times\) 1.356 \(\si{\watt}\)
\end{remark}


\subsubsection{Mechanical efficiency}
The mechanical efficiency is defined as:
\[
  \eta = \frac{\dot{W}_\mathrm{out}}{\dot{W}_\mathrm{in}}
\]


\subsection{Conservation of energy}
Energy can be transformed from one type to another, but energy is never created nor lost.
It is a property of the state of an object at a particular point in time and space.
When changing energy states, work occurs.


\subsubsection{Potential energy}
The potential energy is "stored" energy that has the potential to be converted back into another type of energy (usually kinetic energy).

\paragraph{Gravitational potential energy}
The gravitational potential energy \(E_\mathrm{g}\) is energy stored from the elevation of a mass:
\[
  E_\mathrm{g} = mgz
\]
\[
  \begin{array}{|l}
    E_\mathrm{g} [\si{\joule}] \text{: gravitational potential energy}     \\
    m [\si{\kilogram}] \text{: object mass}                                \\
    g [\si{\meter\per\second\squared}] \text{: gravitational acceleration} \\
    z [\si{\metre}] \text{: elevation from the ground}                     \\
  \end{array}
\]


\paragraph{Elastic potential energy}
The elastic potential energy \(E_\mathrm{e}\) is defined as:
\[
  E_\mathrm{e} = \frac{1}{2}kx^2
\]
\[
  \begin{array}{|l}
    E_\mathrm{e} [\si{\joule}] \text{: elastic potential energy}       \\
    k [\si{\newton\per\meter}] \text{: spring constant}                \\
    x [\si{\meter\per\second\squared}] \text{: spring deflection from} \\
    \text{the non-deformed position}                                   \\
  \end{array}
\]

Thus the work of the spring is:
\[
  W_\mathrm{1 \to 2, e} = \frac{1}{2} k \left( {x_1}^2 - {x_2}^2 \right) = - \frac{1}{2} (F_1 + F_2) (x_2 - x_1)
\]
where \(F = kx\)

\subsubsection{Conservative forces}
A conservative force is a force with the property that the total work done in moving a particle between two points is independent of the path taken.

\begin{example}
  Gravity, elastic and magnetic forces are conservative while friction is a non-conservative force.
\end{example}


\subsubsection{Principle of conservation of energy}
Since energy can only be converted, then the energy between two states must be conserved (without friction):
\[
  E_\mathrm{k,1} + E_\mathrm{g,1} + E_\mathrm{e,1} = E_\mathrm{k,2} + E_\mathrm{g,2} + E_\mathrm{e,2}
\]
\[
  \begin{array}{|l}
    E_\mathrm{k} [\si{\joule}] \text{: kinetic energy}               \\
    E_\mathrm{g} [\si{\joule}] \text{: gravitation potential energy} \\
    E_\mathrm{e} [\si{\joule}] \text{: elastic potential energy}     \\
  \end{array}
\]

\subsubsection{Energy loss from friction}
Friction forces being non-conservative, the total mechanical energy is not conserved, thus a term needs to be added in the equation:
\[
  E_\mathrm{k,1} + E_\mathrm{g,1} + E_\mathrm{e,1} + W_\mathrm{1 \to 2, nc} = E_\mathrm{k,2} + E_\mathrm{g,2} + E_\mathrm{e,2}
\]
where \(W_\mathrm{1 \to 2, nc}\) is the work done by the non-conservative forces.


\section{Rigid bodies}
\subsection{Principle of work and energy}
The principle of work and energy is the same for rigid bodies as for a particle:
\[
  E_\mathrm{kinetic,1} + W_{1 \to 2} = E_\mathrm{kinetic,2}
\]

\subsection{Work of forces acting on a rigid body}
For a displacement force, the work of forces acting on a rigid body is:
\[
  W_{1 \to 2,\mathrm{displacement}} = \int_1^2{\vec{F} \bullet \diffint{\vec{p}}}
\]

However, now we also have to account for moments as we are dealing with rigid bodies:
\[
  W_{1 \to 2,\mathrm{rotation}} = \int_1^2{\normvec{M} \diffint{\theta}}
\]

\begin{remark}
  Some forces do no work: normal forces, support forces, friction force in rolling without sliding.
\end{remark}

\subsection{Kinetic energy of a rigid body in plane motion}
The kinetic energy of a rigid body in general plane motion is broken down into two components (the translation part and the rotation part):
\[
  E_\mathrm{k} = \frac{1}{2} m {v_G}^2 + \frac{1}{2} I_G \omega^2
\]


\subsection{Conservation of energy}
The principle of conservation of energy for rigid bodies is the same as for particles.
Without friction:
\[
  E_\mathrm{k,1} + E_\mathrm{g,1} + E_\mathrm{e,1} = E_\mathrm{k,2} + E_\mathrm{g,2} + E_\mathrm{e,2}
\]
\[
  \begin{array}{|l}
    E_\mathrm{k} [\si{\joule}] \text{: kinetic energy}               \\
    E_\mathrm{g} [\si{\joule}] \text{: gravitation potential energy} \\
    E_\mathrm{e} [\si{\joule}] \text{: elastic potential energy}     \\
  \end{array}
\]

With friction:
\[
  E_\mathrm{k,1} + E_\mathrm{g,1} + E_\mathrm{e,1} + W_\mathrm{1 \to 2, nc} = E_\mathrm{k,2} + E_\mathrm{g,2} + E_\mathrm{e,2}
\]
where \(W_\mathrm{1 \to 2, nc}\) is the work done by the non-conservative forces.


\subsection{Power}
The power of forces in general plane motion on a rigid body is divided into two components, linear and angular:
\begin{align*}
  \dot{W}_\mathrm{linear}  & = \vec{F} \bullet \vec{v}_G \\
  \dot{W}_\mathrm{angular} & = M_G \omega
\end{align*}


\part{Kinetics: momentum method in general plane motion}
\section{Particles}
\subsection{Impulse and momentum of particles}
Recall the equations of momentum:
\begin{align*}
  \sum{\vec{F}}   & = \dot{\vec{L}} = \derivative{}{t}[m\vec{v}]                  \\
  \sum{\vec{M}_G} & = \dot{\vec{H}}_G = \derivative{}{t}[\vec{p} \times m\vec{v}]
\end{align*}

An impulse quantifies the overall effect of a force acting over time:
\[
  \vec{J}_{1 \to 2} = \int_1^2{\vec{F} \diffint{t}}
\]
Thus the linear momentum equation becomes:
\[
  m\vec{v}_1 + \vec{J}_{1 \to 2} = m\vec{v}_2
\]

Impulse and momentum method is used with impacts and collisions.


\subsubsection{Conservation of linear momentum}
In the absence of external impulses, total momentum of particles is conserved:
\[
  m\vec{v}_1 = m\vec{v}_2
\]


\subsubsection{Impulsive motion and impulsive forces}
In certain cases, impulses and changes in momentum occur in very short time frames where \(\Delta t \approx 0\).
Thus slow acting forces such as weight or springs can be ignored.
Forces acting on a particle during a very short time interval but large enough to produce a definite change in momentum are called \emph{impulsive forces} that result in \emph{impulsive motion}.



\subsection{Impacts of particles}
An impact is defined as two bodies that collide with each other with significant force acting on each other.

\begin{description}
  \item[Line of impact:] common normal to the surfaces in contact during impact
  \item[Central impact:] the mass centers of the two bodies lie on the line of impact
  \item[Eccentric impact:] the mass centers of the two bodies does not lie on the line of impact
  \item[Oblique impact:] impact for which one or both of the bodies move along a line other than the line of impact
\end{description}


\subsubsection{Direct central impact}
There are 3 stages of impacts:
\begin{enumerate}
  \item Before impact, the distance between the object is narrowing
  \item During impact
        \begin{oldenumerate}
          \item Objects come into contact
          \item Period of deformation
          \item Achieve maximum deformation and the objects travel at the same speed \(\vec{v}\)
          \item Period of restitution
          \item Objects may or may not separate
        \end{oldenumerate}
  \item After impact, the particles return to original shape or stay permanently deformed and the linear momentum is conserved
\end{enumerate}


\paragraph{During impact - period of deformation}
By isolating one of the two colliding objects \(A\), the impact \(B\) is a force over time applied to \(A\).
Thus, we will use the impulse and momentum methods to analyse \(A\):
\[
  m_A - \int{\vec{P} \diffint{t}} = m_A \vec{u}
\]


\paragraph{During impact - period of restitution}
The restitution is the exact opposite of the deformation, thus the equation becomes:
\[
  m_A \vec{u} - \int{\vec{R} \diffint{t}} = m_A \vec{v}_{A,2}
\]


\paragraph{Coefficient of resitution \(e\)}
Usually the force during the period of deformation \(\vec{P}\) and the force of restitution \(\vec{R}\) are not the same, thus we can define the coefficient of restitution \(e\) as:
\[
  \begin{split}
    e & = \frac{\int{R \diffint{t}}}{\int{P \diffint{t}}} = \frac{u - v_{A,2}}{v_{A,1} - u} \\
    & = \frac{v_{B,2} - v_{A,2}}{v_{A,1} - v_{B,1}}
  \end{split}
\]

By studying the equation of the coefficient of restitution, it measures the changes in relative velocity after impact.
The coefficient of restitution depends on the two materials involved, impact velocity, and shape and size of objects.
\begin{description}
  \item[\(e = 0\):] perfectly plastic impact, there is zero relative velocity between \(A\) and \(B\), meaning they are moving at the same speed
  \item[\(e = 1\):] perfectly elastic impact, the relative velocity is unchanged.
\end{description}

\paragraph{Impacts and energy}
Linear momentum is always conserved since there are no external impulse, but kinetic energy is only conserved if the impact is perfectly elastic.


\subsubsection{Oblique central impact}
In oblique central impact, the mass centers of the two bodies lie on the line of impact, but the bodies move along a line other than the line of impact.
In order to study this more general case of impacts, we define the tangential and normal coordinate system according to surface normal and the line of impact.

Since the impulses are along the line of impact, there is no change in the tangential momenta of particle.
Then looking at the normal component, we can analyze it as a direct central impact.
Thus:
\begin{itemize}
  \item Tangential components are constant and do not change from impact
        \begin{align*}
          v_{A,t,1} = v_{A,t,2} &  & \mathrm{and} &  & v_{B,t,1} = v_{B,t,2}
        \end{align*}
  \item Total momentum along normal direction is conserved
        \[
          m_A v_{A,n,1} + m_B v_{B,n,1} = m_A v_{A,n,2} + m_B v_{B,n,2}
        \]
  \item The coefficient of restitution can be used along normal direction
        \[
          e = \frac{v_{B,n,2} - v_{A,n,2}}{v_{A,n,1} - v_{B,n,1}}
        \]
\end{itemize}


\subsubsection{Constrained oblique central impact}
When there is a constrained system, the trick is to look at different systems and coordinate axis in order to find all the equations needed to solve the problem.


\section{Rigid bodies}
\subsection{Momentum method for a rigid body}
\subsubsection{Principle of impulse and momentum}
The principle of impulse and momentum for rigid bodies needs to account fo the angular momentum, meaning in general plane motion at the center of gravity \(G\):
\begin{align*}
  \text{Linear: }  & m\vec{v}_{G,1} + \sum{\int_1^2{\vec{F} \diffint{t}}} = m\vec{v}_{G,2}         \\
  \text{Angular: } & I_G\vec{\omega}_1 + \sum{\int_1^2{\vec{M}_G \diffint{t}}} = I_G\vec{\omega}_2
\end{align*}

\paragraph{Non-centroidal motion}
For a non-centroidal motion about point \(P\), the angular momentum equation becomes:
\[
  I_G\omega_1 + m\vec{v}_{G,1}d_1 + \sum{\int_1^2{\vec{M}_P \diffint{t}}} = I_G\omega_2 + m\vec{v}_{G,2}d_2
\]

If we have \emph{fixed axis rotation} about point \(P\), the linear and angular velocity can be related (\(\vec{v} = \omega d\)) and thus:
\[
  I_P\omega_1 + \sum{\int_1^2{\vec{M}_P \diffint{t}}} = I_P\omega_2
\]


\subsubsection{Conservation of angular momentum}
If there are no external impulses acting on the system, both linear and angular momenta are conserved.
Also, if an impulse goes through the reference point, and there are not impulse moment, then linear momentum is not conserved, but angular momentum is (case for central forces).


\subsection{Eccentric impacts for rigid bodies}
In an eccentric impact, the mass centers are not located on the line of impact.
During the max deformation phase, the two rigid bodies have the same velocity along the line of action at their contact points:
\[
  \vec{u}_{A,n} = \vec{u}_{B,n}
\]

\subsubsection{Impulse and momentum analysis of eccentric impact}
We split the analysis into 4 cases: linear and angular, and deformation and restitution.
\begin{align*}
  \text{Linear, deformation: }  & m v_{G,n,1} - \int{P_n \diffint{t}} =  m u_{G,n}     \\
  \text{Linear, restitution: }  & m u_{G,n} - \int{R_n \diffint{t}} =  m v_{G,n,2}     \\
  \text{Angular, deformation: } & I_G \omega_1 - r\int{P_n \diffint{t}} =  I_G\omega^* \\
  \text{Angular, restitution: } & I_G \omega^* - r\int{P_n \diffint{t}} =  I_G\omega_2
\end{align*}

The coefficient of restitution for eccentric impacts and fixed axis eccentric impacts is the same as for central impacts:
\[
  e = \frac{v_{B,n,2} - v_{A,n,2}}{v_{A,n,1} - v_{B,n,1}}
\]
where \(A\) and \(B\) are the contact points along the line of impact.


\subsubsection{Summary for impacts}
\begin{enumerate}
  \item Separate impacts into normal (along the line of impact) and tangential (perpendicular to the line of impact) components
  \item Coefficient of restitution generalizes to point of impacts in the surface normal direction
  \item Momentum is conserved along directions without external impulses
\end{enumerate}


\part{Kinetics: Mechanical vibrations}
\section{Particles}
A mechanical vibration is the motion of a particle or body that oscillates about a position of equilibrium.

Typically, vibrations are undesirable and causes stress and energy losses which may lead to part failure.
However, not all vibrations are bad: ultrasound probes, mixers, massage tools, etc.
Also, electricity can be generated using vibrations.

Systems with vibrations can often be represented by mass-spring or mass-spring-damper systems.

Forces that are trying to restore the system to the equilibrium point are restorative forces (gravity in a pendulum, mass-spring system).


\subsection{Terminology}
Vibrations are typically cyclical, meaning they can be defined by:
\begin{description}
  \item[Period \(T\):] time interval to complete a full cycle of motion
  \item[Frequency \(f\):] number of cycles per unit time, \(f = \frac{1}{T}\)
  \item[Amplitude:] max displacement from equilibrium
\end{description}

Then there are different type of vibrations:
\begin{description}
  \item[Undamped vibration:] ignore the effect of frictions, amplitude does not decrease over time
  \item[Damped vibration:] amplitude decrease over time
  \item[Free vibration:] no additional external force are applied on the system
  \item[Forced vibration:] periodic external force applied
\end{description}


\subsection{Vibration without damping}
For a free undamped motion of a mass-spring system of spring constant \(k\) and mass \(m\), using Hooke's law (\(F = -ks\), where \(k \in \set{R}{*}{+}\)) and Newton's Second law (\(F = ma\)), the following equation of motion is obtained:
\[
  \ddot{x} + \omega^2 x = 0
\]
where \(\omega^2 = \frac{k}{m}\).
This equation describes a simple harmonic motion, also called free undamped motion.

The solution of this differential equation is
\[
  x(t) = c_1\cos{\omega t} + c_2\sin{\omega t}
\]
where \(c_1 = \) and \(c_2 = \) are the initial conditions.

An alternative form of \(x(t)\) is used to find the amplitude of vibration \(x_\mathrm{max}\) and the phase angle \(\phi\) of the system:
\[
  x(t) = x_\mathrm{max}\sin(\omega t + \phi) = x_\mathrm{max}\cos(\omega t + \phi - \frac{\pi}{2})
\]
where \(x_\mathrm{max} = \sqrt{{x_0}^2 + \left( \frac{v_0}{\omega} \right)^2 }\) and \(\phi = \arctan\left( \frac{x_0}{c_2} \right)\).
With this equation, we can find that:
\begin{align*}
  v_\mathrm{max} & = x_\mathrm{max} \omega   \\
  a_\mathrm{max} & = x_\mathrm{max} \omega^2
\end{align*}

The period of vibration is
\[
  \tau = \frac{2\pi}{\omega}
\]
and the frequency is
\[
  f = \frac{1}{T} = \frac{\omega}{2\pi}
\]

\begin{remark}
  The natural frequency \(\omega\), the period \(\tau\) and the frequency \(f\) is independent of the weight and the initial conditions.
\end{remark}


\begin{example}
  A pendulum can be modeled using vibrations (assuming the angle is small enough) and we get that \(\omega = \frac{l}{g}\) where \(l\) is the length of the pendulum and \(g\) is the gravity constant.
\end{example}


\subsubsection{Multiple spring systems}
If the springs are in parallel, we have
\[
  k_{\mathrm{equivalent}} = k_1 + k_2 + \cdots
\]
If the springs are in series, we have
\[
  \frac{1}{k_{\mathrm{equivalent}}} = \frac{1}{k_1}+ \frac{1}{k_2} + \cdots
\]


\subsubsection{Importance of natural frequency \(\omega\)}
The natural frequency is the frequency at which resonance occurs for forced vibrating systems, this can leads to bridges moving uncontrollably.
Transmissibility plot can be made in which the amplitude is a function of the frequency and at the natural frequency, without damping, the amplitude can go up to infinity.


\section{Rigid bodies}
\subsection{Free vibrations of rigid bodies}
Simple harmonic motion is expressed in linear and angular motion, where \(\omega\) is the natural frequency of the rigid body:
\begin{description}
  \item[Linear:] \(\ddot{x} + \omega^2 x = 0\)
  \item[Angular:] \(\ddot{\theta} + \omega^2 \theta = 0\)
\end{description}

In order to study vibration of a rigid body, isolate the rigid body, analyse using kinematics the motion for a small displacement and solve for the moment around the center of rotation.



\end{document}
