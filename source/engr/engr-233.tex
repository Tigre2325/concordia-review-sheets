\documentclass[10pt, twocolumn]{article}

%%%%%%%%%%%%%%%%%%%%%%%%%%%%%%%%%%%%%%%%%%%%%%%%%%%%%%%%%%%%%%%%%%%%%%%%%%%%%%%
%%%% Cover page
\title{ENGR 233: Applied advanced calculus}
\date{\today}
\author{Anthony Bourboujas}

\makeatletter
\let\Title\@title
\let\Author\@author
\let\Date\@date
\makeatother

%%%%%%%%%%%%%%%%%%%%%%%%%%%%%%%%%%%%%%%%%%%%%%%%%%%%%%%%%%%%%%%%%%%%%%%%%%%%%%%
%%%% Preamble
%%%%%%%%%%%%%%%%%%%%%%%%%%%%%%%%%%%%%%%%%%%%%%%%%%%%%%%%%%%%%%%%%%%%%%%%%%%%%%%
%%%% Packages
\usepackage[utf8x]{inputenc} % Accept different input encodings
\usepackage[T1]{fontenc} % Standard package for selecting font encodings
\usepackage{lmodern} % Font name; classic: lmodern
\usepackage[english]{babel} % Multilingual support for LaTeX
% \usepackage{abstract} % Control the typesetting of the abstract environment
\usepackage{amsmath} % AMS mathematical facilities for LaTeX
\usepackage{amssymb} % TeX fonts from the American Mathematical Society
\usepackage{amsthm} % Typesetting theorems (AMS style)
\usepackage{array} % Extending the array and tabular environments
\usepackage{cellspace} % Ensure minimal spacing for table cells
\usepackage{chemformula} % Command for typesetting chemical formulas and reactions
% \usepackage{colortbl} % Add colour to LaTeX tables
% \usepackage{comment} % Selectively include/exclude portions of text
% \usepackage[en-US,showdow]{datetime2} % Formats for dates, times and time zones
\usepackage{enumitem} % Control layout of itemize, enumerate, description
\usepackage{esint} % Extended set of integrals for Computer Modern
\usepackage{graphicx} % Enhanced support for graphics
% \usepackage{lipsum} % Easy access to the Lorem Ipsum dummy text
\usepackage{mathrsfs} % Support for using RSFS fonts in maths
\usepackage{moreverb} % Extended verbatim
\usepackage{multicol} % Intermix single and multiple columns
\usepackage{multirow} % Create tabular cells spanning multiple rows
\usepackage[scr]{rsfso} % A mathematical calligraphic font based on rsfs
\usepackage{setspace} % Set space between lines
% \usepackage{soul} % Hyphenation for letterspacing, underlining, and more
\usepackage{threeparttable} % Tables with captions and notes all the same width
\usepackage{wrapfig} % Produces figures which text can flow around
\usepackage{xcolor} % Driver-independent color extensions for LaTeX

%%%%%%%%%%%%%%%%%%%%%%%%%%%%%%%%%%%%%%%%%%%%%%%%%%%%%%%%%%%%%%%%%%%%%%%%%%%%%%%
%%%% Lengths
% 1cm = 10mm = 28pt = 1/2.54in
% 1ex = height of a lowercase 'x' in the current font
% 1em = width of an uppercase 'M' in the current font

%%%% Spacing in math mode
% \!                         = -3/18em
% \,                         = 3/18em
% \:                         = 4/18em
% \;                         = 5/18em
% \ (space after backslash!) = space in normal text
% \quad                      = 1em
% \qquad                     = 2em

% \setlength{\baselineskip}{1em} % Vertical distance between lines in a paragraph
% \renewcommand{\baselinestretch}{1.0} % A factor multiplying \baslineskip
\setlength{\columnsep}{0.75cm} % Distance between columns
% \setlength{\columnwidth}{} % The width of a column
\setlength{\columnseprule}{1pt} % The width of the vertical ruler between columns
% \setlength{\evensidemargin}{} % Margin of even pages, commonly used in two-sided documents such as books
% \setlength{\linewidth}{} % Width of the line in the current environment.
% \setlength{\oddsidemargin}{} % Margin of odd pages, commonly used in two-sided documents such as books
% \setlength{\paperwidth}{} % Width of the page
% \setlength{\paperheight}{} % Height of the page
\setlength{\parindent}{0cm} % Paragraph indentation
\setlength{\parskip}{6pt} % Vertical space between paragraphs
% \setlength{\tabcolsep}{} % Separation between columns in a table (tabular environment)
% \setlength{\textheight}{} % Height of the text area in the page
% \setlength{\textwidth}{} % Width of the text area in the page
% \setlength{\topmargin}{} % Length of the top margin
\setlist{
  %%%% Vertical spacing
  topsep = 0pt,
  partopsep = 0pt,
  parsep = 0pt,
  itemsep = 0pt,
  %%%% Horizontal spacing
  leftmargin = 0.5cm,
  rightmargin = 0cm,
  % listparindent = 0cm,
  % labelwidth = 0cm,
  % labelsep = 0cm,
  % itemindent = 0cm
}
\addtolength{\cellspacetoplimit}{2pt}
\addtolength{\cellspacebottomlimit}{2pt}

%%%%%%%%%%%%%%%%%%%%%%%%%%%%%%%%%%%%%%%%%%%%%%%%%%%%%%%%%%%%%%%%%%%%%%%%%%%%%%%
%%%% Page layout
\usepackage{layout} % View the layout of a document
\usepackage{geometry} % Flexible and complete interface to document dimensions
% 1cm = 10mm = 28pt = 1/2.54in
% ex = height of a lowercase 'x' in the current font
% em = width of an uppercase 'M' in the current font
\geometry{
  a4paper,
  top         = 1cm,
  bottom      = 1cm,
  left        = 1.5cm,
  right       = 1.5cm,
  includehead = true,
  includefoot = true,
  landscape   = false, % Paper orientation
  twoside     = false,
}
% \geometry{showframe} % Show paper outline for the text area and page

%%%%%%%%%%%%%%%%%%%%%%%%%%%%%%%%%%%%%%%%%%%%%%%%%%%%%%%%%%%%%%%%%%%%%%%%%%%%%%%
%%%% Header and footer style
\usepackage{fancyhdr} % Extensive control of page headers and footers in LaTeX
\pagestyle{fancy}
% Options: \leftmark (chapter title), \rightmark(section title), \thepage (page number), \thechapter(chapter number), \thesection (section number)
\lhead{\Title}
\chead{}
\rhead{}
\lfoot{}
\cfoot{\thepage}
\rfoot{}

%%%%%%%%%%%%%%%%%%%%%%%%%%%%%%%%%%%%%%%%%%%%%%%%%%%%%%%%%%%%%%%%%%%%%%%%%%%%%%%
%%%% URL insertion settings
\usepackage{hyperref} % Extensive support for hypertext in LaTeX
\definecolor{black}{RGB}{0, 0, 0} % rgb(0, 0, 0)
\definecolor{blue}{RGB}{0, 0, 255} % rgb(0, 0, 255)
\hypersetup{
  % unicode            = true,
  pdftitle           = {\Title},
  pdfauthor          = {\Author},
  % pdfsubject       = {},
  %%%% Reference
  % bookmarks          = true,
  bookmarksnumbered  = true,
  bookmarksopen      = true, % Open the bookmarks
  bookmarksopenlevel = 2, % Open until 1 level (section)
  %%%% Bookmarks
  breaklinks         = true,
  pdfborder          = {0 0 0},
  % backref            = true, % Add links into bibliography
  % pagebackref        = true,
  % hyperindex         = true, % Add links into index
  %%%% Color
  colorlinks         = true,
  linkcolor          = black, % Internal links color
  citecolor          = black,
  urlcolor           = blue, % Hyperlinks color
  filecolor          = black,
}

\usepackage{varioref} % Intelligent page reference
\usepackage[capitalise,noabbrev]{cleveref}
\usepackage{prettyref} % Make label references "self-identity" with \prettyref{#1}
\newrefformat{cha}{chapter \textbf{\nameref{#1}} \vpageref{#1}} % {chapter \textbf{\nameref{#1}} on page \pageref{#1}}
\newrefformat{sec}{section \textbf{\nameref{#1}} \vpageref{#1}} % {section \textbf{\nameref{#1}} on page \pageref{#1}}
% \newrefformat{fig}{\vref{#1}} % {Figure \ref{#1} on page \pageref{#1}}
% \newrefformat{tab}{\vref{#1}} % {Table \ref{#1} on page \pageref{#1}}
% \newrefformat{eqn}{\vref{#1}}
% \newrefformat{lis}{\emph{\nameref{#1}} \vpageref{#1}}

%%%%%%%%%%%%%%%%%%%%%%%%%%%%%%%%%%%%%%%%%%%%%%%%%%%%%%%%%%%%%%%%%%%%%%%%%%%%%%%
%%%% Physics units settings
% Dependencies
\usepackage{booktabs} % Publication quality tables in LaTeX
\usepackage{caption} % Customizing captions in floating environments
\usepackage{helvet} % Load Helvetica, scaled
\usepackage{cancel} % Place lines through maths formulae

\usepackage{siunitx} % A comprehensive (SI) units package
\sisetup{
  exponent-product     = \cdot, % Symbol between number and power of ten
  group-minimum-digits = 5, % Number of digits when 3 digits separation appear
  % inter-unit-product   = \cdot, % Symbol between units (when several units are used)
  output-complex-root  = \ensuremath{i}, % How i math should be seen
  % prefixes-as-symbols  = false, % Translate prefixes (kilo, centi, milli, micro,...) into a power of ten
  separate-uncertainty = true, % Write uncertainty with +-
  scientific-notation  = engineering,
}

%%%%%%%%%%%%%%%%%%%%%%%%%%%%%%%%%%%%%%%%%%%%%%%%%%%%%%%%%%%%%%%%%%%%%%%%%%%%%%%
%%%% Theorems and proofs
\numberwithin{equation}{section}
% \makeatletter
% \g@addto@macro\th@remark{\thm@headpunct{:}}
% \makeatother
\theoremstyle{remark}
\newtheorem*{example}{Example}
\newtheorem*{remark}{Remark}

%%%%%%%%%%%%%%%%%%%%%%%%%%%%%%%%%%%%%%%%%%%%%%%%%%%%%%%%%%%%%%%%%%%%%%%%%%%%%%%
%%%% User-defined environments
% Remove the space before the enumerate and itemize environments
\let\oldenumerate\enumerate % Keep a copy of \enumerate (or \begin{enumerate})
\let\endoldenumerate\endenumerate % Keep a copy of \endenumerate (or \end{enumerate})
\renewenvironment{enumerate}{
  \begin{oldenumerate}
    \vspace{-6pt}
    }{
  \end{oldenumerate}
}

\let\olditemize\itemize % Keep a copy of \itemize (or \begin{itemize})
\let\endolditemize\enditemize % Keep a copy of \enditemize (or \end{itemize})
\renewenvironment{itemize}{
  \begin{olditemize}
    \vspace{-6pt}
    }{
  \end{olditemize}
}

\let\olddescription\description % Keep a copy of \description (or \begin{description})
\let\endolddescription\enddescription % Keep a copy of \enddescription (or \end{description})
\renewenvironment{description}{
  \begin{olddescription}
    \vspace{-6pt}
    }{
  \end{olddescription}
}

%%%%%%%%%%%%%%%%%%%%%%%%%%%%%%%%%%%%%%%%%%%%%%%%%%%%%%%%%%%%%%%%%%%%%%%%%%%%%%%
%%%% User-defined commands
\newcommand{\Romannumeral}[1]{\MakeUppercase{\romannumeral #1}} % Capital roman numbers
% \newcommand{\gui}[1]{\og #1 \fg{}} % French quotation marks
\renewcommand{\thefootnote}{[\arabic{footnote}]}

%%% Figure command
%% Include SVG files
\newcommand{\executeiffilenewer}[3]{
  \ifnum\pdfstrcmp{\pdffilemoddate{#1}}
    {\pdffilemoddate{#2}}>0
    {\immediate\write18{#3}}\fi
}
\newcommand{\includesvg}[1]{
  \executeiffilenewer{#1.svg}{#1.pdf}
  {
    % Inkscape must be installed in PATH and the user must include '--shell-escape' in the build arguments
    inkscape #1.svg --export-type=pdf --export-latex
  }
  \input{#1.pdf_tex}
}

%%% Math commands
%% Tables (requires cellspace package)
\newcolumntype{L}{>{\(\displaystyle}Cl<{\)}} % Column type for left-aligned math column
\newcolumntype{D}{>{\(\displaystyle}Cc<{\)}} % Column type for centered math column

%% Functions
\newcommand{\constant}{\mathrm{constant}} % Constant
\newcommand{\abs}[1]{\left| #1 \right|} % Absolute function
\newcommand{\erf}[1]{\mathrm{erf} \left( #1 \right)} % Error function
\newcommand{\erfc}[1]{\mathrm{erfc} \left( #1 \right)} % Complementary error function
\newcommand{\unitstep}[1]{\,\mathcal{U}\left( #1 \right)} % Unit step function
\newcommand{\diracdelta}[2]{\,\delta_{#1}\left( #2 \right)} % Dirac delta function


%% Derivatives and integrals
\newcommand{\diff}[2]{\mathrm{d}^{#1} #2} % Letter 'd' of differentials
\newcommand{\diffint}[1]{\,\diff{}{#1}} % Differential with a space for integrals
\newcommand{\derivative}[2]{\frac{\diff{}{#1}}{\diff{}{#2}}} % Derivative
\newcommand{\nderivative}[3]{\frac{\diff{#1}{#2}}{\diff{}{#3^{#1}}}} % Derivative of degree n
\newcommand{\partialderivative}[2]{\frac{\partial #1}{\partial #2}} % Partial derivative
\newcommand{\npartialderivative}[3]{\frac{\partial^{#1} #2}{\partial #3^{#1}}} % Partial derivative of degree n
\newcommand{\direcderivative}[2]{D_{\vec{#1}}\,#2} % Directional derivative

\newcommand{\Laplace}[1]{\mathcal{L}\left\{ #1 \right\}} % Laplace transform notation
\newcommand{\invLaplace}[1]{\mathcal{L}^{-1}\left\{ #1 \right\}} % Inverse Laplace transform notation

%% Set
\newcommand{\set}[3]{\mathbb{#1}_{#2}^{#3}} % Set of numbers
\newcommand{\integerset}{\mathbb{Z}} % Set of integer numbers (compatibility)
\newcommand{\realset}{\mathbb{R}} % Set of real numbers (compatibility)

%% Limits
\newcommand{\limit}[3]{\lim_{#1 \to #2}{#3}} % Limit from a point to another
\newcommand{\rlimit}[3]{\lim_{#1 \to #2^{+}}{#3}} % Right limit from a point to another
\newcommand{\llimit}[3]{\lim_{#1 \to #2^{-}}{#3}} % Left imit from a point to another
\newcommand{\modulus}[1]{\,\left[ #1 \right]} % Modulus notation

%% Vectors
\newcommand{\ivec}{\hat{\mathrm{i}}} % i vector
\newcommand{\jvec}{\hat{\mathrm{j}}} % j vector
\newcommand{\kvec}{\hat{\mathrm{k}}} % k vector
\renewcommand{\Vec}[1]{\overrightarrow{#1}} % Vector notation for expression with more than one letter
\newcommand{\norm}[1]{\left\| #1 \right\|} % Norm notation for expression with just one letter
\newcommand{\normvec}[1]{\left\| \vec{#1} \right\|} % Norm notation for expression with just one letter
\newcommand{\Normvec}[1]{\left\| \Vec{#1} \right\|} % Norm notation for expression with more than one letter
\newcommand{\comp}[2]{\mathrm{comp}_{\vec{#2}}\vec{#1}} % Components
\newcommand{\proj}[2]{\mathrm{proj}_{\vec{#2}}\vec{#1}}
\newcommand{\grad}[1]{\vec{\nabla}#1} % Gradient notation
\newcommand{\frames}[2]{\left( #1 \right)_{#2}} % Frame definition

\newcommand{\curl}[1]{\mathrm{curl}\,\vec{#1}} % Curl of a vector field
\newcommand{\divergence}[1]{\mathrm{div}\,\vec{#1}} % Divergence of a vector field


%%%%%%%%%%%%%%%%%%%%%%%%%%%%%%%%%%%%%%%%%%%%%%%%%%%%%%%%%%%%%%%%%%%%%%%%%%%%%%%
%%%% New document specific commands
% \renewcommand{\contentsname}{Table of contents} % Change the table of contents title

%%%%%%%%%%%%%%%%%%%%%%%%%%%%%%%%%%%%%%%%%%%%%%%%%%%%%%%%%%%%%%%%%%%%%%%%%%%%%%%
%%%% Theorems and proofs
\makeatletter
\g@addto@macro\th@remark{\thm@headpunct{:}}
\makeatother
\theoremstyle{remark}
\newtheorem*{example}{Example}

%%%%%%%%%%%%%%%%%%%%%%%%%%%%%%%%%%%%%%%%%%%%%%%%%%%%%%%%%%%%%%%%%%%%%%%%%%%%%%%
%%%% Beginning of the document
\begin{document}
\maketitle % Insert the cover page
% \tableofcontents
% \layout % Show a drawing of page layout
% \renewcommand{\abstractname}{} % Change the abstract title

\setcounter{section}{6}
\section{Vectors}
\subsection{Vectors in 2-space}
A vector is described by a magnitude, a line of action and a direction.

Vector properties:
\begin{itemize}
  \item \(\vec{a} + \vec{b} = \vec{b} + \vec{a}\)
  \item \(\vec{a} + (\vec{b} + \vec{c}) = (\vec{a} + \vec{b}) + \vec{c}\)
  \item \(\vec{a} + \vec{0} = \vec{a}\)
  \item \(\vec{a} + (-\vec{a}) = \vec{0}\)
  \item \(k(\vec{a} + \vec{b}) = k\vec{a} + k\vec{b}\)
  \item \((k_1 + k_2 )\vec{a} = k_1 \vec{a} + k_2 \vec{a}\)
  \item \(k_1 (k_2 \vec{a}) = (k_1 k_2 )\vec{a}\)
  \item \(1\vec{a} = \vec{a}\)
  \item \(0\vec{a} = \vec{0}\)
\end{itemize}

A 2-space vector \(\vec{a}\) have two components: \(\vec{a} = (a_1 , a_2 )\).
Vector operation with components:
\begin{description}
  \item[Equality] \(\vec{a} = \vec{b} \iff a_1 = b_1 \text{ and } a_2 = b_2 \)
  \item[Addition] \(\vec{a} + \vec{b} = (a_1 + b_1 , a_2 + b_2 )\)
  \item[Scalar multiplication] \(k\vec{a} = (ka_1 , ka_2 )\)
  \item[Magnitude] \(\normvec{a} = \sqrt{a_1 ^2 + a_2 ^2 }\)
  \item[Unit vector] \(\hat{a} = \frac{\vec{a}}{\normvec{a}}\)
  \item[Linear combination] \(\vec{u} = c_1 \vec{a} + c_2 \vec{b}\)
\end{description}

The elementary vectors in \(\set{R}{2}{}\) are \(\ivec = (1, 0)\) and \(\jvec = (0, 1)\).
Every vectors \(\vec{a}\) in \(\set{R}{2}{}\) can be represented as a linear combination of \(\ivec\) and \(\jvec\)
\[
  \vec{a} = a_1 \ivec + a_2 \jvec
\]

\subsection{Vectors in 3-space}
A 3-space vector \(\vec{a}\) have three components: \(\vec{a} = (a_1 , a_2 , a_3 )\).

The distance between two points \(A = (x_A , y_A , z_A )\) and \(B = (x_B , y_B , z_B )\) in \(\set{R}{3}{}\) is
\[
  AB = \Normvec{AB} = \sqrt{(x_B - x_A )^2 + (y_B - y_A )^2 + (z_B - z_A )^2 }
\]

The midpoint between two points \(A = (x_A , y_A , z_A )\) and \(B = (x_B , y_B , z_B )\) in \(\set{R}{3}{}\) is
\[
  I = \left( \frac{x_A + x_B }{2}, \frac{y_A + y_B }{2}, \frac{z_A + z_B }{2} \right)
\]

The elementary vectors in \(\set{R}{3}{}\) are \(\ivec = (1, 0, 0)\), \(\jvec = (0, 1, 0)\) and \(\kvec = (0, 0, 1)\).
Every vectors \(\vec{a}\) in \(\set{R}{3}{}\) can be represented as a linear combination of \(\ivec\), \(\jvec\) and \(\kvec\)
\[
  \vec{a} = a_1 \ivec + a_2 \jvec + a_3 \kvec
\]

\subsection{Dot product}
The dot product of two vectors \(\vec{a}\) and \(\vec{b}\) is a scalar and is defined by:
\begin{align*}
  \vec{a} \bullet \vec{b} & = a_1 b_1 + a_2 b_2 + a_3 b_3    \\
                          & = \normvec{a}\normvec{b}\cos\theta
\end{align*}

Properties of the dot product:
\begin{itemize}
  \item \(\vec{a} \bullet \vec{b} = 0 \iff \vec{a} = \vec{0} \text{ or } \vec{b} = \vec{0} \text{ or } \theta = \frac{\pi}{2} \modulus{\pi}\)
  \item \(\vec{a} \bullet \vec{b} = \vec{b} \bullet \vec{a}\)
  \item \(\vec{a} \bullet (\vec{b} + \vec{c}) = \vec{a} \bullet \vec{b} + \vec{a} \bullet \vec{c}\)
  \item \(\vec{a} \bullet (k\vec{b}) = (k\vec{a}) \bullet \vec{b} = k(\vec{a} \bullet \vec{b})\)
  \item \(\vec{a} \bullet \vec{a} = \normvec{a}^2 \geqslant 0\)
\end{itemize}

We can use these properties to determine the perpendicularity between two vectors: two non zero vectors \(\vec{a}\) and \(\vec{b}\) are perpendicular if and only if \(\vec{a} \bullet \vec{b} = 0\).

\subsubsection*{Direction angles}
The dot product is useful to determine the direction angles of a vector \(\vec{a}\), which is the angle between the vector and the axis:
\begin{align*}
   & x\text{-axis:} \cos\alpha = \frac{a_1 }{\normvec{a}} \\
   & y\text{-axis:} \cos\beta = \frac{a_2 }{\normvec{a}}  \\
   & z\text{-axis:} \cos\gamma = \frac{a_3 }{\normvec{a}}
\end{align*}
These formulas implies that
\[
  \hat{a} = (\cos\alpha, \cos\beta, \cos\gamma)
\]
\subsubsection*{Component and projection}
The component of a vector \(\vec{a}\) along a vector \(\vec{b}\) is \(\comp{a}{b} = \normvec{a}\cos\theta = \frac{\vec{a} \bullet \vec{b}}{\normvec{b}}\).
The projection of a vector \(\vec{a}\) along a vector \(\vec{b}\) is \(\proj{a}{b} = (\comp{a}{b})\hat{b} = \frac{\vec{a} \bullet \vec{b}}{\normvec{b}^2 }\vec{b}\).

\subsection{Cross product}
The cross product between two vectors \(\vec{a}\) and \(\vec{b}\) is
\begin{align*}
  \vec{a} \times \vec{b} & =
  \begin{vmatrix}
    \ivec & \jvec & \kvec \\
    a_1   & a_2   & a_3   \\
    b_1   & b_2   & b_3
  \end{vmatrix}                                                                                       \\
                         & = (a_2 b_3 - b_2 a_3 )\ivec + (a_3 b_1 - b_3 a_1 )\jvec + (a_1 b_2 - b_1 a_2 )\kvec \\
                         & = \left( \normvec{a}\normvec{b}\sin\theta \right)\hat{n}
\end{align*}
where \(\hat{n}\) is a unit vector orthogonal to the plane of \(\vec{a}\) and \(\vec{b}\).

Properties of the dot product:
\begin{itemize}
  \item \(\vec{a} \times \vec{b} = 0 \iff \vec{a} = \vec{0} \text{ or } \vec{b} = \vec{0} \text{ or } \theta = 0 \modulus{\pi}\)
  \item \(\vec{a} \times \vec{b} = -\vec{b} \times \vec{a}\)
  \item \(\vec{a} \times (\vec{b} + \vec{c}) = \vec{a} \times \vec{b} + \vec{a} \times \vec{c}\)
  \item \((\vec{a} + \vec{b}) \times \vec{c} = \vec{a} \times \vec{c} + \vec{b} \times \vec{c}\)
  \item \(\vec{a} \times (k\vec{b}) = (k\vec{a}) \times \vec{b} = k(\vec{a} \times \vec{b})\)
  \item \(\vec{a} \times \vec{a} = 0\)
  \item \(\vec{a} \bullet (\vec{a} \times \vec{b}) = 0\)
  \item \(\vec{b} \bullet (\vec{a} \times \vec{b}) = 0\)
\end{itemize}

We can use these properties to determine the parallelism between two vectors: two non zero vectors \(\vec{a}\) and \(\vec{b}\) are parallel if and only if \(\vec{a} \times \vec{b} = \vec{0}\).

\subsubsection*{Special products}
Scalar triple product:
\[
  \vec{a} \bullet (\vec{b} \times \vec{c}) = ( \vec{a} \times \vec{b}) \bullet \vec{c} =
  \begin{vmatrix}
    a_1 & a_2 & a_3 \\
    b_1 & b_2 & b_3 \\
    c_1 & c_2 & c_3
  \end{vmatrix}
\]

Vector triple product:
\[
  \vec{a} \times (\vec{b} \times \vec{c}) = (\vec{a} \bullet \vec{c})\vec{b} - (\vec{a} \bullet \vec{b})\vec{c}
\]

\subsubsection*{Applications}
Area of a parallelogram: \(A = \norm{\vec{a} \times \vec{b}}\)

Area of a triangle: \(A = \frac{1}{2}\norm{\vec{a} \times \vec{b}}\)

Volume of a parallelepiped: \(V = \abs{\vec{a} \bullet \left( \vec{b} \times \vec{c} \right)}\)

Coplanar vectors: \(\vec{a}\), \(\vec{b}\) and \(\vec{c}\) are coplanar if and only if \(\vec{a} \bullet (\vec{b} \times \vec{c}) = 0\).


\subsection{Lines and planes in space}
\subsubsection*{Vector equations of a line}
The parametric equation of a line is of the form
\[
  \vec{r} = \vec{a}t + \vec{r}_0
\]
where \(\vec{r} = (x, y, z)\) are the coordinates in \(\set{R}{3}{}\), \(\vec{a} = (a_1 , a_2 , a_3 )\) is the direction vector of the line, \(\vec{r}_0 = (x_0 , y_0 , z_0 )\) is any point on the line and \(t\) is a scalar variable.
This equation can also be written as a system:
\[
  \begin{cases}
    x = a_1 t + x_0 \\
    y = a_2 t + y_0 \\
    z = a_3 t + z_0
  \end{cases}
\]

From those equation, the symmetric equation can be found:
\[
  t = \frac{x - x_0 }{a_1 } = \frac{y - y_0 }{a_2 } = \frac{z - z_0 }{a_3 }
\]
If any components of \(\vec{a}\) is equal to 0, then, the dependent variable is set to its corresponding component of \(\vec{r}_0 \).
\begin{example}
  If \(\vec{a} = (0, a_2 , a_3 )\), then we have \(x = x_0 \) and \(\frac{y - y_0 }{a_2 } = \frac{z - z_0 }{a_3 }\)
\end{example}

\subsubsection*{Vector equations of a plane}
The cartesian equation of a plane is of the form
\[
  \vec{n} \bullet \left( \vec{r} - \vec{r}_0 \right) = 0
\]
where \(\vec{n} = (a, b, c)\) is the normal vector of the plane, \(\vec{r} = (x, y, z)\) are the coordinates in \(\set{R}{3}{}\) and \(\vec{r}_0 = (x_0 , y_0 , z_0 )\) is any point on the plane.
This equation can be expanded in the form
\[
  ax + by + cz = d \text{, where } d = ax_0 + by_0 + cz_0
\]

Method to find the equation of the plane containing three points \(A\), \(B\) and \(C\):
\begin{enumerate}
  \item Build 3 vectors \(\Vec{AB}\), \(\Vec{CB}\) and \(\Vec{CX}\), where \(X = (x, y, z)\)
  \item Compute the scalar triple product \((\Vec{AB} \times \Vec{CB}) \bullet \Vec{CX} = 0\)
\end{enumerate}

Method to find the line of intersection between to planes:
\begin{enumerate}
  \item Let either \(x\), \(y\) or \(z\) equal to \(t\)
  \item Build the system where the other 2 variables depends on \(t\)
  \item Solve to get \(x\), \(y\), and \(z\) dependent on \(t\)
\end{enumerate}

\addtocounter{section}{1}
\section{Vector calculus}
\subsection{Vector functions}
A vector function is a function such that at least one component of a vector \(\vec{r}\) is dependent on another variable:
\[
  \vec{r}(t) = \left( x(t), y(t), z(t) \right)
\]
\begin{example}
  The vector function of a circular helix is of the form:
  \[
    \vec{r}(t) = \left( \alpha\cos\beta t, \alpha\sin\beta t, kt \right)
  \]
\end{example}

\subsubsection*{Limit of a vector function}
If \(\limit{a}{t}{x(t)}\), \(\limit{a}{t}{y(t)}\) and \(\limit{a}{t}{z(t)}\) exist, then
\[
  \limit{a}{t}{\vec{r}(t)} = \left( \limit{a}{t}{x(t)}, \limit{a}{t}{y(t)}, \limit{a}{t}{z(t)} \right)
\]

\paragraph*{Properties of limits}
If \(\limit{a}{t}{\vec{r}_1 (t)} = \vec{L}_1 \) and \(\limit{a}{t}{\vec{r}_2 (t)} = \vec{L}_2 \), then:
\begin{align*}
   & \limit{a}{t}{k\vec{r}_1 (t)} = k\vec{L}_1                                        \\
   & \limit{a}{t}{\vec{r}_1 (t) + \vec{r}_2 (t)} = \vec{L}_1 + \vec{L}_2             \\
   & \limit{a}{t}{\vec{r}_1 (t) \bullet \vec{r}_2 (t)} = \vec{L}_1 \bullet \vec{L}_2
\end{align*}

A vector function \(\vec{r}\) is said to be continuous at \(t = a\) if:
\begin{itemize}
  \item \(\vec{r}(a)\) is defined
  \item \(\limit{a}{t}{\vec{r}(t)}\) exists
  \item \(\limit{a}{t}{\vec{r}(t)} = \vec{r}(a)\)
\end{itemize}

\subsubsection*{Derivative of a vector function}
If \(\vec{r}(t) = \left( x(t), y(t), z(t) \right)\), where \(x\), \(y\) and \(z\) are differentiable, then
\[
  \derivative{}{t}\left[ \vec{r}(t) \right] = \left( \derivative{x}{t}, \derivative{y}{t}, \derivative{z}{t} \right)
\]

\paragraph*{Smooth curves}
When the components functions of a vector function \(\vec{r}\) have continuous first derivatives and \(\vec{r}'(t) \neq 0\) for all \(t\) in the open interval \((a, b)\), then \(\vec{r}\) is said to be a smooth function and the curve \(\mathscr{C}\) traced by \(\vec{r}\) is called a smooth curve.

Method to find parametric equation of the tangent line to a curve for \(t = k\):
\begin{enumerate}
  \item If not given, find the vector function \(\vec{r}\)
  \item Find the point of tangency \(\vec{r}(k)\)
  \item Compute \(\vec{r}'\)
  \item Find the direction vector at the point of tangency \(\vec{r}'(k)\)
  \item The parametric equation is \(\vec{T} = \vec{r}'(k)t + \vec{r}(k)\)
\end{enumerate}

\paragraph*{Properties of derivatives}
Let \(\vec{r}_1 (t)\) and \(\vec{r}_2 (t)\) be differentiable vector functions and \(u(t)\) a differentiable scalar function:
\begin{align*}
   & \derivative{}{t}\left[ \vec{r}_1 (t) + \vec{r}_2 (t) \right] = \vec{r}_1 '(t) + \vec{r}_2 '(t)                                                   \\
   & \derivative{}{t}\left[ u(t)\vec{r}_1 (t) \right] = u(t)\vec{r}_1 '(t) + u'(t)\vec{r}_1 (t)                                                       \\
   & \derivative{}{t}\left[ \vec{r}_1 (t) \bullet \vec{r}_2 (t) \right] = \vec{r}_1 '(t) \bullet \vec{r}_2 (t) + \vec{r}_1 (t) \bullet \vec{r}_2 '(t) \\
   & \derivative{}{t}\left[ \vec{r}_1 (t) \times \vec{r}_2 (t) \right] = \vec{r}_1 '(t) \times \vec{r}_2 (t) + \vec{r}_1 (t) \times \vec{r}_2 '(t)    \\
\end{align*}

\subsubsection*{Integrals of a vector function}
If \(\vec{r}(t) = \left( x(t), y(t), z(t) \right)\), where \(x\), \(y\) and \(z\) are integrable, then
\begin{align*}
  \int{\vec{r}(t) \diffint{t}}       & = \left( \int{x(t) \diffint{t}}, \int{y(t) \diffint{t}}, \int{z(t) \diffint{t}} \right)                   \\
  \int_a ^b {\vec{r}(t) \diffint{t}} & = \left( \int_a ^b {x(t) \diffint{t}}, \int_a ^b {y(t) \diffint{t}}, \int_a ^b {z(t) \diffint{t}} \right)
\end{align*}

\subsubsection*{Length of a space curve}
If \(\vec{r}(t) = \left( x(t), y(t), z(t) \right)\) is a smooth function, then it can be shown that the length of the smooth curve traced by \(\vec{r}\) is given by
\[
  s = \int_a ^b {\norm{\vec{r}'(t)} \diffint{t}} = \int_a ^b {\sqrt{\left[ x'(t)^2 \right] + \left[ y'(t)^2 \right] + \left[ z'(t)^2 \right]} \diffint{t}}
\]

\subsection{Motion on a curve}
Suppose a body or a particle moves along a curve \(\mathscr{C}\) so that its position at time \(t\) is given by the vector function
\[
  \vec{r}(t) = \left( x(t), y(t), z(t) \right)
\]
The velocity and acceleration of the particle are
\begin{align*}
  \vec{v}(t) = \vec{r}'(t)  & = \left( x'(t), y'(t), z'(t) \right)    \\
  \vec{a}(t) = \vec{r}''(t) & = \left( x''(t), y''(t), z''(t) \right)
\end{align*}
The speed of the particle is the magnitude of the velocity:
\[
  v(t) = \norm{\vec{v}(t)} = \sqrt{x'(t)^2 + y'(t)^2 + z'(t)^2 }
\]

\subsubsection*{Centripetal motion}
If \(\vec{r}(t) = (r_0 \cos\omega t, r_0 \sin\omega t)\), where \(r_0 \) and \(\omega\) are constants, then the acceleration is \(\vec{r}'' = -\omega^2 \vec{r}\) and the position and acceleration vectors are in opposite direction (this is the case for a circular motion).
Also, \(\normvec{a} = \frac{\normvec{v}^2 }{\norm{\vec{r}_0 }}\).

\subsubsection*{Trajectory of a projectile}
Due to the gravity, we have \(\vec{a} = (0, -g)\).
Let the initial velocity be \(\vec{v}_0 = (v_0 \cos\theta, v_0 \sin\theta)\) and the initial height be \(\vec{s}_0 = (0, s_0 )\).
Then the velocity vector \(\vec{v}\) is \(\vec{v} = (v_0 \cos\theta, -gt + v_0 \sin\theta)\), and the position vector \(\vec{r}\) is \(\vec{r} = (v_0 t\cos\theta, -\frac{1}{2}gt^2 + v_0 t\sin\theta + s_0 )\).

\subsection{Curvature and components of acceleration}
\subsubsection*{Unit tangent vector}
Let \(\vec{r}(t)\) be a vector function defining a smooth curve \(\mathscr{C}\) and let the unit tangent vector \(\hat{T}(t)\) be
\[
  \hat{T}(t) = \hat{r}'(t) = \frac{\vec{r}'(t)}{\norm{\vec{r}'(t)}}
\]
If \(s\) is the arc length parameter and  is the unit tangent vector, then the curvature of \(\mathscr{C}\) at a point is
\[
  \kappa = \norm{\derivative{\hat{T}}{s}} \text{ and } \kappa(t) = \frac{\norm{\vec{T}'(t)}}{\norm{\vec{r}'(t)}}
\]

Method to find the curvature of a curve defined by \(\vec{r}(t)\):
\begin{enumerate}
  \item Compute \(\vec{r}'(t)\) and \(\norm{\vec{r}'(t)}\)
  \item Compute \(\hat{T}(t) = \hat{r}'(t) = \frac{\vec{r}'(t)}{\norm{\vec{r}'(t)}}\), \(\vec{T}'(t)\) and \(\norm{\vec{T}'(t)}\)
  \item Compute \(\kappa(t) = \frac{\norm{\vec{T}'(t)}}{\norm{\vec{r}'(t)}}\)
\end{enumerate}

\subsubsection*{Principal normal}
Then, from the unit tangent vector \(\hat{T}(t)\), we can find the principal normal which is defined as
\[
  \hat{N}(t) = \hat{T}'(t) = \frac{\vec{T}'(t)}{\norm{\vec{T}'(t)}}
\]

In the case of motion, the curvature \(\kappa(t)\) can be re-written as \(\kappa(t) = \frac{\norm{\vec{T}'(t)}}{\norm{\vec{v}(t)}}\) and velocity \(\vec{v}(t)\) will be \(\vec{v}(t) = \normvec{v}\hat{T}(t)\)
Finally, using the unit tangent vector \(\hat{T}(t)\), the curvature \(\kappa(t)\) and the principal normal \(\hat{N}(t)\), the acceleration vector can be re-written as
\begin{align*}
  \vec{a}(t)        & = a_N \hat{N} + a_T \hat{T}                                          \\
  \text{where } a_N & = \kappa \normvec{v}^2 \text{ is the normal acceleration}            \\
  a_T               & = \derivative{\normvec{v}}{t} \text{ is the tangential acceleration}
\end{align*}

\subsubsection*{Binormal vector}
The binormal vector is defined as \(\hat{B}(t) = \hat{T}(t) \times  \hat{N}(t)\).

Now, three planes can be defined:
\begin{description}
  \item[Plane TN:] osculating plane
  \item[Plane NB:] normal plane
  \item[Plane TB:] rectifying plane
\end{description}

The vectors \(\hat{T}(t)\), \(\hat{N}(t)\) and \(\hat{B}(t)\) form a right-handed system referred to as trihedral system.

Using all those formulas, we can find several formulas
\begin{align*}
  a_T       & = \derivative{v}{t} = \frac{\vec{v} \bullet \vec{a}}{\normvec{v}}          \\
  a_N       & = \kappa \normvec{v}^2 = \frac{\norm{\vec{v} \times \vec{a}}}{\normvec{v}} \\
  \kappa(t) & = \frac{\norm{\vec{v} \times \vec{a}}}{\normvec{v}^3 }
\end{align*}

Finally, the reciprocal \(\rho\) of curvature is called radius of curvature and is defined as
\[
  \rho = \frac{1}{\kappa}
\]

\subsection{Partial derivatives}
Let \(z = f(x, y)\) denote a surface.
The level curves are \(f(x, y) = c\) (same thing as lines of altitude on elevation maps).

For a function of three variables, we have \(w = f(x, y, z)\)
The level surfaces are \(f(x, y, z) = c\).

For \(y = f(x)\), we had
\[
  \derivative{y}{x} = \limit{\Delta x}{0}{\frac{f(x + \Delta x) - f(x)}{\Delta x}}
\]
Now, in case of partial derivatives, for \(z = f(x, y)\) we have
\begin{align*}
  \partialderivative{z}{x} & = \limit{\Delta x}{0}{\frac{f(x + \Delta x, y) - f(x, y)}{\Delta x}} \\
  \partialderivative{z}{y} & = \limit{\Delta y}{0}{\frac{f(x, y + \Delta y) - f(x, y)}{\Delta y}}
\end{align*}

To compute \(\partialderivative{z}{x}\), we use the laws of ordinary differentiation while treating \(y\) as a constant.
The same applies for computing \(\partialderivative{z}{y}\).

\subsubsection*{Notation of partial derivatives}
Partial derivative representation are:
\begin{itemize}
  \item First-order:
        \[
          \partialderivative{z}{x} = \partialderivative{f}{x} = f_x
        \]
  \item Second-order:
        \[
          \npartialderivative{2}{z}{x} = f_{xx}
        \]
  \item Mixed second-order:
        \[
          \frac{\partial^2 z}{\partial x \partial y} = \partialderivative{}{y} \left( \partialderivative{z}{y} \right) = f_{yx}
        \]
        If a function \(f\) has continuous second partial derivatives, then the order in which a mixed second partial derivative is done is irrelevant, i.e. \(f_{yx} = f_{xy}\)
\end{itemize}

\subsubsection*{Chain rule}
If \(z = f(u, v)\) is differentiable and \(u = g(x, y)\) and \(v = h(x, y)\) have continuous first-order partial derivatives, then
\[
  \partialderivative{z}{x} = \partialderivative{z}{u}\partialderivative{u}{x} + \partialderivative{z}{v}\partialderivative{v}{x}
\]
In order to simplify the calculations, we can use the tree method (can be expanded with more functions and more independent variables):
\begin{enumerate}
  \item Place the dependent variable \(z\) at the top
  \item Place \(u\) and \(v\) under \(z\)
  \item Place \(x\) and \(y\) under each \(u\) and \(v\)
  \item The "roots" of the tree, are partial derivative of the top with respect of the bottom
  \item To get \(\partialderivative{z}{x}\) multiply the partial derivatives from \(z\) until \(x\) and sum them up for each root.
\end{enumerate}

\subsection{Directional derivatives}
\subsubsection*{Gradient}
The gradient of a function is a vector which points in the direction of most increase and is defined as
\begin{align*}
  \grad{f(x, y)}    & = \left( \partialderivative{f}{x}, \partialderivative{f}{y} \right)                           \\
  \grad{f(x, y, z)} & = \left( \partialderivative{f}{x}, \partialderivative{f}{y}, \partialderivative{f}{z} \right)
\end{align*}


\subsubsection*{Directional derivative}
The directional derivative \(\direcderivative{u}{z}\) of \(z = f(x, y)\) in the direction of a unit vector \(\vec{u} = (\cos\theta, \sin\theta)\) is the generalization of partial differentiation of the function \(f\), and it is defined as:
\[
  \direcderivative{u}{f(x, y)} = \grad{f(x, y)} \bullet \vec{u} = \norm{\grad{f}}\cos\phi
\]
Therefore, we have \(-\norm{\grad{f}} \leqslant D_{\vec{u}}\,f \leqslant \norm{\grad{f}}\).
This can also be interpreted as: the gradient vector \(\grad{f}\) points in the direction in which \(f\) increases most rapidly, whereas \(-\grad{f}\) points in the direction in which \(f\) decreases most rapidly.

\subsection{Tangent planes and normal lines}
\subsubsection*{Tangent planes}
Suppose \(f(x, y) = c\) is the level curve of the differentiable function \(z = f(x, y)\) that passe through a specified point \(P(x_0 , y_0 )\), i.e. \(f(x_0 , y_0 ) = c\).
Then, the following property can be shown: \(\grad{f(x_0 , y_0 )} \bullet \vec{r}'(t_0 ) = 0\) which means that \(\grad{f}\) is orthogonal to the level curve at \(P\).
For a function of three variables, we have \(\grad{f(x_0 , y_0 , z_0 )} \bullet \vec{r}'(t_0 ) = 0\), i.e. \(\grad{f}\) is normal to the level surface at \(P\).

The tangent plane at \(P\) is the plane normal to \(\grad{f}\) evaluated at P.
If \(P(x, y, z)\) and \(P(x_0 , y_0 , z_0 )\) are points on the tangent plane and \(\vec{r}\) and \(\vec{r}_0 \) are their respective position vectors, then a vector equation of the tangent plane is
\[
  \grad{f(x_0 , y_0 , z_0 )} \bullet (\vec{r} - \vec{r}_0 ) = 0
\]
with the following expanded form:
\begin{multline*}
  f_x (x_0 , y_0 , z_0 )(x - x_0 ) \\ + f_y (x_0 , y_0 , z_0 )(y - y_0 ) \\ +  f_z (x_0 , y_0 , z_0 )(z - z_0 ) = 0
\end{multline*}

\subsubsection*{Normal line}
The normal line to the surface at a point \(P(x_0 , y_0 , z_0 )\) is the line normal to the tangent plane of the surface at \(P\).
The parametric equation of the normal line is
\[
  \frac{x - x_0 }{f_x (x_0 , y_0 , z_0 )} = \frac{y - y_0 }{f_y (x_0 , y_0 , z_0 )} = \frac{z - z_0 }{f_z (x_0 , y_0 , z_0 )}
\]

\subsection{Curl and divergence}
\subsubsection*{Vector field}
A vector field is made of a set of vector.
It can be seen as a vector function in which the components can depend on several variables:
\begin{align*}
  \vec{v}(x, y)    & = \left( P(x, y), Q(x, y) \right)                   \\
  \vec{v}(x, y, z) & = \left( P(x, y, z), Q(x, y, z), R(x, y, z) \right)
\end{align*}

\subsubsection*{Curl}
The curl of a vector field \(\vec{v}\) is another vector field such that
\[
  \curl{v}= \grad{} \times \vec{v}
\]

\subsubsection*{Flux}
The flux is the volume of the fluid flowing through an element of surface area \(\Delta S\) per unit of time.
The flux can be obtained using
\[
  \text{flux } = \left( \comp{v}{n} \right){\Delta S}_{\mathrm{base}} = \left( \vec{v} \bullet \vec{n} \right){\Delta S}_{\mathrm{base}}
\]

The net flux of \(\vec{v}\) is defined as
\[
  \left( \partialderivative{P}{x} + \partialderivative{Q}{y} + \partialderivative{R}{z} \right)\Delta x \Delta y \Delta z
\]
and the outward flux of \(\vec{v}\) per unit volume is \(\partialderivative{P}{x} + \partialderivative{Q}{y} + \partialderivative{R}{z}\).

\subsubsection*{Divergence}
The divergence of a vector field \(\vec{F} = (P, Q, R)\) is the scalar function
\[
  \divergence{v} = \grad{} \bullet \vec{v} = \partialderivative{P}{x} + \partialderivative{Q}{y} + \partialderivative{R}{z}
\]

\subsubsection*{Physical interpretation}
\begin{description}
  \item[Curl] The curl of the velocity field \(\vec{v}\) is a measure of the tendency of the fluid to turn a device about its vertical axis.
  \item[Divergence] The divergence is a measure of the fluid's compressibility
        \begin{itemize}
          \vspace{6pt}
          \item If \(\divergence{v}(P) > 0\), then \(P\) is said to be a source for \(\vec{v}\).
          \item If \(\divergence{v}(P) = 0\), then the fluid is said to be incompressible.
          \item If \(\divergence{v}(P) < 0\), then \(P\) is said to be a sink for \(\vec{v}\).
        \end{itemize}
\end{description}

If \(F\) is a vector field having continuous second partial derivatives, then \(\mathrm{div}(\curl{F}) = 0\).
If \(f\) is a scalar function with continuous second partial derivatives, then \(\mathrm{curl}(\grad{f}) = \vec{0}\).

\subsection{Line integrals}
\subsubsection*{Terminology}
Suppose \(\mathscr{C}\) is a curve parameterized by \(\vec{F} = \left( f(t), g(t) \right)\), \(a \leqslant t \leqslant b\), and \(A = \left( f(a), g(a) \right)\) and \(B = \left( f(b), g(b) \right)\)
\begin{description}
  \item[Smooth curve:] if \(f'\) and \(g'\) are continuous on the closed interval \([a, b]\) and not simultaneously 0 on the open interval \((a, b)\).
  \item[Piecewise smooth:] if it consists of a finite number of smooth curves joined en to end.
  \item[Closed curve:] is \(A = B\)
  \item[Simple closed curve:] if \(A = B\) and the curve does not cross itself.
\end{description}

\subsubsection*{Line integrals definition}
Let \(\vec{F} = \left( P(x, y), Q(x, y) \right)\) be a function of two variables \(x\) and \(y\) defined on a region of the place containing a smooth curve \(\mathscr{C}\).
The line integral of \(\vec{F}\) along \(\mathscr{C}\) from \(A\) to \(B\) is
\[
  \int_{\mathscr{C}}{\vec{F} \bullet \diffint{\vec{r}}} = \int_{\mathscr{C}}{ P(x, y) \diffint{x} + Q(x, y) \diffint{y}}
\]

A line integral along a piecewise-smooth curve \(\mathscr{C}\) is defined as the sum of the integrals over the various smooth curves whose \(\mathscr{C}\) is made of.
This also applies to contour integrals which might be made of piecewise-smooth curve.

\subsubsection*{Physical interpretation}
\begin{description}
  \item[Work] If we have a force \(\vec{F} = \left( P(x, y, z), Q(x, y, z), R(x, y, z) \right)\) along a curve \(\mathscr{C}\), then the work done by this force is
        \[
          W_{\mathscr{C}} = \int_{\mathscr{C}}{\vec{F} \bullet \diffint{\vec{r}}} = \int_{\mathscr{C}}{\comp{F}{T} \diffint{s}}
        \]
        where \(\hat{T}\) is the unit tangent vector.
        This means that the work done by a force \(\vec{F}\) along a curve \(\mathscr{C}\) is due entirely to the tangential component of \(\vec{F}\).
  \item[Mass] If a wire corresponding to a curve \(\mathscr{C}\) has a variable density \(\rho(x, y)\) in mass per unit length, then the mass of the wire along the curve \(\mathscr{C}\) defined by \(\vec{r}(t)\) is
        \[
          m = \int_{\mathscr{C}}{\rho \diffint{s}} = \int_{\mathscr{C}}{\rho \norm{\vec{r}'} \diffint{t}}
        \]
  \item[Circulation] The circulation is defined as
        \[
          \mathrm{circulation} = \oint_{\mathscr{C}}{\vec{F} \bullet \diffint{\vec{r}}} = \oint_{\mathscr{C}}{\comp{F}{T} \diffint{s}}
        \]
\end{description}

\subsection{Independent of the path}
An integral is independent of the path if no matter what is the curve, the integral remains the same.

A vector function \(\vec{F}\) is said to be conservative if \(\vec{F}\) can be written as the gradient of a scalar function \(\phi\).
The function \(\phi\) is called a potential function of \(\vec{F}\).
In other words, \(\vec{F}\) is conservative if there exists function \(\phi\) such that \(\vec{F} = \grad{\phi}\).
A conservative vector field is also called a gradient vector field.

\paragraph*{Fundamental theorem of the line integral}
\[
  \int_{\mathscr{C}}{\vec{F} \bullet \diffint{\vec{r}}} = \int_{\mathscr{C}}{\grad{\phi} \bullet \diffint{\vec{r}}} = \phi(B) - \phi(A)
\]
where \(A\) is the starting point and \(B\) is the end point.

In an open connected region \(R\), the integral is independent of the path \(\mathscr{C}\) if and only if the vector field \(\vec{F}\) is conservative in \(R\) or if \(\oint_{\mathscr{C}}{\vec{F} \bullet \diffint{\vec{r}}} = 0\) for every closed path \(\mathscr{C}\) in \(R\).

Therefore, we have:
\begin{align*}
  \vec{F} \text{ conservative} & \iff \text{path independence}                                   \\
                               & \iff \oint_{\mathscr{C}}{\vec{F} \bullet \diffint{\vec{r}}} = 0
\end{align*}
\begin{itemize}
  \item In \(\set{R}{2}{}\), \(\vec{F} = \left( P(x, y), Q(x, y) \right)\) is conservative if
        \[
          \partialderivative{P}{y} = \partialderivative{Q}{x}
        \]
  \item In \(\set{R}{3}{}\), \(\vec{F} = \left( P(x, y, z), Q(x, y, z), R(x, y, z) \right)\) is conservative if
        \[
          \partialderivative{P}{y} = \partialderivative{Q}{x} \text{ and } \partialderivative{P}{z} = \partialderivative{R}{x}  \text{ and }\partialderivative{Q}{z} = \partialderivative{R}{y}
        \]
        This is equivalent to \(\curl{F} = \vec{0}\).
\end{itemize}


Method for finding \(\phi(x, y)\) in \(\set{R}{2}{}\) (very similar to solving an exact differential equation):
\begin{enumerate}
  \item Check for \(\vec{F} = \left( P(x, y), Q(x, y) \right)\) being conservative
  \item If \(\vec{F} = \left( P(x, y), Q(x, y) \right)\) is conservative, there exists a function \(\phi(x,y)\) such that:
        \[
          \partialderivative{\phi(x, y)}{x} = P(x,y) \text{ and } \partialderivative{\phi(x, y)}{y} = Q(x,y)
        \]
  \item Find \(\phi(x,y)\) by integrating \(P(x,y)\) with respect to \(x\), while holding \(y\) constant.
        This gives:
        \[
          \phi(x,y) = \int{P(x,y)\diffint{x}} + g(y)
        \]
        where an arbitrary function \(g(y)\) is the "constant" of integration
  \item Differentiate \(\phi(x,y)\) with respect to \(y\) and set it equals to \(Q(x,y)\):
        \[
          \partialderivative{\phi(x, y)}{y} = \partialderivative{}{y}\left[ \int{P(x,y)\diffint{x}} \right] + g'(y) = Q(x,y)
        \]
  \item This gives:
        \[
          g'(y) = Q(x,y) - \partialderivative{}{y}\left[ \int{P(x,y)\diffint{x}} \right]
        \]
  \item Integrate \(g'(y)\) with respect to \(y\)
  \item Substitute the result in \(\phi(x,y) = \int{P(x,y)\diffint{x}} + g(y)\)
\end{enumerate}

Therefore, if \(\vec{F} = \left( P(x, y), Q(x, y) \right)\) is conservative, the full solution becomes:
\begin{multline*}
  \phi(x,y) = \int{P(x,y)\diffint{x}} \\
  + \int{Q(x,y) - \partialderivative{}{y}\left[ \int{P(x,y)\diffint{x}} \right]\diffint{y}}
\end{multline*}

The method for finding \(\phi(x, y, z)\) in \(\set{R}{3}{}\) is very similar, but \(g(y)\) becomes \(g(y, z)\) and therefore, we will need to do another integral in order to find a third function \(h(z)\).

\subsection{Double integrals}
Let \(f\) be a function of two variables defined on a closed region \(R\) of \(\set{R}{2}{}\).
Then the double integral of \(f\) over \(R\) is given by
\[
  \iint_R {f(x, y) \diffint{A}} = \iint_R {f(x, y) \diffint{x} \diffint{y}}
\]

\subsubsection*{Properties}
Let \(f\) and \(g\) be functions of two variables that are integrable over a region \(R\), then:
\begin{itemize}
  \item \(\iint_R {kf(x, y) \diffint{A}} = k\iint_R {f(x, y) \diffint{A}}\), where \(k \in \set{R}{}{}\).
  \item \(\iint_R {\left[ f(x, y) \pm g(x, y) \right] \diffint{A}} = \iint_R {f(x, y) \diffint{A}} \pm \iint_R {g(x, y) \diffint{A}}\)
  \item \(\iint_R {f(x, y) \diffint{A}} = \iint_{R_1 }{f(x, y) \diffint{A}} + \iint_{R_1 }{f(x, y) \diffint{A}}\), where \(R_1 \) and \(R_2 \) are subregions of \(R\) that not overlap and \(R = R_1 \cup R_2 \)
\end{itemize}

\subsubsection*{Computation}
For region of Type \Romannumeral{1}:
\[
  \int_a ^b {\int_{g_1 (x)}^{g_2 (x)}{f(x, y) \diffint{y}}\diffint{x}} = \int_a ^b {\left[ \int_{g_1 (x)}^{g_2 (x)}{f(x, y) \diffint{y}} \right]\diffint{x}}
\]
For region of Type \Romannumeral{2}:
\[
  \int_c ^d {\int_{h_1 (x)}^{h_2 (x)}{f(x, y) \diffint{x}}\diffint{y}} = \int_c ^d {\left[ \int_{h_1 (x)}^{h_2 (x)}{f(x, y) \diffint{x}} \right]\diffint{y}}
\]

\paragraph*{Fubini's theorem}
Let \(f\) be continuous on a region \(R\).
If \(R\) is of Type \Romannumeral{1}, then
\[
  \iint_R {f(x, y) \diffint{A}} = \int_a ^b {\int_{g_1 (x)}^{g_2 (x)}{f(x, y) \diffint{y}}\diffint{x}}
\]
If \(R\) is of Type \Romannumeral{2}, then
\[
  \iint_R {f(x, y) \diffint{A}} = \int_c ^d {\int_{h_1 (x)}^{h_2 (x)}{f(x, y) \diffint{x}}\diffint{y}}
\]

\subsubsection*{Mass}
If a lamina corresponding to a region \(R\) has a variable density \(\rho(x, y)\) continuous on \(R\), then
\[
  m = \iint_R {\rho(x, y) \diffint{A}}
\]

The coordinates of the center of mass of the lamina are:
\begin{align*}
  x & = \frac{M_y }{m} = \frac{1}{m} \iint_R {x\rho(x, y) \diffint{A}} \\
  y & = \frac{M_x }{m} = \frac{1}{m} \iint_R {y\rho(x, y) \diffint{A}}
\end{align*}
The moments of inertia of the lamina are:
\begin{align*}
  I_x & = \iint_R {y^2 \rho(x, y) \diffint{A}} \\
  I_y & = \iint_R {x^2 \rho(x, y) \diffint{A}}
\end{align*}

The radius of gyration \(K\) of a lamina of mass \(m\) is defined by \(K = \sqrt{\frac{I}{m}}\).

\subsection{Double integrals in polar coordinates}
The double integral of a function \(f(r, \theta)\) with respect to an area in polar coordinates is
\begin{align*}
  \iint_R {f(r, \theta) \diffint{A}} & = \int_{\alpha}^{\beta}{\int_{g_1 (\theta)}^{g_2 (\theta)}{f(r, \theta)r \diffint{r}}\diffint{\theta}} \\
                                     & = \int_a ^b {\int_{h_1 (r)}^{h_2 (r)}{f(r, \theta)r \diffint{\theta}}\diffint{r}}
\end{align*}

A standard double integral function of \(x\) and \(y\) can be written as a double integral using polar coordinates:
\[
  \iint_R {f(x, y) \diffint{A}} = \int_{\alpha}^{\beta}{\int_{g_1 (\theta)}^{g_2 (\theta)}{f(r\cos\theta, r\sin\theta)r \diffint{r}}\diffint{\theta}}
\]
which is particularly useful when \(f\) contains the expression \(x^2 + y^2 \) since \(x^2 + y^2 = r^2 \).

\subsection{Green's theorem}
For Green's theorem, we introduce the concept of direction in the contour integrals: \(\varointctrclockwise\) is in the positive direction and \(\ointclockwise\) is in the negative direction.

Suppose that \(\mathscr{C}\) is a piecewise-smooth simple closed curve bounding a simply connected region \(R\).
If \(P\), \(Q\), \(\partialderivative{P}{y}\) and \(\partialderivative{Q}{x}\) are continuous on \(R\), then
\[
  \varointctrclockwise_{\mathscr{C}}{\vec{F} \bullet \diffint{\vec{r}}} = \varointctrclockwise_{\mathscr{C}}{P \diffint{x} + Q \diffint{y}} = \iint_R {\partialderivative{Q}{x} - \partialderivative{P}{y} \diffint{A}}
\]

If \(\partialderivative{P}{y} = \partialderivative{Q}{x}\), then the contour integral can be taken on a curve that is more convenient, as long as this curve is fully enclosed in the region bounded by the previous curve.

\subsection{Surface integrals}
\subsubsection*{Surface area}
Let \(f\) be a function for which the first partial derivatives \(f_x \) and \(f_y \) are continuous on a closed region \(R\).
Then the area of the surface over \(R\) is given by
\[
  S = \iint_R {\sqrt{1 + f_x (x, y)^2 + f_y (x, y)^2 }\diffint{A}}
\]
The surface area of a parameterized vector function \(\vec{F}(x, y)\) over \(R\) is
\[
  S = \iint_R {\norm{\partialderivative{\vec{F}}{x} \times \partialderivative{\vec{F}}{y}}\diffint{x}\diffint{y}}
\]

The differential of the surface area is the function function
\[
  \diff{}{S} = \sqrt{1 + f_x (x, y)^2 + f_y (x, y)^2 }\diffint{A}
\]

\subsubsection*{Surface integral}
Let \(G\) be a function of three variables defined over a region of \(\set{R}{3}{}\) containing the surface \(S\).
Then the surface integral of \(G\) over \(S\) is given by
\[
  \iint_S {G(x, y, z)\diffint{S}}
\]

In order to evaluate this surface integral, we project it along a planes:
\begin{description}
  \item[\(xy\)-plane:]
        \begin{multline*}
          \iint_S {G(x, y, z)\diffint{S}} = \\
          \iint_R {G(x, y, f(x, y))\sqrt{1 + f_x (x, y)^2 + f_y (x, y)^2 }\diffint{A}}
        \end{multline*}
  \item[\(xz\)-plane:]
        \begin{multline*}
          \iint_S {G(x, y, z)\diffint{S}} = \\
          \iint_R {G(x, g(x, z), z)\sqrt{1 + g_x (x, z)^2 + g_z (x, z)^2 }\diffint{A}}
        \end{multline*}
  \item[\(yz\)-plane:]
        \begin{multline*}
          \iint_S {G(x, y, z)\diffint{S}} = \\
          \iint_R {G(h(y, z), y, z)\sqrt{1 + h_y (y, z)^2 + h_z (y, z)^2 }\diffint{A}}
        \end{multline*}
\end{description}

The mass \(m\) of a surface represented by \(\rho(x, y, z)\) as the density of this shape at any point is given by
\[
  m = \iint_S {\rho(x, y, z)\diffint{S}}
\]

\subsubsection*{Orientable surface}
A surface \(S\) defined as \(g(x, y, z) = 0\) can be an oriented surface.
The orientation of \(S\) can be found using the normal vector function \(\hat{n}(x, y, z)\), where
\[
  \hat{n} = \frac{\grad{g}}{\norm{\grad{g}}}
\]
If \(S\) is defined by \(z = f(x, y)\), then we define \(g(x, y, z) = z - f(z, y) = 0\) or \(g(x, y, z) = f(z, y) - z = 0\) depending on the orientation of \(S\).

A two-sided surface \(S\) defined by \(z = f(x, y)\) has an upward orientation when the unit normals are directed upward (positive \(\kvec\) components), and it has a downward orientation when the unit normals are direct downward (negative \(\kvec\) components).

\subsubsection*{Integrals of vector fields}
If \(\vec{v} = (P(x, y, z), Q(x, y, z), R(x, y, z))\) is the velocity field of a fluid, then the total of a fluid passing through \(S\) per unit of time is called the flux of \(\vec{v}\) through \(S\) and is given by
\[
  \text{flux } = \iint_S {\vec{v} \bullet \hat{n}\diffint{S}}
\]
In the case of a closed surface \(S\):
\begin{itemize}
  \item If \(\hat{n}\) is the outer normal, the surface integral gives the volume of fluid flowing out through \(S\) per unit of time
  \item If \(\hat{n}\) is the inner normal, the surface integral gives the volume of fluid flowing in through \(S\) per unit of time
\end{itemize}

\subsection{Stokes' theorem}
Let \(S\) be a piecewise-smooth orientable surface bounded by a piecewise-smooth simple closed curve \(\mathscr{C}\).
Let \(\vec{F} = (P(x, y, z), Q(x, y, z), R(x, y, z))\) be a vector field for which \(P\), \(Q\) and \(R\) are continuous and have continuous first partial derivatives in a region of \(\set{R}{3}{}\) containing \(S\).
If \(\mathscr{C}\) is traversed in the positive direction, then
\begin{align*}
  \ointclockwise_{\mathscr{C}}{\vec{F} \bullet \diffint{\vec{r}}} & = \ointclockwise_{\mathscr{C}}{\vec{F} \bullet \hat{T}\diffint{S}} \\
                                                                  & = \iint_S {\left( \curl{F} \right) \bullet \hat{n}\diffint{S}}
\end{align*}
where \(\hat{n}\) is a unit normal to \(S\) in the direction of the orientation of \(S\).

\subsubsection*{Curl interpretation}
The curl of \(\vec{F}\) is the circulation of \(\vec{F}\) per unit of area:
\[
  \left( \curl{F} \bullet \hat{n} \right) = \limit{r}{0}{\frac{1}{A_r }\ointclockwise_{\mathscr{C}_r }{\vec{F} \bullet \diffint{\vec{r}}}}
\]

\subsection{Triple integral}
Let \(F\) be a function of three variables defined on a closed region \(R\) of \(\set{R}{3}{}\).
Then the triple integral of \(F\) over \(R\) is given by
\[
  \iiint_D {F(x, y, z) \diffint{V}} = \iiint_D {F(x, y, z) \diffint{x} \diffint{y} \diffint{z}}
\]

\subsubsection*{Computation}
If the region \(D\) is bounded above by the graph of \(z = f_1 (x, y)\) and bounded below by the graph of \(z = f_2 (x, y)\), then it can be shown that the triple integral can be expressed as a double integral of a partial integral:
\begin{multline*}
  \iiint_D {F(x, y, z) \diffint{V}} = \iint_R \left[ \int_{f_1 (x, y)}^{f_2 (x, y)}{F(x, y, z) \diffint{z}} \right] \diffint{A}                            \\
  = \int_a ^b {\int_{g_1 (x)}^{g_2 (x)}{\int_{f_1 (x, y)}^{f_2 (x, y)}{F(x, y, z) \diffint{z}}\diffint{y}}\diffint{x}}
\end{multline*}

\subsubsection*{Applications}
\paragraph*{Volume}
If \(F(x, y, z) = 1\), then the volume of the solid \(D\) is
\[
  V = \iiint_D {\diffint{V}}
\]

\paragraph*{Mass}
If \(\rho(x, y, z)\) is density, then the mass of the solid \(D\) is given by
\[
  m = \iiint_D {\rho(x, y, z) \diffint{V}}
\]

\paragraph*{First moments}
The first moments of the solid about the coordinate planes indicated by the subscripts are given by
\begin{align*}
  M_{xy} & = \iiint_D {z\rho(x, y, z) \diffint{V}} \\
  M_{xz} & = \iiint_D {y\rho(x, y, z) \diffint{V}} \\
  M_{yz} & = \iiint_D {x\rho(x, y, z) \diffint{V}}
\end{align*}

\paragraph*{Center of mass}
The coordinates of the center of mass of \(D\) are given by
\begin{align*}
  \bar{x} = \frac{M_{yz}}{m} &  & \bar{y} = \frac{M_{xz}}{m} &  & \bar{z} = \frac{M_{xy}}{m}
\end{align*}

\paragraph*{Moments of inertia}
The moments of inertia of \(D\) about the coordinate axes indicated by the subscripts are given by
\begin{align*}
  I_x & = \iiint_D {(y^2 + z^2 )\rho(x, y, z) \diffint{V}} \\
  I_y & = \iiint_D {(x^2 + z^2 )\rho(x, y, z) \diffint{V}} \\
  I_z & = \iiint_D {(x^2 + y^2 )\rho(x, y, z) \diffint{V}}
\end{align*}

\paragraph*{Radius of gyration}
If \(I\) is a moment of inertia of the solid about a given axis, then the radius of gyration is
\[
  R_g = \sqrt{\frac{I}{m}}
\]

\subsubsection*{Cylindrical coordinates}
The cylindrical coordinate system combines the polar description of a point in the plane with cartesian description of the \(z\)-component of a point in space.
To transform the point \(P = (x, y, z)\) in cartesian coordinates from its cylindrical coordinates \(P = (r, \theta, z)\), we use
\begin{align*}
  x = r\cos\theta &  & y = r\sin\theta &  & z = z
\end{align*}
Conversely, to get the cartesian coordinates from cylindrical coordinates, we use
\begin{align*}
  r = \sqrt{x^2 + y^2 } &  & \theta = \arctan{\frac{y}{x}} &  & z = z
\end{align*}

In order to obtain the triple integral using cylindrical coordinates, we use \(\diff{}{A} = r\diffint{r}\diffint{\theta}\):
\begin{multline*}
  \iiint_D {F(r, \theta, z) \diffint{V}} = \iint_R \left[ \int_{f_1 (r, \theta)}^{f_2 (r, \theta)}{F(r, \theta, z) \diffint{z}} \right] \diffint{A} \\
  = \int_{\alpha}^{\beta}{\int_{g_1 (\theta)}^{g_2 (\theta)}{\int_{f_1 (r, \theta)}^{f_2 (r, \theta)}{F(r, \theta, z) r \diffint{z}}\diffint{r}}\diffint{\theta}}
\end{multline*}

\subsubsection*{Spherical coordinates}
The spherical coordinates of a point \(P\) are given by the ordered triple \((\rho, \phi, \theta)\), where \(\rho\) is the distance from the origin to \(P\), \(\phi\) is the angle between the positive \(z\)-axis and the vector \(\Vec{OP}\), and \(\theta\) is the angle measured from the positive \(x\)-axis to the vector \(\Vec{OP}\) projected onto the \(xy\)-plane.
To transform \(P = (x, y, z)\) in cartesian coordinates from spherical coordinates \(P = (\rho, \phi, \theta)\) are
\begin{align*}
  x = \rho \sin\phi \cos\theta &  & y = \rho \sin\phi \sin\theta &  & z = \rho \cos\phi
\end{align*}
Conversely, to get the cartesian coordinates from spherical coordinates, we use
\begin{align*}
  r = \sqrt{x^2 + y^2 + z^2 } &  & \theta = \arctan{\frac{y}{x}} &  & \phi = \arccos{\frac{z}{\rho}}
\end{align*}
Finally, to transform from spherical coordinates \((\rho, \phi, \theta)\) to cylindrical coordinates \((r, \theta, z)\), we use:
\begin{align*}
  r = \rho \sin\phi &  & \theta = \theta &  & z = \rho \cos\phi
\end{align*}

In order to obtain the triple integral using spherical coordinates, we use \(\diff{}{V} = \rho^2 \sin\phi \diffint{\rho}\diffint{\phi}\diffint{\theta}\):
\begin{multline*}
  \iiint_D {F(\rho, \phi, \theta) \diffint{V}} = \\
  \int_{\alpha}^{\beta}{\int_{g_1 (\theta)}^{g_2 (\theta)}{\int_{f_1 (\phi, \theta)}^{f_2 (\phi, \theta)}{F(\rho, \phi, \theta) \rho^2 \sin\phi  \diffint{\rho}}\diffint{\phi}}\diffint{\theta}}
\end{multline*}

\subsection{Divergence theorem}
Let \(\vec{F}(x, y) = (P(x, y), Q(x, y))\) be a \(\set{R}{2}{}\) vector field.
Then, another vector form of Green's theorem is
\[
  \varointctrclockwise_{\mathscr{C}}{\vec{F} \bullet \hat{n} \diffint{s}} = \iint_R {\divergence{F} \diffint{A}}
\]

The generalization of this to \(\set{R}{3}{}\), where  \(\vec{F}(x, y, z) = (P(x, y, z), Q(x, y, z), R(x, y, z))\) is a vector field, then
\[
  \iint_S {\vec{F} \bullet \hat{n} \diffint{S}} = \iiint_D {\divergence{F} \diffint{A}}
\]

For the surface between two spheres, we have
\[
  \iint_{S_a}{\vec{F} \bullet \hat{n} \diffint{S}} + \iint_{S_b}{\vec{F} \bullet \hat{n} \diffint{S}} = \iiint_D {\divergence{F} \diffint{A}}
\]
where \(\hat{n}\) points outward from \(D\), i.e. \(\hat{n}\) points away from the origin on \(S_a\) and \(\hat{n}\) points toward the origin on \(S_b\).

\subsubsection*{Physical interpretation of divergence}
The divergence can be interpreted as the flux per unit of volume at a specific point:
\[
  \divergence{F}(P_0) = \limit{r}{0}{\frac{1}{V_r}\iint_{S_r}{\vec{F} \bullet \hat{n} \diffint{S}}}
\]

\subsubsection*{Continuity equation}
Using the divergence theorem, we can find the equation of continuity for fluid flow:
\[
  \partialderivative{\rho}{t} + \mathrm{div\,}\left( \rho\vec{F} \right) = 0
\]


\subsection{Change of variables in multiple integrals}
For simple integrals, the change of variable is as follow: let \(x = g(u)\), then
\[
  \int_a ^b {f(x) \diffint{x}} = \int_c ^d {f(g(u)) g'(u) \diffint{u}}
\]
where \(a = g(c)\) and \(b = g(d)\).
In multivariable calculus, we use the Jacobian \(J(u)\) instead of \(g'(u)\).

Let \(x = f(u, v)\) and \(y = g(u, v)\), then the change of variable in a double integral is of the form
\[
  \iint_R {F(x, y) \diffint{A}} = \iint_S {F(f(u, v), g(u, v)) J(u, v) \diffint{A}}
\]

Method for change of variable from \((x, y)\) to \((u, v) = (f(x, y),g(x, y))\):
\begin{enumerate}
  \item Find the region image \(S\) in terms of \((u, v)\) based on the region \(R\), which include functions and boundary points
  \item Compute the non-zero Jacobian \(J(u, v)\), which is defined as
        \[
          \partialderivative{(x, y)}{(u, v)} =
          \begin{vmatrix}
            \partialderivative{x}{u} & \partialderivative{x}{v} \\
            \partialderivative{y}{u} & \partialderivative{y}{v}
          \end{vmatrix}
          = \partialderivative{x}{u} \partialderivative{y}{v}  - \partialderivative{y}{u} \partialderivative{x}{v}
        \]
  \item If \(F\) is continuous on \(R\), then
        \[
          \iint_R {F(x, y) \diffint{A}} = \iint_S {F(f(u, v), g(u, v)) \abs{\partialderivative{(x, y)}{(u, v)}} \diffint{A}}
        \]
\end{enumerate}

For triple integrals, let \(x = f(u, v, w)\), \(y = g(u, v, w)\) and \(z = h(u, v, w)\), then
\begin{multline*}
  \iiint_D {F(x, y, z) \diffint{V}} = \\
  \iiint_E {F(f(u, v, w), g(u, v, w), h(u, v, w)) \abs{\partialderivative{(x, y, z)}{(u, v, w)}} \diffint{V}}
\end{multline*}
where
\[
  \partialderivative{(x, y, z)}{(u, v, w)} =
  \begin{vmatrix}
    \partialderivative{x}{u} & \partialderivative{x}{v} & \partialderivative{x}{w} \\
    \partialderivative{y}{u} & \partialderivative{y}{v} & \partialderivative{y}{w} \\
    \partialderivative{z}{u} & \partialderivative{z}{v} & \partialderivative{z}{w}
  \end{vmatrix}
\]
\end{document}