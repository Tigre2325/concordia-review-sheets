\documentclass[10pt, twocolumn]{article}

%%%%%%%%%%%%%%%%%%%%%%%%%%%%%%%%%%%%%%%%%%%%%%%%%%%%%%%%%%%%%%%%%%%%%%%%%%%%%%%
%%%% Cover page
\title{PHYS 205: Electricity and Magnetism}
\date{\today}
\author{Anthony Bourboujas}

\makeatletter
\let\Title\@title
\let\Author\@author
\let\Date\@date
\makeatother

%%%%%%%%%%%%%%%%%%%%%%%%%%%%%%%%%%%%%%%%%%%%%%%%%%%%%%%%%%%%%%%%%%%%%%%%%%%%%%%
%%%% Preamble
%%%%%%%%%%%%%%%%%%%%%%%%%%%%%%%%%%%%%%%%%%%%%%%%%%%%%%%%%%%%%%%%%%%%%%%%%%%%%%%
%%%% Packages
\usepackage[utf8x]{inputenc} % Accept different input encodings
\usepackage[T1]{fontenc} % Standard package for selecting font encodings
\usepackage{lmodern} % Font name; classic: lmodern
\usepackage[english]{babel} % Multilingual support for LaTeX
% \usepackage{abstract} % Control the typesetting of the abstract environment
\usepackage{amsmath} % AMS mathematical facilities for LaTeX
\usepackage{amssymb} % TeX fonts from the American Mathematical Society
\usepackage{amsthm} % Typesetting theorems (AMS style)
\usepackage{array} % Extending the array and tabular environments
% \usepackage[backend=biber,style=ieee,sorting=none]{biblatex}
\usepackage{bold-extra} % Use bold small caps and typewriter fonts
\usepackage{cellspace} % Ensure minimal spacing for table cells
\usepackage{chemformula} % Command for typesetting chemical formulas and reactions
% \usepackage{colortbl} % Add colour to LaTeX tables
\usepackage{comment} % Selectively include/exclude portions of text
\usepackage{csquotes} % Context sensitive quotation facilities
% \usepackage[en-US,showdow]{datetime2} % Formats for dates, times and time zones
% \usepackage{diagbox} % Table heads with diagonal lines
\usepackage{enumitem} % Control layout of itemize, enumerate, description
\usepackage{esint} % Extended set of integrals for Computer Modern
\usepackage{graphicx} % Enhanced support for graphics
% \usepackage{listings} % Typeset source code listings using LaTeX
% \usepackage{lipsum} % Easy access to the Lorem Ipsum dummy text
\usepackage{mathrsfs} % Support for using RSFS fonts in maths
% \usepackage{matlab-prettifier} % Pretty-print Matlab source code
\usepackage{moreverb} % Extended verbatim
\usepackage{multicol} % Intermix single and multiple columns
\usepackage{multirow} % Create tabular cells spanning multiple rows
% \usepackage{pgfplots} % Plots
% \usepackage{pgfplotstable} % Loads, rounds, format and post-processes numerical tables (generates table from CSV)
% \usepackage{pdfpages} % Include PDF document in LaTeX
% \usepackage{rotating} % Rotation tools, including rotated full-page floats with sidewaysfigure
\usepackage[scr]{rsfso} % A mathematical calligraphic font based on rsfs
\usepackage{setspace} % Set space between lines
\usepackage{soul} % Hyphenation for letterspacing, underlining, and more
\usepackage{threeparttable} % Tables with captions and notes all the same width
% \usepackage{verbatim} % Reimplementation of and extensions to LaTeX verbatim
\usepackage{wrapfig} % Produces figures which text can flow around
\usepackage{xcolor} % Driver-independent color extensions for LaTeX
\usepackage{xurl} % Verbatim with URL-sensitive line breaks, allow URL breaks at any alphanumerical character

%%%%%%%%%%%%%%%%%%%%%%%%%%%%%%%%%%%%%%%%%%%%%%%%%%%%%%%%%%%%%%%%%%%%%%%%%%%%%%%
%%%% Lengths
% 1cm = 10mm = 28pt = 1/2.54in
% 1ex = height of a lowercase 'x' in the current font
% 1em = width of an uppercase 'M' in the current font

%%%% Spacing in math mode
% \!                         = -3/18em
% \,                         = 3/18em
% \:                         = 4/18em
% \;                         = 5/18em
% \ (space after backslash!) = space in normal text
% \quad                      = 1em
% \qquad                     = 2em

% \setlength{\baselineskip}{1em} % Vertical distance between lines in a paragraph
% \renewcommand{\baselinestretch}{1.0} % A factor multiplying \baslineskip
\setlength{\columnsep}{0.75cm} % Distance between columns
% \setlength{\columnwidth}{} % The width of a column
\setlength{\columnseprule}{1pt} % The width of the vertical ruler between columns
% \setlength{\evensidemargin}{} % Margin of even pages, commonly used in two-sided documents such as books
% \setlength{\linewidth}{} % Width of the line in the current environment.
% \setlength{\oddsidemargin}{} % Margin of odd pages, commonly used in two-sided documents such as books
% \setlength{\paperwidth}{} % Width of the page
% \setlength{\paperheight}{} % Height of the page
\setlength{\parindent}{0cm} % Paragraph indentation
\setlength{\parskip}{6pt} % Vertical space between paragraphs
% \setlength{\tabcolsep}{} % Separation between columns in a table (tabular environment)
% \setlength{\textheight}{} % Height of the text area in the page
% \setlength{\textwidth}{} % Width of the text area in the page
% \setlength{\topmargin}{} % Length of the top margin
\setlist{
  %%%% Vertical spacing
  topsep = 0pt,
  partopsep = 0pt,
  parsep = 0pt,
  itemsep = 0pt,
  %%%% Horizontal spacing
  leftmargin = 0.5cm,
  rightmargin = 0cm,
  % listparindent = 0cm,
  % labelwidth = 0cm,
  % labelsep = 0cm,
  % itemindent = 0cm
}
\addtolength{\cellspacetoplimit}{2pt}
\addtolength{\cellspacebottomlimit}{2pt}

%%%%%%%%%%%%%%%%%%%%%%%%%%%%%%%%%%%%%%%%%%%%%%%%%%%%%%%%%%%%%%%%%%%%%%%%%%%%%%%
%%%% Page layout
\usepackage{layout} % View the layout of a document
\usepackage{geometry} % Flexible and complete interface to document dimensions
% 1cm = 10mm = 28pt = 1/2.54in
% ex = height of a lowercase 'x' in the current font
% em = width of an uppercase 'M' in the current font
\geometry{
  a4paper,
  top         = 1cm,
  bottom      = 1cm,
  left        = 1.5cm,
  right       = 1.5cm,
  includehead = true,
  includefoot = true,
  landscape   = false, % Paper orientation
  twoside     = false,
}
% \geometry{showframe} % Show paper outline for the text area and page

%%%%%%%%%%%%%%%%%%%%%%%%%%%%%%%%%%%%%%%%%%%%%%%%%%%%%%%%%%%%%%%%%%%%%%%%%%%%%%%
%%%% Header and footer style
\usepackage{fancyhdr} % Extensive control of page headers and footers in LaTeX
\pagestyle{fancy}
% Options: \leftmark (chapter title), \rightmark(section title), \thepage (page number), \thechapter(chapter number), \thesection (section number)
\lhead{\thetitle}
\chead{}
\rhead{}
\lfoot{}
\cfoot{\thepage}
\rfoot{}

%%%%%%%%%%%%%%%%%%%%%%%%%%%%%%%%%%%%%%%%%%%%%%%%%%%%%%%%%%%%%%%%%%%%%%%%%%%%%%%
%%%% URL insertion settings
\usepackage{hyperref} % Extensive support for hypertext in LaTeX
\definecolor{black}{RGB}{0, 0, 0} % rgb(0, 0, 0)
\definecolor{blue}{RGB}{0, 0, 255} % rgb(0, 0, 255)
\hypersetup{
  % unicode            = true,
  pdftitle           = {\thetitle},
  pdfauthor          = {\theauthor},
  % pdfsubject       = {},
  %%%% Reference
  % bookmarks          = true,
  bookmarksnumbered  = true,
  bookmarksopen      = true, % Open the bookmarks
  bookmarksopenlevel = 2, % Open until 1 level (section)
  %%%% Bookmarks
  breaklinks         = true,
  pdfborder          = {0 0 0},
  % backref            = true, % Add links into bibliography
  % pagebackref        = true,
  % hyperindex         = true, % Add links into index
  %%%% Color
  colorlinks         = true,
  linkcolor          = black, % Internal links color
  citecolor          = black,
  urlcolor           = blue, % Hyperlinks color
  filecolor          = black,
}

\usepackage{varioref} % Intelligent page reference
\usepackage[capitalise,noabbrev]{cleveref}
\usepackage{prettyref} % Make label references "self-identity" with \prettyref{#1}
\newrefformat{cha}{chapter \textbf{\nameref{#1}} \vpageref{#1}} % {chapter \textbf{\nameref{#1}} on page \pageref{#1}}
\newrefformat{sec}{section \textbf{\nameref{#1}} \vpageref{#1}} % {section \textbf{\nameref{#1}} on page \pageref{#1}}
% \newrefformat{fig}{\vref{#1}} % {Figure \ref{#1} on page \pageref{#1}}
% \newrefformat{tab}{\vref{#1}} % {Table \ref{#1} on page \pageref{#1}}
% \newrefformat{eqn}{\vref{#1}}
% \newrefformat{lis}{\emph{\nameref{#1}} \vpageref{#1}}

%%%%%%%%%%%%%%%%%%%%%%%%%%%%%%%%%%%%%%%%%%%%%%%%%%%%%%%%%%%%%%%%%%%%%%%%%%%%%%%
%%%% Physics units settings
% Dependencies
\usepackage{booktabs} % Publication quality tables in LaTeX
\usepackage{caption} % Customizing captions in floating environments
\usepackage{helvet} % Load Helvetica, scaled
\usepackage{cancel} % Place lines through maths formulae

\usepackage{siunitx} % A comprehensive (SI) units package
\sisetup{
  exponent-product     = \cdot, % Symbol between number and power of ten
  group-minimum-digits = 5, % Number of digits when 3 digits separation appear
  % inter-unit-product   = \cdot, % Symbol between units (when several units are used)
  output-complex-root  = \ensuremath{i}, % How i math should be seen
  % prefixes-as-symbols  = false, % Translate prefixes (kilo, centi, milli, micro,...) into a power of ten
  separate-uncertainty = true, % Write uncertainty with +-
  scientific-notation  = engineering,
}

%%%%%%%%%%%%%%%%%%%%%%%%%%%%%%%%%%%%%%%%%%%%%%%%%%%%%%%%%%%%%%%%%%%%%%%%%%%%%%%
%%%% Theorems and proofs
\numberwithin{equation}{section}
% \makeatletter
% \g@addto@macro\th@remark{\thm@headpunct{:}}
% \makeatother
\theoremstyle{remark}
\newtheorem*{example}{Example}
\newtheorem*{remark}{Remark}

%%%%%%%%%%%%%%%%%%%%%%%%%%%%%%%%%%%%%%%%%%%%%%%%%%%%%%%%%%%%%%%%%%%%%%%%%%%%%%%
%%%% User-defined environments
% Remove the space before the enumerate and itemize environments
\let\oldenumerate\enumerate % Keep a copy of \enumerate (or \begin{enumerate})
\let\endoldenumerate\endenumerate % Keep a copy of \endenumerate (or \end{enumerate})
\renewenvironment{enumerate}{
  \begin{oldenumerate}
    \vspace{-6pt}
    }{
  \end{oldenumerate}
}

\let\olditemize\itemize % Keep a copy of \itemize (or \begin{itemize})
\let\endolditemize\enditemize % Keep a copy of \enditemize (or \end{itemize})
\renewenvironment{itemize}{
  \begin{olditemize}
    \vspace{-6pt}
    }{
  \end{olditemize}
}

\let\olddescription\description % Keep a copy of \description (or \begin{description})
\let\endolddescription\enddescription % Keep a copy of \enddescription (or \end{description})
\renewenvironment{description}{
  \begin{olddescription}
    \vspace{-6pt}
    }{
  \end{olddescription}
}

%%%%%%%%%%%%%%%%%%%%%%%%%%%%%%%%%%%%%%%%%%%%%%%%%%%%%%%%%%%%%%%%%%%%%%%%%%%%%%%
%%%% User-defined commands
\newcommand{\Romannumeral}[1]{\MakeUppercase{\romannumeral #1}} % Capital roman numbers
% \newcommand{\gui}[1]{\og #1 \fg{}} % French quotation marks
\renewcommand{\thefootnote}{[\arabic{footnote}]}

%%% Figure command
%% Include SVG files
\newcommand{\executeiffilenewer}[3]{
  \ifnum\pdfstrcmp{\pdffilemoddate{#1}}
    {\pdffilemoddate{#2}}>0
    {\immediate\write18{#3}}\fi
}
\newcommand{\includesvg}[1]{
  \executeiffilenewer{#1.svg}{#1.pdf}
  {
    % Inkscape must be installed in PATH and the user must include '--shell-escape' in the build arguments
    inkscape #1.svg --export-type=pdf --export-latex
  }
  \input{#1.pdf_tex}
}

%%% Math commands
%% Tables (requires cellspace package)
\newcolumntype{L}{>{\(\displaystyle}Cl<{\)}} % Column type for left-aligned math column
\newcolumntype{D}{>{\(\displaystyle}Cc<{\)}} % Column type for centered math column

%% Functions
\newcommand{\constant}{\mathrm{constant}} % Constant
\newcommand{\abs}[1]{\left| #1 \right|} % Absolute function
\newcommand{\erf}[1]{\mathrm{erf} \left( #1 \right)} % Error function
\newcommand{\erfc}[1]{\mathrm{erfc} \left( #1 \right)} % Complementary error function
\newcommand{\unitstep}[1]{\,\mathcal{U}\left( #1 \right)} % Unit step function
\newcommand{\diracdelta}[2]{\,\delta_{#1}\left( #2 \right)} % Dirac delta function


%% Derivatives and integrals
\newcommand{\diff}[2]{\mathrm{d}^{#1} #2} % Letter 'd' of differentials
\newcommand{\diffint}[1]{\,\diff{}{#1}} % Differential with a space for integrals
\newcommand{\derivative}[2]{\frac{\diff{}{#1}}{\diff{}{#2}}} % Derivative
\newcommand{\nderivative}[3]{\frac{\diff{#1}{#2}}{\diff{}{#3^{#1}}}} % Derivative of degree n
\newcommand{\partialderivative}[2]{\frac{\partial #1}{\partial #2}} % Partial derivative
\newcommand{\npartialderivative}[3]{\frac{\partial^{#1} #2}{\partial #3^{#1}}} % Partial derivative of degree n
\newcommand{\direcderivative}[2]{D_{\vec{#1}}\,#2} % Directional derivative

\newcommand{\Laplace}[1]{\mathcal{L}\left\{ #1 \right\}} % Laplace transform notation
\newcommand{\invLaplace}[1]{\mathcal{L}^{-1}\left\{ #1 \right\}} % Inverse Laplace transform notation

%% Set
\newcommand{\set}[3]{\mathbb{#1}_{#2}^{#3}} % Set of numbers
\newcommand{\integerset}{\mathbb{Z}} % Set of integer numbers (compatibility)
\newcommand{\realset}{\mathbb{R}} % Set of real numbers (compatibility)

%% Limits
\newcommand{\limit}[3]{\lim_{#1 \to #2}{#3}} % Limit from a point to another
\newcommand{\rlimit}[3]{\lim_{#1 \to #2^{+}}{#3}} % Right limit from a point to another
\newcommand{\llimit}[3]{\lim_{#1 \to #2^{-}}{#3}} % Left imit from a point to another
\newcommand{\modulus}[1]{\,\left[ #1 \right]} % Modulus notation

%% Vectors
\newcommand{\ivec}{\hat{\mathrm{i}}} % i vector
\newcommand{\jvec}{\hat{\mathrm{j}}} % j vector
\newcommand{\kvec}{\hat{\mathrm{k}}} % k vector
\renewcommand{\Vec}[1]{\overrightarrow{#1}} % Vector notation for expression with more than one letter
\newcommand{\norm}[1]{\left\| #1 \right\|} % Norm notation for expression with just one letter
\newcommand{\normvec}[1]{\left\| \vec{#1} \right\|} % Norm notation for expression with just one letter
\newcommand{\Normvec}[1]{\left\| \Vec{#1} \right\|} % Norm notation for expression with more than one letter
\newcommand{\comp}[2]{\mathrm{comp}_{\vec{#2}}\vec{#1}} % Components
\newcommand{\proj}[2]{\mathrm{proj}_{\vec{#2}}\vec{#1}}
\newcommand{\grad}[1]{\vec{\nabla}#1} % Gradient notation
\newcommand{\frames}[2]{\left( #1 \right)_{#2}} % Frame definition

\newcommand{\curl}[1]{\mathrm{curl}\,\vec{#1}} % Curl of a vector field
\newcommand{\divergence}[1]{\mathrm{div}\,\vec{#1}} % Divergence of a vector field


%%%%%%%%%%%%%%%%%%%%%%%%%%%%%%%%%%%%%%%%%%%%%%%%%%%%%%%%%%%%%%%%%%%%%%%%%%%%%%%
%%%% Table of contents
% \renewcommand{\contentsname}{Table of contents title} % Change the table of contents title

%%%%%%%%%%%%%%%%%%%%%%%%%%%%%%%%%%%%%%%%%%%%%%%%%%%%%%%%%%%%%%%%%%%%%%%%%%%%%%%
%%%% Beginning of the document
\begin{document}
\maketitle % Insert the cover page
% \tableofcontents
% \layout % Show a drawing of page layout

\renewcommand{\abstractname}{Constants} % Change the abstract title
\begin{abstract}
  \vspace{-20pt}
  \begin{align*}
    c                     & = \SI{3.00 e8}{\meter\per\second}                                                         \\
    N_A                   & = \SI{6.02 e23}{\per\mole}                                                                \\
    m_{\mathrm{proton}}   & = \SI{1.672 e-27}{\kilogram}                                                              \\
    m_{\mathrm{neutron}}  & = \SI{1.674 e-27}{\kilogram}                                                              \\
    m_{\mathrm{electron}} & = \SI{9.11 e-31}{\kilogram}                                                               \\
    e                     & = \SI{1.602 e-19}{\coulomb}                                                               \\
    k_e                   & = \frac{1}{4 \pi \epsilon_0 } = \SI{8.8976 e9}{\newton\metre\squared\per\coulomb\squared} \\
    \epsilon_0            & = \SI{8.8542 e-12}{\coulomb\squared\per\newton\per\metre\squared}                         \\
    \mu_0                 & = \SI{4 \pi e-7}{\tesla\metre\per\ampere}
  \end{align*}
\end{abstract}

\setlength{\abovedisplayskip}{2pt} % Space above displayed equations
\setlength{\belowdisplayskip}{2pt} % Space below displayed equations

\section{Electric forces and fields}
Conservation of charge:
\[
  \sum {q_{\mathrm{initial}}} = \sum{}^{}{q_{\mathrm{final}}}
\]

Quantization of charge:
\[
  q = \left( N_{\mathrm{proton}} - N_{\mathrm{electron}} \right)e
\]

Electric force between two point-charges:
\[
  F_E = k_e \frac{\left| q_1 \right|\left| q_2 \right|}{r^2 }
\]

Electric field:
\begin{itemize}
  \item Point-charge:
        \[
          \vec{E} = \frac{\vec{F}_E }{q_0 } = k_e \frac{q}{r^2 } \hat{r}
        \]
  \item Uniform electric field:
        \[
          E = \frac{\left| \sigma \right|}{2\epsilon_0 }
        \]
  \item Two parallel plates:
        \[
          E = \frac{\left| \sigma \right|}{\epsilon_0 }
        \]
\end{itemize}

\section{Continuous charge distribution}
Charge densities:
\begin{itemize}
  \item Linear charge density:
        \[
          \lambda = \frac{Q}{l}
        \]
  \item Area charge density:
        \[
          \sigma = \frac{Q}{A}
        \]
  \item Volume charge density (insulator):
        \[
          \rho = \frac{Q}{V}
        \]
  \item Varying charge density:
        \[
          Q_{\mathrm{tot}} = \int{\diff{}{q}} = \int{\lambda\left( x \right)\diff{}{x}}
        \]
\end{itemize}

Electric field of a continuous charge distribution:
\[
  \vec{E} = \int{\diff\vec{E}} = k_e \int{\frac{\diff{}{q}}{r^2 }\hat{r}}
\]

Electric flux:
\begin{itemize}
  \item Constant electric field:
        \[
          \Phi_E = \vec{E} \bullet \vec{A} = EA\cos{\theta}
        \]
  \item Gauss's law:
        \[
          \Phi_E = \oint{\vec{E} \bullet \diff\vec{A}} = \frac{q_{\mathrm{in}}}{\epsilon_0 }
        \]
\end{itemize}

\section{Electrical potential}
Work:
\begin{itemize}
  \item Uniform electric field:
        \[
          W_{1 \rightarrow 2} = q\vec{E} \bullet \Delta\vec{r} = qE\Delta r\cos{\theta}
        \]
  \item Non-uniform electric field:
        \[
          W_{1 \rightarrow 2} = q\int_1 ^2 {\vec{E} \bullet \diff\vec{s}}
        \]
\end{itemize}

Electric potential energy:
\begin{itemize}
  \item Uniform electric field:
        \[
          \Delta U = - W_{1 \rightarrow 2} = - q\vec{E} \bullet \Delta\vec{r} = - \mathrm{qE}\Delta r\cos{\theta}
        \]
  \item Non-uniform electric field:
        \[
          \Delta U = - W_{1 \rightarrow 2} = - q\int_1 ^2 {\vec{E} \bullet \diff\vec{s}}
        \]
  \item Two point-charges:
        \[
          U = k_e \frac{{q_1 q}_2 }{r}
        \]
\end{itemize}
Potential:
\begin{itemize}
  \item Point-charge:
        \[
          V = \frac{U}{q_0 } = k_e \frac{q}{r}
        \]
  \item Potential difference of a point charge:
        \[
          \Delta V = \frac{\Delta U}{q_0 }
        \]
  \item Uniform electric field:
        \[
          \Delta V = - \vec{E} \bullet \Delta\vec{r} = - E\Delta r\cos{\theta}
        \]
  \item Non-uniform electric field:
        \[
          \Delta V = - \int_1 ^2 {\vec{E} \bullet \diff\vec{s}}
        \]
        \begin{itemize}
          \item Electric field from potential:
                \[
                  E = - \derivative{V}{s}
                \]
        \end{itemize}
  \item Continuous charge distribution:
        \[
          V = \int{\diff{}{V}} = k_e \int{\frac{\diff{}{q}}{r}}
        \]
\end{itemize}

\section{Capacitance and dielectric}
Capacitance:
\[
  C = \frac{Q}{\Delta V_C }
\]
\begin{itemize}
  \item Parallel plate capacitor:
        \[
          C = \frac{\epsilon_0 A}{d}
        \]
\end{itemize}
Capacitor in circuits:
\begin{itemize}
  \item Parallel:
        \[
          C_{\mathrm{eq}} = C_1 + C_2 + C_3 + \cdots = \sum_i {C_i }
        \]
  \item Series:
        \[
          C_{\mathrm{eq}} = \frac{1}{\frac{1}{C_1 } + \frac{1}{C_2 } + \frac{1}{C_3 } + \cdots} = \frac{1}{\sum_i {\frac{1}{C_i }}} = \left( \sum_i {\frac{1}{C_i }} \right)^{-1}
        \]
\end{itemize}
Energy related to a capacitor:
\begin{itemize}
  \item Energy stored in a capacitor:
        \[
          U_C = \frac{Q^2 }{2C} = \frac{1}{2}Q\Delta V_C = \frac{1}{2}C\left( \Delta V_C \right)^2
        \]
  \item Energy stored in the electric field between the two plates:
        \[
          U_C = \frac{1}{2}\epsilon_0 AdE^2
        \]
  \item Energy density of the electric field:
        \[
          u_E = \frac{U_C }{Ad} = \frac{1}{2}\epsilon_0 E^2
        \]
\end{itemize}
Dielectric:
\begin{align*}
  E_1    & = E_0 - E_{\mathrm{induced}} \\
  \kappa & = \frac{E_0 }{E_1 }          \\
  C_1    & = \kappa C_0
\end{align*}

Electrostatic breakdown:
\[
  V = E_{\mathrm{max}}d
\]

\section{Current and resistance}
Current:
\[
  I = \derivative{Q}{t}
\]

Drift speed (speed of the electron in a conductor):
\[
  v_d = \frac{I}{Aen_e }
\]

Ohm's law:
\[
  J = \frac{I}{A} = v_d en_e
\]

\begin{itemize}
  \item Conductivity of a material:
        \[
          \sigma = \frac{J}{E}
        \]
  \item Resistivity of a material:
        \[
          \rho = \frac{1}{\sigma} = \frac{E}{J}
        \]
  \item Resistance:
        \[
          R = \frac{\rho d}{A} = \frac{d}{\sigma A}
        \]
  \item Most known:
        \[
          \Delta V = RI
        \]
\end{itemize}

Resistor in circuits:
\begin{itemize}
  \item Parallel:
        \[
          R_{\mathrm{eq}} = \frac{1}{\frac{1}{R_1 } + \frac{1}{R_2 } + \frac{1}{R_3 } + \cdots} = \frac{1}{\sum_i {\frac{1}{R_i }}} = \left( \sum_i {\frac{1}{R_i }} \right)^{-1}
        \]
  \item Series:
        \[
          R_{\mathrm{eq}} = R_1 + R_2 + R_3 + \cdots = \sum_i {R_i }
        \]
\end{itemize}

\section{DC circuits}
EMF:
\[
  \varepsilon = \frac{W_{\mathrm{chem}}}{q}
\]
\begin{itemize}
  \item Ideal battery:
        \[
          \Delta V = \varepsilon
        \]
  \item Realistic battery:
        \[
          \Delta V = \varepsilon - IR_{\mathrm{internal}}
        \]
\end{itemize}

Power:
\begin{itemize}
  \item Power output of a DC source:
        \[
          P_{\mathrm{elec}} = \derivative{U}{t} = \derivative{Q}{t}\Delta V = I\Delta V
        \]
  \item Power dissipated in a resistor:
        \[
          P_R = I\Delta V_R = RI^2 = \frac{\left( \Delta V_R \right)^2 }{R}
        \]
\end{itemize}

Kirchhoff's rules:
\begin{itemize}
  \item Junction rule:
        \[
          \sum_{\mathrm{in}}{I_{\mathrm{in}}} = \sum_{\mathrm{out}}{I_{\mathrm{out}}}
        \]
  \item Loop rule:
        \[
          \Delta V_{\mathrm{loop}} = \sum_i {\Delta V_i } = 0
        \]
\end{itemize}

RC circuits:
\begin{itemize}
  \item Time constant:
        \[
          \tau = RC
        \]
  \item Discharging capacitor:
        \begin{itemize}
          \item Charge:
                \[
                  Q\left( t \right) = Q_{\mathrm{max}}e^{-\frac{t}{\tau}} = C\Delta V_C e^{-\frac{t}{\tau}}
                \]
          \item Current:
                \[
                  I\left( t \right) = I_{\mathrm{max}}e^{-\frac{t}{\tau}} = \frac{\Delta V_C }{R}e^{-\frac{t}{\tau}} = \frac{Q_{\mathrm{max}}}{\tau}e^{-\frac{t}{\tau}}
                \]
          \item Potential:
                \[
                  {\Delta V}\left( t \right) = \Delta V_C e^{-\frac{t}{\tau}} = \frac{Q_{\mathrm{max}}}{C}e^{-\frac{t}{\tau}}
                \]
        \end{itemize}
        \columnbreak
  \item Charging capacitor:
        \begin{itemize}
          \item Charge:
                \[
                  Q\left( t \right) = Q_{\mathrm{max}}\left( 1 - e^{- \frac{t}{\tau}} \right) = C\Delta V_C \left( 1 - e^{- \frac{t}{\tau}} \right)
                \]
          \item Current:
                \[
                  I\left( t \right) = I_{\mathrm{max}}e^{- \frac{t}{\tau}} = \frac{\Delta V_C }{R}e^{- \frac{t}{\tau}} = \frac{Q_{\mathrm{max}}}{\tau}e^{- \frac{t}{\tau}}
                \]
          \item Potential:
                \[
                  \Delta V\left( t \right) = \Delta V_C \left( 1 - e^{- \frac{t}{\tau}} \right) = \frac{Q_{\mathrm{max}}}{C}\left( 1 - e^{- \frac{t}{\tau}} \right)
                \]
        \end{itemize}
\end{itemize}

\section{Magnetic fields and forces}
Magnetic force:
\[
  \vec{F}_B = q\vec{v} \times \vec{B}\
\]
Radius of the circle path of a charge:
\[
  r = \frac{mv_{\perp  B}}{qB}
\]
Lorentz force:
\[
  \vec{F} = \vec{F}_E + \vec{F}_B = q\vec{E} + q\vec{v} \times \vec{B}
\]
Force on a current carrying wire:
\[
  \vec{F}_{\mathrm{wire}} = I\vec{L} \times \vec{B}
\]
Torque on a current carrying loop:
\[
  T = I\left\| \vec{A} \times \vec{B} \right\| = IAB\sin{\theta}
\]

\section{Sources of magnetic fields}
Biot-Savart law:
\[
  \vec{B} = \frac{\mu_0 I}{4\pi}\int{\frac{\diff\vec{s} \times \hat{r}}{r^2 }}
\]
\begin{itemize}
  \item Segment of straight wire:
        \[
          B = \frac{\mu_0 I}{4\pi d}\left( \sin{\alpha} + \sin{\beta} \right)
        \]
  \item Infinite wire:
        \[
          B = \frac{\mu_0 I}{2\pi d}
        \]
  \item Circular section of wire:
        \[
          B = \frac{\mu_0 I\theta}{4\pi r}
        \]
\end{itemize}

Ampere's law:
\[
  \oint{\vec{B} \bullet \diff\vec{s} = \mu_0 I_{\mathrm{through}}}
\]
\begin{itemize}
  \item Field strength inside a solenoid:
        \[
          B = \mu_0 nI = \mu_0 \frac{N}{l}I
        \]
\end{itemize}

\section{Faraday's law}
Motional EMF:
\begin{itemize}
  \item Current:
        \[
          I = \frac{vlB}{R}
        \]
  \item Force exerted:
        \[
          F_{\mathrm{pull}} = F_B = \frac{vl^2 B^2 }{R}
        \]
\end{itemize}

\columnbreak
Magnetic flux:
\begin{itemize}
  \item Uniform magnetic field:
        \[
          \Phi_B = \vec{A} \bullet \vec{B} = AB\cos{\theta}
        \]
  \item Non-uniform magnetic field:
        \[
          \Phi_B = \int{\vec{B} \bullet \diff\vec{A}}
        \]
\end{itemize}

Faraday's law:
\begin{itemize}
  \item Single loop induced EMF:
        \[
          \Delta V_{\mathrm{induced}} = \left| \derivative{\Phi_B }{t} \right|
        \]
  \item Multiple loops induced EMF:
        \[
          \Delta V_{\mathrm{induced}} = N\left| \derivative{\Phi_B }{t} \right|
        \]
  \item AC generators:
        \[
          \Delta v = ABN\omega\sin{\omega t}
        \]
\end{itemize}

\section{Inductance}
Self-inductance:
\[
  L = \frac{\Delta V}{\left| \derivative{I}{t} \right|} = \frac{\Phi_B }{I}
\]

Inductance of a solenoid:
\[
  L = \frac{\mu_0 AN^2 }{l}
\]

RL circuits:
\begin{itemize}
  \item Time constant:
        \[
          \tau = \frac{L}{R}
        \]
  \item Decreasing current:
        \[
          I\left( t \right) = I_{\mathrm{max}}e^{-\frac{t}{\tau}} = \frac{\Delta V_R }{R}e^{-\frac{t}{\tau}}
        \]
  \item Increasing current:
        \[
          I\left( t \right) = I_{\mathrm{max}}\left( 1 - e^{-\frac{t}{\tau}} \right) = \frac{\Delta V_R }{R}\left( 1 - e^{-\frac{t}{\tau}} \right)
        \]
  \item Power dissipation:
        \[
          P_L = LI\derivative{I}{t}
        \]
  \item Energy stored in the magnetic field of and inductor:
        \[
          U_L = \frac{1}{2}LI^2
        \]
\end{itemize}

LC circuits with a charged capacitor:
\begin{itemize}
  \item Angular frequency:
        \[
          \omega = \sqrt{\frac{1}{LC}}
        \]
  \item Natural frequency:
        \[
          f_{\mathrm{natural}} = \frac{1}{2\pi}\sqrt{\frac{1}{LC}}
        \]
  \item Charge:
        \[
          Q\left( t \right) = Q_{\mathrm{max}}\cos{\omega t}
        \]
  \item Current:
        \[
          I\left( t \right) = - I_{\mathrm{max}}\sin{\omega t} = - \omega Q_{\mathrm{max}}\sin{\omega t}
        \]
\end{itemize}

\section{AC circuits}
Angular frequency:
\[
  \omega = 2\pi f = \frac{2\pi}{T}
\]

Instantaneous voltage:
\[
  \Delta v = \Delta V_{\mathrm{max}}\sin{\omega t}
\]

RMS:
\begin{itemize}
  \item Current:
        \[
          I_{\mathrm{RMS}} = \frac{\sqrt{2}}{2}I_{\mathrm{max}} \approx 0.707I_{\mathrm{max}}
        \]
  \item Voltage:
        \[
          \Delta V_{\mathrm{RMS}} = \frac{\sqrt{2}}{2}\Delta V_{\mathrm{max}} \approx 0.707\Delta V_{\mathrm{max}}
        \]
\end{itemize}

Resistor circuits:
\begin{itemize}
  \item Max voltage:
        \[
          \Delta V_{\mathrm{max}} = I_{\mathrm{max}}R
        \]
  \item Instantaneous current:
        \[
          i_R = I_{\mathrm{max}}\sin{\omega t} = \frac{\Delta V_{\mathrm{max}}}{R}\sin{\omega t}
        \]
  \item Instantaneous voltage:
        \[
          \Delta v_R = \Delta V_{\mathrm{max}}\sin{\omega t} = I_{\mathrm{max}}R\sin{\omega t}
        \]
  \item Average power dissipation:
        \[
          P_R = RI_{RMS}^2 = \frac{1}{2}RI_{\mathrm{max}}^2
        \]
\end{itemize}

Inductor circuits:
\begin{itemize}
  \item Inductive reactance:
        \[
          X_L = \omega L
        \]
  \item Max voltage:
        \[
          \Delta V_{\mathrm{max}} = I_{\mathrm{max}}X_L
        \]
  \item Instantaneous current:
        \[
          i_L = I_{\mathrm{max}}\sin\left( \omega t - \frac{\pi}{2} \right) = \frac{\Delta V_{\mathrm{max}}}{X_L }\sin\left( \omega t - \frac{\pi}{2} \right)
        \]
  \item Instantaneous voltage:
        \[
          \Delta v_L = - \Delta V_{\mathrm{max}}\sin{\omega t} = -I_{\mathrm{max}}X_L \sin{\omega t}
        \]
\end{itemize}

Capacitor circuits:
\begin{itemize}
  \item Capacitive reactance:
        \[
          X_C = \frac{1}{\omega C}
        \]
  \item Max voltage:
        \[
          \Delta V_{\mathrm{max}} = I_{\mathrm{max}}X_R
        \]
  \item Instantaneous current:
        \[
          i_C = I_{\mathrm{max}}\sin\left( \omega t + \frac{\pi}{2} \right) = \frac{\Delta V_{\mathrm{max}}}{X_C }\sin\left( \omega t + \frac{\pi}{2} \right)
        \]
  \item Instantaneous voltage:
        \[
          \Delta v_C = \Delta V_{\mathrm{max}}\sin{\omega t} = I_{\mathrm{max}}X_C \sin{\omega t}
        \]
\end{itemize}

RLC series circuits:
\begin{itemize}
  \item Impedance:
        \[
          Z = \sqrt{R^2 + \left( X_L - X_C \right)^2 }
        \]
  \item Phase angle:
        \[
          \varphi = \tan^{- 1}\left( \frac{X_L - X_C }{R} \right)
        \]
  \item Max voltage:
        \[
          \Delta V_{\mathrm{max}} = I_{\mathrm{max}}Z
        \]
  \item Instantaneous current:
        \[
          i = I_{\mathrm{max}}\sin\left( \omega t - \varphi \right) = \frac{\Delta V_{\mathrm{max}}}{Z}\sin\left( \omega t - \varphi \right)
        \]
  \item Instantaneous voltage:
        \begin{alignat*}{2}
          \Delta v_R & = I_{\mathrm{max}}R\sin{\omega t}                                 & = & \Delta V_R \sin{\omega t}   \\
          \Delta v_L & = I_{\mathrm{max}}X_L \sin\left( \omega t - \frac{\pi}{2} \right) & = & - \Delta V_L \cos{\omega t} \\
          \Delta v_C & = I_{\mathrm{max}}X_C \sin\left( \omega t + \frac{\pi}{2} \right) & = & \Delta V_C \cos{\omega t}
        \end{alignat*}
\end{itemize}

\section{Electromagnetic waves}
Maxwell's equations (in vacuum):
\begin{itemize}
  \item Gauss's law:
        \[
          \Phi_E = \oint{\vec{E} \bullet \diff\vec{A}} = \frac{q_{\mathrm{in}}}{\epsilon_0 }
        \]
  \item Gauss’s law for magnetism (magnetic monopoles do not exist):
        \[
          \Phi_B = \oint{\vec{B} \bullet \diff\vec{A}} = 0
        \]
  \item Faraday's law:
        \[
          \oint{\vec{E} \bullet \diff\vec{s}} = -\derivative{\Phi_B }{t}
        \]
  \item Ampere-Maxwell law:
        \begin{itemize}
          \item Displacement current:
                \[
                  I_{\mathrm{displacement}} = \epsilon_0 \derivative{\Phi_E }{t}
                \]
          \item Ampere-Maxwell law:
                \[
                  \oint{\vec{B} \bullet \diff\vec{s}} = \mu_0 I_{\mathrm{displacement}} + \mu_0 I_{\mathrm{through}}
                \]
        \end{itemize}
  \item Lorentz force:
        \[
          \vec{F} = \vec{F}_E + \vec{F}_B = q\vec{E} + q\vec{v} \times \vec{B}
        \]
\end{itemize}

Wave equations:
\[
  \npartialderivative{2}{E}{x} = \mu_0 \epsilon_0 \npartialderivative{2}{E}{t} \text{ and } \npartialderivative{2}{B}{x} = \mu_0 \epsilon_0 \npartialderivative{2}{B}{t}
\]
\begin{itemize}
  \item Speed of light equations:
        \begin{align*}
          c & = \lambda f = \frac{\omega}{k} = \frac{E}{B}                                   \\
          c & = \frac{1}{\sqrt{\mu_0 \epsilon_0 }} \approx \SI{2.9979 e8}{\metre\per\second}
        \end{align*}
  \item Solutions to wave equations:
        \[
          E = E_{\mathrm{max}}\cos\left( kx - \omega t \right) \text{ and } B = B_{\mathrm{max}}\cos\left( kx - \omega t \right)
        \]
\end{itemize}
\end{document}
